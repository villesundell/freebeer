% Copyright 2009 FSCONS, Superflex and the individual authors.
% This entire book and all its source files is licenced under Creative Commons Attribution-ShareAlike 2.5
\begin{savequote}
    \qauthor{\LARGE{Mike Linksvayer}}
\end{savequote}
\chapter{Free Culture in Relation to Software Freedom}
\label{c:free_culture_software_freedom}

Richard Stallman announced the GNU project (GNU's Not Unix) to create a free
operating system in 1983, making the free software movement at least 25 years
old\footnote{See \url{http://ur1.ca/f6pj} for my perspective on the 25th
anniversary of GNU.}. In a number of ways, free culture is harder to pin down
than free software. No single event marks the obvious beginning of the free
culture movement. Candidates might include the launches of the f\hbox{}irst Open
Content licences (1998\footnote{See ``10 Years of Open Content'' at
\url{http://ur1.ca/f6pm} by David Wiley, creator of the f\hbox{}irst open
content licence.}), Wikipedia (2001), and Creative Commons (2002).  One reason
may be that there is no free culture equivalent of a free operating system - an
objective that is clearly necessary, and for at least some people,
suf\hbox{}f\hbox{}icient to fully achieve software freedom.

This chapter compares and contrasts software and culture and the free software
and free culture movements. The ideas herein formed, with my observations as a
free software advocate working at Creative Commons for f\hbox{}ive years, then
took the form of f\hbox{}ive presentations on the topic during 2008\footnote{See
\url{http://ur1.ca/f6pp}, \url{http://ur1.ca/f6pr}, \url{http://ur1.ca/f6ps},
\url{http://ur1.ca/f6pv} and \url{http://ur1.ca/f6pw}.}. I gave the second to
last of those presentations at FSCONS (not coincidentally, a conference
dedicated to free software \textit{and} free culture), the book version of which
this chapter is being written for.

I start by examining dif\hbox{}ferences between software and culture \textit{as
they relate to the need for and ability to collaborate across individual and
organizational boundaries}, then move on to the implications of those
dif\hbox{}ferences for free software and free culture. Next I look at the
history of each movement and indicators of what each has achieved - mostly by
loosely analogizing free culture indicators to free software, the latter taken
as a given. F\hbox{}inally, I attempt to draw some lessons, again mostly for
free culture, and point out some useful ways for the free software and free
culture movements to collaborate.

In this chapter I take ``cultural works'' to mean ``non-software works of a type
often restricted by copyright''. Admittedly this is not perfect - software is
culture (as is everything of human construction in some sense), some
recognizably ``cultural'' works include software, and many non-software works
are not usually thought of as ``cultural''.

While plenty may be said about the relative properties of cultural and software
works usually recognized as such without creating precise def\hbox{}initions for
each set, it is worth noting that Stallman, at least since 2000, has delineated
three categories of works - functional (software, recipes, dictionaries,
textbooks), representative (essays, memoirs, scientif\hbox{}ic papers), and
aesthetic (music, novels, f\hbox{}ilms)\footnote{See \url{http://ur1.ca/f6px}
(speech transcription, 2000) and \url{http://ur1.ca/f6py} (interview, 2002).}.
Although Stallman’s evaluation of the freedoms required for representative works
has had some unfortunate ef\hbox{}fects\footnote{Verbatim-only permissions for
GNU essays on which I comment in another GNU 25th anniversary post at
\url{http://ur1.ca/f6q0} leading directly to an over-complicated Free
Documentation Licence with non-free options, discussed brief\hbox{}ly on The
Software Freedom Law Show: Episode 0x16 concerning documentation licensing; see
\url{http://ur1.ca/f6q1}.}, these categories are very insightful and have some
correspondence with my claims below that some cultural works more than others
share similarities with software.


\section{Obvious Software, Ubiquitous Culture}
\label{s:free_culture_software_freedom:obvious_software}


\subsection{Reuse}
\label{ss:free_culture_software_freedom:obvious_software:reuse}

The case for reusing software code is obvious, compelling, and pragmatic. If one
can use or improve existing code, it often makes sense to do so rather than
writing new code from scratch. For example, if one needed a HTML renderer, it
would be very dif\hbox{}f\hbox{}icult to justify starting over rather than using
Gecko or WebKit, the renderers used most notably by the F\hbox{}irefox and
Safari web browsers respectively, and also many other projects. On the other
hand, the case for reusing software code is very narrow. If one is writing a
device driver, code from an HTML renderer is useless, as is nearly all other
software code.

Any particular cultural reuse does not seem necessary. If one needs music for a
f\hbox{}ilm soundtrack, any number of existing pieces might work, and one would
hardly question a decision to create a new piece just for the f\hbox{}ilm in
question.  However, no particular cultural reuse is absurd, excepting when
absurdity is a cultural feature. Cat photos and heavy metal music can make a
music video. I challenge you to think of \textit{any} combination of artefacts
that some artist could not incorporate together in a new work.

Software is usually fairly clearly used in some part of a ``stack'' and an
entire stack forms a self-contained nearly universally multi-purpose whole -
usually an operating system with applications. Cultural works can of course be
layered, but don't sort naturally into a ``stack'' - a f\hbox{}ilm may need a
soundtrack in roughly the same way a song needs a video, while a video player
needs an audio codec, but not \textit{vice versa}. There is no cultural
equivalent of a shippable operating system.


\subsection{Maintenance}
\label{ss:free_culture_software_freedom:obvious_software:maintenance}

Maintenance of software is almost necessary. Unmaintained software eventually is
surpassed in features, becomes incompatible with new formats, has security holes
discovered, is not included in current distributions, is only runnable on
emulators, and if it is still useful, may be rewritten by a new generation of
programmers who can't understand or even can't f\hbox{}ind the code.
Non-maintained software is dead, or at least moribund.

A ``maintained'' cultural work is pretty special. Most are consumed verbatim,
unchanged from the artefact originally published, \textit{modulo} technical
medium shifts. This may be a primarily 20th century phenomenon - beginning
earlier for text, which could be mechanically reproduced on an industrial scale
earlier.  Arguably culture before mass reproduction required maintenance of a
sort to survive just as much as software does - manual copying since the dawn of
writing and repeated performance before that. It is possible to imagine a future
in which a lack of truly mass media and tremendously powerful and accessible
modif\hbox{}ication tools mean that in order to survive, a cultural work must be
continually modif\hbox{}ied to remain relevant. However, it is clear that at
least now and in the recent past, an old verbatim cultural work is at least
potentially useful, while old verbatim software work seldom is useful.


\subsection{Modifiable Form and Construction}
\label{ss:free_culture_software_freedom:obvious_software:modifiable_form}

Software's modif\hbox{}iable form is roughly all or nothing - you have the
source code or not. Some reverse engineering and decompilation is possible, but
clearly source code is hugely more useful than binaries for modifying -
including maintaining - software.

The modif\hbox{}iable forms of cultural works are varied and degradable. For
example, text with mark-up is more useful than a PDF, which is more useful than
a bitmap scan. Audio multi-tracks are better than a lossless mixdown, which is
better than a high bitrate mixdown, which is better than a low bitrate mixdown,
which is better than a cassette recording of an AM radio broadcast during a
storm. At the extremes, the most preferred form is much better than the most
degraded, but the degradation is fairly steady and all forms have potential for
cultural reuse.

The closest to such steady degradation for software source code might be that
commented code is better than uncommented code, which is better than obfuscated
code, which is better than binaries, which are better than obfuscated binaries -
but most of these forms are fairly unnatural - while it is hard to avoid
encountering most of the continuum of modif\hbox{}iable form degradation for
cultural works - except that the most preferred form is often unavailable.

Relatedly, there's a gulf in the construction of software and cultural works.
Creating software is identical to creating its modif\hbox{}iable form. Creating
cultural works often involves iteratively leaving materials on the cutting room
f\hbox{}loor or the digital equivalent.

It makes intuitive sense that that which does not degrade gracefully requires
maintenance and that which does not degrade gracefully does not require
maintenance, though it is unclear there is any causality in either direction.


\subsection{Distributed Collaboration}
\label{ss:free_culture_software_freedom:obvious_software:distributed_collaboration}

The compelling case to reuse specif\hbox{}ic software and the need to maintain
software means that individuals and organizations with similar needs are likely
to benef\hbox{}it from using the same software - and for some of them to work
together (closely or loosely) to maintain and improve the software.

Given lack of a compelling case for reusing specif\hbox{}ic cultural works and
the lack of need to maintain cultural works means the need to collaborate across
entity boundaries around \textit{a specif\hbox{}ic work} is much lower - though
there remains a strong desire to collaborate across entities around any number
of cultural works, and once a project that cannot be completed by a single
entity is under way or a work gains cultural signif\hbox{}icance, there can be a
very strong need or desire for distributed collaboration around that
specif\hbox{}ic project or work.


\subsection{Wikis}
\label{ss:free_culture_software_freedom:obvious_software:wikis}

Note that typical Wikis are somewhat like software in many of these respects.
They require maintenance so as not to become stale and overrun with spam. Reuse
may be more pragmatic and modif\hbox{}iable form more singular than most
cultural works. Wikipedia is much more like a self-contained nearly universally
multi-purpose whole than most cultural works.


\section{Freedom}
\label{s:free_culture_software_freedom:freedom}

What do these dif\hbox{}ferences in reuse, maintenance, and modif\hbox{}iable
form mean for free software and free culture, in particular the latter relative
to the former?  Much has been written about software freedom, and there is wide
agreement about what it entails. Distillations such as the Debian Free Software
Guidelines\footnote{\url{http://ur1.ca/f6q2}}, the Open Source
Def\hbox{}inition\footnote{\url{http://ur1.ca/f6q4}}, and the Free Software
Def\hbox{}inition\footnote{\url{http://ur1.ca/f6q5}} almost completely agree
with each other about which software is free (or open) and which is
not\footnote{See \url{http://ur1.ca/f6q6} for a rare exception.}.

Why software freedom? The Free Software Def\hbox{}inition's four freedoms state
(somewhat redundantly) things we want to be able to do with software - use, read
and adapt, share, and improve and share improvements. More abstractly, free
software grants users some autonomy (and the ability to get more), promotes a
sharing ethic, facilitates collaboration, unlocks value, reduces transaction
costs, makes distributed maintenance tenable, and arguably is congruent with and
facilitation of broader social goals such as access, participation, democracy,
innovation, security, and freedom\footnote{F\hbox{}ind a broad discussion of how
free software and similar phenomena further these liberal goals in The Wealth of
Networks by Yochai Benkler, available from \url{http://ur1.ca/f6q7}. I
highlighted the positive impact of free software and free culture on freedom and
security in particular in another FSCONS 2008 presentation, see
\url{http://ur1.ca/f6q8}.}.


\subsection{Software Services and Fee Software and Free Culture}
\label{ss:free_culture_software_freedom:freedom:software_services}

Software services delivered over a network have reignited the debate over what
constitutes necessary software freedom. No doubt the rise of software services
has aided and been helped by free software - the applications themselves are
often not free software, but are usually built of and on top of many layers of
free software, while the move of the most important applications to the web
means that free software users only really need a web browser to be on a par
with non-free users (there are important caveats, in particular the dominance of
patent-encumbered media codecs, but the web is fairly clearly an equalizer).
However, some see software services as a gigantic threat to software freedom.
Not only is the source to most popular applications unavailable and not freely
licensed, operations of software services are completely opaque, they have your
data, and could shut down or deny you access at any time!

Among the vanguard that sees a problem in software services and an answer in
more software freedom, there is broad agreement in outline, e.g., the Franklin
Street Statement\footnote{\url{http://ur1.ca/f6qa}; see \url{http://ur1.ca/f6qe}
for my perspective.} and Open Software Services
Def\hbox{}inition\footnote{\url{http://ur1.ca/f6qi}} probably would agree most
of the time on which services are free, but many details and a huge amount of
practise remains to be worked out\footnote{See \url{http://ur1.ca/f6qj} for
ongoing discussion of ``free network services.''}.

The Franklin Street Statement and Open Software Services Def\hbox{}inition each
recognize the need for content freedom. Private content makes things
interesting, but both broadly agree on what constitutes free cultural works.
Indeed, both build on def\hbox{}initions of freedom (or openness) for
non-software works that plainly map software freedom to cultural works, the
Def\hbox{}inition of Free Cultural Works\footnote{\url{http://ur1.ca/f6qm}} and
the Open Knowledge Def\hbox{}inition\footnote{\url{http://ur1.ca/f6qo}}
respectively.


\subsection{Definitions of Freedom for Culture}
\label{ss:free_culture_software_freedom:freedom:def_freedom_culture}

These def\hbox{}initions have gained considerable traction - the former is used
as Wikipedia's def\hbox{}inition of acceptable content licensing and is
recognized (reciprocally) with an ``Approved for Free Cultural Works'' seal on
qualifying Creative Commons instruments (public domain, Attribution,
Attribution-ShareAlike)\footnote{\url{http://ur1.ca/f6qp}}. In debates about
free culture licensing, it is regularly assumed and asserted that licences that
do not meet the translated standards of free software are non-free.

However, there is some explicit disagreement about whether freedom can be
def\hbox{}ined singularly across all cultural works or that non-software
communities have not arrived at their own def\hbox{}initions (Lawrence
Lessig\footnote{Discussed at \url{http://ur1.ca/f6qq}; also see Lessig
presentation at 23C3 available at \url{http://ur1.ca/f6qr} starting at 41
minutes.}) or that many cultural works require less freedom
(Stallman\footnote{Ibid. 4.}), to say nothing of graduated and multiple
def\hbox{}initions in related movements such as those for Open
Access\footnote{See \url{http://ur1.ca/f6qu} for an overview that unfortunately
uses ``libre'' to indicate that at least some permission barriers have been
removed, a much looser indicator than the standard of Free, Libre, and Open
Source Software, which requires that all permission barriers be removed, with
exceptions only for notice, attribution, and copyleft.} and Open Educational
Resources\footnote{See \url{http://ur1.ca/f6qv} for one conversation
demonstrating lack of consensus on freedoms required for Open Educational
Resources.}. More importantly, approximately two thirds of cultural works
released under public copyright licences use such licences that do not qualify
as free as in (software) freedom - those including prohibitions of derivative
works and commercial use\footnote{\url{http://ur1.ca/f6re}}.

Does culture need freedom? As in free software? I take this as a given until
proven otherwise, but the case for has not been adequately captured. The
Def\hbox{}inition of Free Cultural Works says ``The easier it is to re-use and
derive works, the richer our cultures become. \ldots These freedoms should be
available to anyone, anywhere, any time. They should not be restricted by the
context in which the work is used. Creativity is the act of using an existing
resource in a way that had not been envisioned before.''\footnote{Ibid. 14.} So
free as in software freedom culture is asserted to result in richer cultures.

The Def\hbox{}inition of Free Cultural Works maps the Free Software
Def\hbox{}inition's four freedoms for works of authorship to (1) the freedom to
use the work and enjoy the benef\hbox{}its of using it, (2) the freedom to study
the work and to apply knowledge acquired from it, (3) the freedom to make and
redistribute copies, in whole or in part, of the information or expression, and
(4) the freedom to make changes and improvements, and to distribute derivative
works\footnote{Ibid. 14.}.

It is easy to argue that free culture of\hbox{}fers many of the benef\hbox{}its
free software does, as enumerated above: grants users some autonomy (and the
ability to get more), promotes a sharing ethic, facilitates collaboration,
unlocks value, reduces transaction costs, makes distributed maintenance tenable,
and arguably is congruent with and facilitating of broader social goals such as
access, participation, democracy, innovation, security, and freedom. And could
lead to richer cultures.


\subsection{Why Semi-Free Culture?}
\label{ss:free_culture_software_freedom:freedom:semi-free}

So why the semi-freedom (relative to free as in software freedom) granted by
cultural licences that include terms prohibiting derivative works or commercial
use? Are such terms helpful or harmful to the free culture movement? I don't
know of any empirical work on why people use semi-free licences, but anecdotally
reasons include not wanting others to change the meaning of a work (derivatives
prohibition) and having a business model that depends on restricting commercial
uses or having feelings that are sensitive to anyone prof\hbox{}iting without
you being part of the deal (commercial use prohibition).

Prohibition of derivative works seems particularly misguided and
non-benef\hbox{}icial.  Misguided because free licences do have limited
mechanisms to restrict disagreeable uses - the licensee distributing a
derivative work must describe changes made and must not imply endorsement of the
licensor, while the licensor can mandate that credit be removed so they are not
associated with the derivative and (unfortunately) retains ``moral rights''
against derogatory uses (these vary in strength around the world). Furthermore,
given the diminution of fair use, fair dealing, and other copyright exceptions
(which tend to be weakest where moral rights are strongest), lack of explicit
permission to create derivative works is a free speech issue.

Most of the problems with prohibition of commercial use from a free culture
perspective are comparatively well
documented\footnote{\url{http://ur1.ca/f6qy}}.

While the problems of semi-free licences should not be underestimated, there are
some reasons for their existence, some reasons to think they are less
problematic for culture than they are for software (where they have been roundly
rejected) and some possibility that their impact is net positive.

Battles over f\hbox{}ile sharing are one reason. These may have reached their
peak relevance around the time Creative Commons launched in December, 2002
(since then the web has become the increasingly dominant platform for sharing -
and for media, period). People were (and are) getting sued simply for making
verbatim works available via f\hbox{}ile sharing at no charge and many
innovative P2P startups were shut down. Many in the copyright industries hoped
that DRM, a threat to computer users, civil liberties, and free software
specif\hbox{}ically, would render f\hbox{}ile sharing useless. In this
environment, merely allowing legal sharing of verbatim works would be a
signif\hbox{}icant statement against shutting down innovation and mandating DRM.

Because reuse of cultural works is non-pragmatic relative to reuse of software
code, it is possible that a derivatives prohibition on some cultural works is
less impactful than such a restriction would be on software. Lower requirements
for maintenance also mean that the importance of allowing derivative works is
lessened for culture.

Restrictions on f\hbox{}ield of use (namely, commercial use) may also be less
harmful for culture than they would be for software. Lack of interoperability is
one of the problems created by non-commercial licensing. However, if prohibiting
derivative works is less impactful in culture, so too are interoperability
problems, which are triggered by the inability to use derivatives created from
works under incompatible licences.

When distributed maintenance is important, non-commercial licensing is unusable
for business - a commercial anti-commons is created - no commercial use can be
made as there are too many parties with copyright claims who have not cleared
commercial use. This is perhaps one explanation of why free software $\cong$
open source - although the latter is seen by some as business-friendly, to the
detriment of freedom, businesses require full freedom, at least for software.

Maybe some artists want a commercial anti-commons: nobody can be ``exploited''
because commercial use is essentially impossible. If most of culture were
encumbered by impossible to clear prohibitions against commercial use, the
commercial sector disliked by Adbusters types would be disadvantaged. However, I
suspect very few licensors of\hbox{}fering works under a non-commercial licence
have thought so far ahead. Among those who have thought ahead, even those with
far left sympathies, seem to appreciate forcing commercial interests to
contribute to free culture \textit{via} copyleft rather than barring their
participation.

Many licensors do want to exploit commerce under fairly traditional models.
There is a case to be made that copyleft (e.g., ShareAlike) licences have an
under-appreciated and under-explored role in business models, but it certainly
requires less imagination to see how traditional models map onto only permitting
non-commercial use - the pre-cleared uses are promotional, while the copyright
holder authorizes sales of copies and commercial licensing in the usual manner.
While businesses based on selling copies of digital goods are cratering,
commercial licensing of digital goods (e.g., for use in advertisements) is a
huge business. I do not know what fraction of this business results in creating
derivatives of the works licensed, but it is at least possible that a
signif\hbox{}icant fraction does not, and hence ShareAlike may be a poor
business model substitute for commercial use prohibition.

By contrast, free commercial use is less impactful on the bulk of the software
industry, which is mostly about maintenance and custom development. While impact
on existing business models is not directly part of the calculus of how much
freedom is necessary, high impact on existing business models may drastically
limit willingness to use fully free licences. So while for software, semi-free
licences may compete with free licences (fortunately the latter won), for
culture semi-free licences may largely be used by licensors who would not have
of\hbox{}fered a public licence if only fully free licences were available,
meaning that semi-free licences produce a net gain. It is entirely possible that
many licensors of\hbox{}fering works under semi-free licences would have used
free licences if no prominent semi-free licences were available, producing a net
loss or ambiguous result from semi-free licensing. I hope social scientists
f\hbox{}ind a means of testing these conjectures with f\hbox{}ield data and lab
experiments.

Although the direct impact of prominent licence choices on the freedoms
af\hbox{}forded to cultural works is important, so is the indirect impact on
norms and movements. One complaint about semi-free licences is that they weaken
the consensus meaning of free culture - licensors can feel like they're
participating without of\hbox{}fering full freedom.

There is another, older consensus around ``non-commercial'' that doesn't have
much if anything directly to do with licences, that we could return to - that
non-commercial use should not be restricted by copyright, as the default. We are
a very long way from reaching such a consensus, but it would be a huge
improvement over the current consensus, that nearly all uses are restricted by
copyright. ``Huge'' is an understatement.

It is at least possible to imagine widespread adoption of public licences with a
non-commercial term as being an important component of a shift back to the
second kind of non-commercial consensus. If non-commercial public licences were
to have a positive role to play in this story, it seems two things would have to
be true: (1) many more people use non-commercial public licences than would
otherwise use public licences if only fully free public licences were available;
and (2) use of non-commercial public licences sets a norm for the minimum
freedom a responsible party would of\hbox{}fer rather than all the freedom
people need.  In other words, the expectation should be that if you don't at
least promise to not censor non-commercial uses, you're an evil jerk, but if you
only promise to not censor non-commercial uses, you're merely not an evil jerk.

As someone who strongly prefers fully free licences, I even more strongly prefer
to see ef\hbox{}fort put into building and promoting free cultural works rather
than bashing semi-free licences, for roughly three reasons: (1) use of semi-free
licences could have a positive impact, to the extent they don't crowd out free
licences (see above); (2) building is so much more interesting and fun than
advocacy, especially negative advocacy - in the history of free software, the
people who are remembered are those who built free software, not those who
sniped at shareware authors (roughly equivalent to semi-free licensors); and (3)
pure rationalization - as of this writing, I work for an organization that
of\hbox{}fers both free and semi-free public copyright licences.

It is unsurprising Stallman only supports cultural freedom necessary for free
software, rather than that which is necessary for building equivalently free
culture - software freedom is his overriding mission. Although he has not made
such a claim, and has a coherent explanation for why works of opinion and
entertainment do not require full freedom\footnote{Ibid. 4.}, there is a case to
be made that semi-free cultural licences do everything necessary to facilitate
free software, e.g., allowing format shifting (to non-patent encumbered formats)
and presenting a counter-argument to mandating DRM.

It should be noted that for some communities free as in free software is not
free enough, for example the Science Commons Protocol for Implementing Open
Access Data\footnote{\url{http://ur1.ca/f6r0}} claims that only the public
domain (or its approximation through waiving all rights that are possible to
waive) is free enough for scientif\hbox{}ic data.


\subsection{Copyleft Scope}
\label{ss:free_culture_software_freedom:freedom:copyleft}

Copyleft scope or ``strength'' is another theme that cuts across free software
and free culture, possibly dif\hbox{}ferently. In software, copyleft strength
ranges from zero (permissive licences) to limited (LGPL) to what most expect
(GPL) to including triggering by of\hbox{}fering an interface over a network
(AGPL). It is possible to imagine taking copyleft strength to an absurd limit -
a licence that only permits licensed code to run in a universe in which all
software in that universe is under the same licence.

For culture, copyleft strength depends on what constitutes an adaptation that
triggers copyleft (ShareAlike). For example, version 2.0 of the Creative Commons
licences explicitly declared that syncing video to audio creates a derivative
work\footnote{See \url{http://ur1.ca/f6r1} for a post announcing and explaining
changes in version 2.0 of the Creative Commons licences.}, and thus triggers
copyleft. There is debate concerning whether ``semantically linked'' images with
text triggers copyleft\footnote{See part of the debate at
\url{http://ur1.ca/f6r3}}.

If the goal is to expand free universe, optimal copyleft is where the
opportunity cost of under-use due to copyleft equals the benef\hbox{}it of
additional works released under free terms due to copyleft at the margin. Again,
there is an opportunity for social scientists to address this question, possibly
with f\hbox{}ield data, certainly with lab experiments.


\section{Relative Progress of Free Software and Free Culture}
\label{s:free_culture_software_freedom:relative_progress}

Given dif\hbox{}ferences between software and culture, one may expect free
software and free culture to progress dif\hbox{}ferently. One quick and dirty
means to gauge their relative development is to list the years of milestones in
each f\hbox{}ield, as I have done in the table below. These are certainly not
the best milestones for comparison - particular licences are over-emphasized -
the reader is urged to render this analysis obsolete by publishing better
analysis.

If crude analogies can be made between free software and free culture project
timelines, what do they indicate?

Perhaps the earliest massive community software project is Debian, started in
1993. Wikipedia began 8 years later, in 2001. Wikipedia's success came faster,
more visibly, and within the context of its f\hbox{}ield, far greater. Wikipedia
exploded the encyclopaedia category - comparison to previous encyclopaedias is
fairly ridiculous as Wikipedia is orders of magnitude bigger and excels for many
uses completely out of scope for an encyclopaedia, perhaps most obviously as a
database and current events tracker.

Debian is a very successful GNU/Linux distribution and an even more interesting
community, but has not remotely exploded the GNU/Linux distribution category,
let alone the computer operating system category. Nor has Ubuntu (2004), a
commercially supported distribution based on Debian, that has greatly increased
the market share of Debian-based distributions. In contrast, there has been some
commercial activity around Wikipedia content, it is uninteresting and
unimpactful relative to the main project. Wikia, a commercial wiki hosting
venture using the same MediaWiki software as Wikipedia, but not a substantial
amount of Wikipedia content, could be very roughly analogized to Ubuntu. Wikia
is successful, but not relative to Wikipedia.
\newpage

\begin{table}[h]
\label{t:free_culture_software_freedom:relative_progress:milestones}
\begin{tabular}{|p{7cm}|p{7cm}|}

\hline
\begin{center}\textbf{Free Software}\end{center} & \begin{center}\textbf{Free Culture}\end{center}\\
\hline

1983: Launch of GNU Project & 1998: Open Content Licence\\
1989: GPLv1, Cygnus Solutions & 1999: Open Publication Licence\\
1991: Linux kernel, GPLv2 & 2000: GFDL, Free Art Licence\\
1993: Debian & 2001: EFF Open Audio Licence, launch of Wikipedia\\
1996: Apache & Other early 2000s open content licences (some of them Free):
Design Science Licence, Ethymonics Free Music Public Licence, Open Music
Green/Yellow/Red/Rainbow Licences, Open Source Music Licence, No Type Licence,
Public Library of Science Open Access Licence, Electrohippie Collective's
Ethical Open Documentation Licence.\\
1998: Mozilla, ``open source'' term coined, IBM embraces Linux, other open source
software & 2002: OpenCourseWare, Creative Commons version 1.0 licences\\
1999: Cygnus acquired by Red Hat & 2003: PLoS Biology, Magnatune\\
2000: .com bubble peaks and pops, includes open source bubble & 2004: CC version 2.0 licences\\
2002: OpenOf\hbox{}f\hbox{}ice.org 1.0 & 2005: CC version 2.5 licences\\
2004: F\hbox{}irefox 1.0, Ubuntu & 2007: CC version 3.0 licences\\
2007: [A]GPLv3 & 2009: Wikipedia migrates to CC BY-SA\\
????: World Domination & ????: Free Culture\\

\hline
\end{tabular}
\caption{Selected free software and free culture milestones.}
\end{table}

Many of the licences from this period are described at \cite{culture-licenses}.


The canonical free software business is Cygnus Solutions (best known for work on
the GNU Compiler Collection, perhaps the most ``core'' software in the free
stack), started in 1989 and acquired by Red Hat in 1999. There is no canonical
free culture business, but Magnatune (a record label) has often been held up as
a leading example, started 14 years after Cygnus. Cygnus was acquired by Red Hat
in 1999, while Magnatune's long term impact is unknown. Unlike Cygnus, Magnatune
uses a semi-free licence (CC BY-NC-SA), so for some it may not even qualify as a
free culture business.

Wikitravel (collaboratively edited travel guides) is another early free culture
business - both a business success, having been acquired by Internet
Brands\footnote{See notice of the acquisition at \url{http://ur1.ca/f6r4} as
well as my comments at \url{http://ur1.ca/f6r5}. I also highly recommend
Wikitravel founder Evan Prodromou's advice for businesses involving community
wikis or other tools with ``WikiNature'' - see \url{http://ur1.ca/f6r6} and my
commentary at \url{http://ur1.ca/f6r8}.}, and using a fully free licence (CC
BY-SA).

Like Magnatune and unlike Cygnus, Wikitravel could not be said to be near the
``core'' of the free stack - probably because there is no such thing for
culture, excepting fundamentals such as human language and music notation that
fortunately reside in the public domain.

Another point of comparison is investment and resistance from major
corporations. In 1998 IBM's beginning of major investments in free software was
a business adoption landmark. No analogous major investments have been made in
free culture. Most large computer companies have now made large investments in
free/open source software. In 1998 Microsoft was a bitter opponent of free
software - many would say they still are\footnote{See for example
\url{http://ur1.ca/f6r9}.}. In 2009
Microsoft's public messages and its activities, including release of some
software under free licences, is considerably more nuanced than a decade ago. In
2009, big media still largely has its head buried in the sand - and continues to
randomly kick and punch its customers from this position. Could Microsoft's
\textit{animus} towards openness a decade ago, be loosely analogous to big
media's Neanderthalism today?


\subsection{Licence Deproliferation}
\label{s:free_culture_software_freedom:relative_progress:licence_deproliferation}

One dif\hbox{}ference in the development of free software and free culture not
fully revealed by the table above (because it only mentions versions of the GPL
for software licences) is that free culture has not experienced licence
proliferation as free software has - and has even experienced licence
deproliferation. In 2003 the author of the Open Content and Open Publication
licences recommended using a Creative Commons licence instead\footnote{David
Wiley discusses the history of the Open Content License and Open Publication
Licence at \url{http://ur1.ca/f6rb}.} and PLoS adopted the Creative Commons
Attribution licence. In 2004 the EFF's Open Audio Licence 2.0 declared that its
next version is CC Attribution-ShareAlike 2.0\footnote{See the Open Audio
License v2 at \url{http://ur1.ca/f6rd}.}. There have been no signif\hbox{}icant
new free culture licences since 2002. In June, 2009  Wikipedia and other
Wikimedia Foundation projects migrated from the FDL to CC Attribution-ShareAlike
3.0 as their main content licence\footnote{For my take on this migration see
\url{http://ur1.ca/f6rf} and \url{http://ur1.ca/f6rg}.}.

Presumably this dif\hbox{}ference is largely due to both free culture having had
the benef\hbox{}it of over a decade of free software learning - including
learning through making many new licences - and that a fairly well-resourced
organization, Creative Commons, was able to establish its central role as a
creator of free (and semi-free) culture licences relatively early in the history
of free culture licences. It should be noted that Creative Commons was able to
be relatively well-resourced early due to the pre-existing success of free
software - both because such success made Creative Commons' plan credible and
directly via donations from a fortune made in free software\footnote{Early
Creative Commons funding came from a foundation started by Bob Young, the
founder of Red Hat. See pp. 102-103 of Viral Spiral by David Bollier, available
at \url{http://ur1.ca/f6ri}.}.

However, some of the dif\hbox{}ference in proliferation may be due to the narrow
case for reuse of specif\hbox{}ic software and broad case for reuse of
specif\hbox{}ic culture.  Licence proliferation may actually be less harmful to
software than culture, since most combinations of software in a way that would
create a derivative work are absurd, while no such combinations of culture are -
so most of the time it doesn't matter that any given pair of software packages
have incompatible free licences. Still, licence incompatibility does especially
hurt free software when it does happen to be material, and proliferation guarded
against and compatibility strived for.


\section{How Free Can We Be?}
\label{s:free_culture_software_freedom:how_free}

Generally culture is much more varied than software, and the success of free
culture projects relative to free software projects may ref\hbox{}lect this. It
seems that free culture is at least a decade behind free software, with at least
one major exception - Wikipedia. Notably, Wikipedia to a much greater extent
than most cultural works has requirements for mass collaboration and maintenance
similar to those of software. Even more notably, Wikipedia has completely
transformed a sector in a way that free software has not.

One, perhaps the, key question for free culture advocates is how more cultural
production can gain WikiNature\footnote{\url{http://ur1.ca/f6rj}} - made through
wiki-like processes of community creation, or more broadly, peer
production\footnote{See \url{http://ur1.ca/f6rk} for one discussion of relevant
terminology.}. To the extent this can be done, free culture may ``win'' faster
than free software - for consuming free culture does not require installing
software with dependencies, in many cases replacing an entire operating system,
and contributing often does not require as specialized skills as contributing to
free software often does.

A question for those interested specif\hbox{}ically in free software and free
culture licences is what is the impact of dif\hbox{}ferent licensing approaches
- in particular semi-free licences, copyleft scope, and incompatibility and
proliferation. I don't think we have much theory or evidence on these impacts,
rather we hold to some ``just so'' stories and have religious debates based on
such stories. If we believe the use of dif\hbox{}ferent licences have
signif\hbox{}icantly dif\hbox{}ferent impacts and we want free software and free
culture to succeed, we should really want rigorous analysis of those impacts!

One f\hbox{}inal point of comparison between free software and free culture -
how free can an individual be? Now it is just possible to run only free software
on an individual computer, down to the BIOS if one selects their computer very
carefully. However, visit almost any web site and one is running non-free
software, to say nothing of more ambient uses - consumer electronics, vehicles,
electronic transactions, and much more. Similarly one could only have free
cultural works on a computer\footnote{I don't know anyone who does this
consciously, which perhaps indicates the hard-core free software movement also
leads the hard-core free culture movement - there are many people who try very
hard to only run free software on their computers. For the record on my computer
I run Ubuntu, which is close to but not 100\% free and my cultural consumption
consists of a higher proportion of free cultural works than does anyone's I
know, though nowhere near 100\% - e.g., see \url{http://ur1.ca/f6rl} or
\url{http://ur1.ca/f6rm} for data on my music consumption.} (not counting
private data), though visiting almost any web site will result in experiencing
non-free cultural works, which are also ambient to an even greater extent than
is non-free software. My point is not to encourage living in a cave, but to
elucidate further points of comparison between free software and free culture.

One f\hbox{}inal question of broad interest to people interested in free
software or free culture - how can these movements help each other? What are the
shared battles and dependencies?\footnote{For example, see
\url{http://ur1.ca/f6rn}.} Knowledge sharing and dissemination is an obvious
starting point. To the extent processes or conceptions of freedom are similar,
learnings and credibility gained from successes (and learnings from failures)
are transferable.

We should set high goals for free software and free culture. Freedom, yes. We
should also constantly look for ways freedom can enable ``blowing up'' a
category, as Wikipedia has done for encyclopaedias. The benef\hbox{}it to
humanity from more freedom should not just be more freedom (or, per an
uncharitable rendering of the open source story, only fewer bugs), it should
include radically cool, disruptive, and participatory tools, projects, and
works. \textit{King Kong}, sometimes shorthand for expensive Hollywood
productions that free culture can supposedly never compete with - this is far
too low a bar!

