\begin{savequote}
    \qauthor{\LARGE{Henrik Moltke}}
\end{savequote}
\chapter[RMS on FREE BEER]{RMS on FREE BEER\\ \Large{Transcribed by Gunhild Andersen}}
\label{c:rms_free_beer}

\paragraph{HM:}{Hello, my name is Henrik. I'm calling on behalf of
Superf\hbox{}lex \dots}

\paragraph{RMS:}{Sorry, you said super-what?}

\paragraph{HM:}{Superf\hbox{}lex.}

\paragraph{RMS:}{I don't recall that name.}

\paragraph{HM:}{Do you remember the Free Beer?}

\paragraph{RMS:}{Yes!}

\paragraph{HM:}{What we hoped to do with you was to ask you to taste and review
the beer, which is \ldots}

\paragraph{RMS:}{It wouldn't work, because I don't like beer. I also don't like
the emphasis that most people put on getting drunk. I have only got drunk once
in my life, on a transatlantic f\hbox{}light. I had made the mistake of putting
my sleeping pills into my suitcase which I'd checked. I tried using whiskey to
achieve the same ef\hbox{}fect. It didn't work very well, partly because it was
so disgusting I could hardly swallow it.}

\paragraph{HM:}{Did you manage to sleep in the end?}

\paragraph{RMS:}{I slept a little bit.}

\paragraph{HM:}{But I was thinking that maybe we could try and do something
remotely similar to a review, just without actually talking about the taste and
the hue and the \dots}

\paragraph{RMS:}{OK!}

\paragraph{HM:}{So if you could pretend that you were reviewing this idea of a
free beer \dots}

\paragraph{RMS:}{Oh, I love the idea as long as I don't have to drink it!}

\paragraph{HM:}{I was wondering about the name, because most people will think
about this only as free beer in the free beer sense \dots}

\paragraph{RMS:}{\dots Well,}

\paragraph{HM:}{\dots but there is another \dots}

\paragraph{RMS:}{\dots are you selling samples of it?}

\paragraph{HM:}{Well, actually we do sell free beer in a shop, but we also
\dots}

\paragraph{RMS:}{Yeah, I hope so! It probably costs you money to produce a
batch.}

\paragraph{HM:}{Exactly.}

\paragraph{RMS:}{So it makes sense to sell bottles of it, or glasses of it. And
so that will make people think: they'll see this is free in the sense of
freedom, but it's not gratis.}

\paragraph{HM:}{Exactly, that was the concept from day one \dots}

\paragraph{RMS:}{Mmm?}

\paragraph{HM:}{So, do you have anything against or for naming a beer Free
Beer?}

\paragraph{RMS:}{I like the idea, because it's a cute way of making a point.}

\paragraph{HM:}{And could it be called a hack in the sense of \dots}

\paragraph{RMS:}{Yes! Yes, it is a hack. Playful cleverness is hacking, so this
is hacking.}

\paragraph{HM:}{I remember that we received an email with some very constructive
comments about intellectual property and the way we use \dots}

\paragraph{RMS:}{Well, actually, my comments may have been about quote
``intellectual property'' \dots}

\paragraph{HM:}{Exactly.}

\paragraph{RMS:}{\dots unquote, because I never talk about - I never use that
term \dots}

\paragraph{HM:}{And that's what you were telling us.}

\paragraph{RMS:}{\dots to describe anything, and it's a mistake to do so because
that term mixes together various dif\hbox{}ferent laws with totally
dif\hbox{}ferent ef\hbox{}fects as if they were a single thing. So anyone who
tries to think about the supposed quote ``issue of intellectual property''
unquote is already so badly confused that he can't think clearly about it.}

\paragraph{HM:}{Now, in the same email you also suggested that we call the beer
a free software beer instead of an open source beer.}

\paragraph{RMS:}{Yes. I founded the Free Software movement, and ``open source''
is a term used to co-opt our work; to separate our work from our ideals that
motivated it. See, we developed software that users are free to run and share
and change as they wish, for the sake of freedom. Because those freedoms, we
believe, are essential. Then there were millions of people who appreciated the
software and appreciated being able to share and change it, and found that it
was very good software too. But they didn't want to present this as an ethical
issue. So they started using a dif\hbox{}ferent term, open source, as a way to
describe the same software without ever bringing it up as an ethical issue: as a
matter of freedoms that people are entitled to. Well, they're entitled to their
opinions. But I don't share their opinions, and I hope you don't either. So to
support awareness of the ethical issues of free software the most basic thing to
do is talk about free software.}

\paragraph{HM:}{Do you think this will come about by discussing for example a
beer that actually isn't software?}

\paragraph{RMS:}{It's a similar kind of issue arising here. A beer doesn't
actually have source code either. A recipe is not like source code, you can't
just compile it. There's no program that turns the recipe into food.}

\paragraph{HM:}{What if we speak about the general idea of taking ideas from the
free software movement, and from the open source movement even, and transferring
those values onto something which is not software?}

\paragraph{RMS:}{I'm all in favour of it. Whenever they're applicable. When
these ideas make sense in one context they may make sense in another context,
but that's not guaranteed. They're not applicable to everything in life, they're
applicable to certain things. Specif\hbox{}ically, they're applicable when there
are works made of information that are useful.}

\paragraph{HM:}{So where do you draw the line? Does an open source cook book
make more sense than an open source car?}

\paragraph{RMS:}{I'd rather not use the term open source. I'm not a supporter of
the open source movement.}

\paragraph{HM:}{I'm sorry. That's the problem: if \dots}

\paragraph{RMS:}{Recipes should be free.}

\paragraph{HM:}{But I was thinking, is there a way that we could use this word
in a better way than speaking about an open source beer? Because a free software
beer also sounds strange.}

\paragraph{RMS:}{Yes, they both are strange. Neither one really f\hbox{}its
because a beer is not software and has no source. So if you're going to strain
things to refer to a movement, you might as well pick the movement you support.}

\paragraph{HM:}{Because we've taken a bit from one and a bit from the other.}

\paragraph{RMS:}{Anyway.}

\paragraph{HM:}{We tried to recount the whole story of what happened in the
early seventies up till now to sort of explain what the idea of the beer was,
and I f\hbox{}ind this quite complex.}

\paragraph{RMS:}{It is!}

\paragraph{HM:}{Is there any way that these kinds of ideas could travel to the
minds of people in an easier way?}

\paragraph{RMS:}{Well, I f\hbox{}ind that recipes make a good analogy for
explaining the ideas of free software to people. Because people who cook
commonly share recipes and commonly change recipes, and they take for granted
that they're free to cook recipes when they wish. So imagine if the Government
took away those freedoms; if they said ``starting today, if you copy and share,
or if you change a recipe, we'll call you a pirate.'' Imagine how angry they
would be. Well that anger, that exact anger, is what I felt when they said I
couldn't change and share software any more. And I said ``No way, I refuse to
accept that.''}

\paragraph{HM:}{Why do you think this had to happen within software and
computers, why haven't people demanded the same kind of freedoms before?}

\paragraph{RMS:}{Well, there weren't enough people using computers, and in the
early days software was free, actually.}

\paragraph{HM:}{Yeah. When you started \dots}

\paragraph{RMS:}{It was in the seventies that software became proprietary. And
that change for the worse was complete by the early eighties. But I had had the
experience of participating in a community of programmers where sharing software
was normal. And when it disappeared and died, and I saw a morally ugly way of
life as my probable future I rejected that.}

\paragraph{HM:}{That was back in the beginning of the eighties?}

\paragraph{RMS:}{That was in 1983. I formed the Free Software Movement and
launched a plan to develop a free software operating system so that we could use
computers and have this freedom.}

\paragraph{HM:}{Do you think that the way that things are now and the way that
you have a GNU/Linux option or you can do many things with dif\hbox{}ferent
kinds of open source software \dots}

\paragraph{RMS:}{Please?}

\paragraph{HM:}{I'm sorry, I'm sorry.}

\paragraph{RMS:}{I don't want you to use the term open source.}

\paragraph{HM:}{I'm very sorry.}

\paragraph{RMS:}{It's not what I stand for. You're putting me in a very bad
position by talking with me about my work and using the term, the name of a
party that was formed to reject my views.}

\paragraph{HM:}{This is something very dif\hbox{}f\hbox{}icult for someone like
me to actually - because I am not a computer programmer. I am not somebody who
has lived this for 20 years. So for me it is dif\hbox{}f\hbox{}icult although
I'm trying to \dots}

\paragraph{RMS:}{Think of open source and free software as the name of two
dif\hbox{}ferent political parties \dots}

\paragraph{HM:}{I fully understand that.}

\paragraph{RMS:}{\dots with dif\hbox{}ferent programmes. If you invited the
leader from the Green party - which, by the way, I more or less support - and
you started talking to him about his work in the Conservative party, and you did
that several times, he'd probably get mad at you.}

\paragraph{HM:}{And I could imagine that this is something that happens often
with the political press and journalists and \dots}

\paragraph{RMS:}{Yes. Yes it does, and in fact before I give an interview I
raise this issue and I make sure that they've agreed not to do this. Because it
would be pointless to do an interview if I'd be misreported as a supporter of
open source.}

\paragraph{HM:}{Well, you know, I actually did my homework, and this is
something that I f\hbox{}ind must be as dif\hbox{}f\hbox{}icult for ordinary
people \dots}

\paragraph{RMS:}{It's not that dif\hbox{}f\hbox{}icult. You're talking about
changing a habit.  It takes a little bit of work and you make mistakes a few
times but don't exaggerate it. You can change a habit.}

\paragraph{HM:}{When you started the Free Software Movement and the GNU project,
would you ever have imagined that this kind of idea would turn into something
outside of the computer world, something like a beer or \dots}

\paragraph{RMS:}{No, I didn't think for a minute about that.}

\paragraph{HM:}{When did that start happening, when did you start seeing those
possibilities?}

\paragraph{RMS:}{About f\hbox{}ive years ago.}

\paragraph{HM:}{Is that what you hope will happen in the future from now on?}

\paragraph{RMS:}{Well, I hope so. But mainly what I'm hoping for and working for
is that software should be free.}

\paragraph{HM:}{And do you think a project like this will help?}

\paragraph{RMS:}{Yes. It'll help. It will bring the ideas home to people who
wouldn't have thought about them otherwise. And that's useful.}

\paragraph{HM:}{I hope this will get some repercussions and that we may use this
\dots}

\paragraph{RMS:}{Happy hacking!}

\paragraph{HM:}{And thanks very much for your time!}

\paragraph{RMS:}{Bye.}

\paragraph{HM:}{OK, bye bye.}
