% Copyright 2009 FSCONS, Superflex and the individual authors.
% This entire book and all its source files is licenced under Creative Commons Attribution-ShareAlike 2.5
\begin{savequote}
    \qauthor{\LARGE{Nikolaj Hald Nielsen}}
\end{savequote}
\chapter[From Consumer to Creator]{From Consumer to Creator\\ \Large{The Lego Generation in the Digital Age}}
\label{c:consumer_to_creator}

\section{Intro}
\label{s:consumer_to_creator:intro}

I spent much of my childhood playing with Lego. My parents were never at a loss
for what to get me for my birthday. While soft packages were scorned, the hard,
box-shaped packages with that very special sound when you shake them were always
a hit. I quickly outgrew building f\hbox{}ixed models based on other people's
ideas and started exploring the boundaries of what could be achieved with my
imagination and my, unfortunately not as large as I would have wanted,
collection of bricks. I would spend long afternoons building a spaceship that
could transform itself into a moon base once it had landed, castles
f\hbox{}illed with secret rooms and traps, or weird machines that did a whole
lot of nothing, but looked very cool doing it. Once built, I quickly lost
interest though. For me, the fun part was not so much in playing with the things
I built, as the creative process of actually building them. I know I was far
from the only one.

Today I have replaced the Lego bricks with something else. Instead of small
pieces of plastic, I am piecing together virtual building blocks of code on a
computer screen. The basic desire to create, to use my mind and my hands to
build something that no one else has done before is the same, however, the
satisfaction when my ideas slowly become real no less exhilarating. There are
important dif\hbox{}ferences though. Whereas in my childhood, building my Lego
contraptions was mostly a solitary activity, today I am working with like-minded
creators, accomplishing together what we could never hope to achieve on our own.
And we are doing this in a spirit of openness and freedom, sharing the results
of our labour, our software, freely with each other and the rest of the world. 

Thanks to the ideas that were f\hbox{}irst formalized with Stallman's definition
of Free Software\footnote{See \url{http://ur1.ca/f6q5}},
which have long since spread into other areas, such as Free Culture, we now have
a conceptual and legal framework in place to foster this kind of collaboration
and creative process, and the results are starting to show in a very big way. 

For people who, like me, have grown up spending a great deal of time dreaming up
crazy new ideas and trying to make them real with their hands and a
f\hbox{}inite number of bricks, the role as a consumer is not a natural
f\hbox{}it. The notion of always receiving the creative works of others, only
being allowed to play with the toys that others have built, feels strange. Yet
this is how, for a large part, modern society works. A relatively small number
of creators of software and culture try to convince us that their latest
of\hbox{}fering is what will make us happy, at least until the next big thing
comes along. To make matters worse, the companies whose business is dependent on
people constantly ``consuming'' their virtual goods have seen it in their best
interest to start locking down their content by ever more sophisticated
technical and legal means designed to make tinkering impossible. This is the
digital equivalent of buying a Lego set that is not only pre-built, but where
the pieces have been glued together.

The reasons why companies claim a need to lock down their contents are many,
piracy being not the least. This discussion, and whether the countermeasures
actually make economic sense, is a very large discussion all by itself that is
better left for others with more knowledge of the area. One big issue I do see
is that the companies value a creative work dif\hbox{}ferently from society as a
whole.  For a record company or book publisher, value is proportionally related
to the ability to monetize a given work. For society at large, the value of a
creative work is something else completely, and something that is much harder to
quantify. How do you determine the cultural value of a creative work? It would
seem logical that cultural value is related to how many people come into contact
with the work and how many new ideas it contains. But perhaps more importantly,
a great indicator of a work's cultural value is how much it is referenced,
quoted and perhaps even remixed\footnote{See \url{http://ur1.ca/fcu2}} (to
borrow a term from Lessig) into derivative works, thus becoming a part of
Culture in general. Based on this, it is my strong belief that the more
controlled a creative work is, the less its cultural value will be as it becomes
harder (or the barrier of entry becomes greater) to remix the work and integrate
it with other works and other ideas in our shared cultural heritage.


\section{Making the bricks play sound}
\label{s:consumer_to_creator:play_sound}

My current involvement in Free Software is centred around the popular *nix (and
slowly moving on to other platforms as well) audio player and manager, Amarok
2\footnote{See \url{http://ur1.ca/fcu4}}. This is something I am quite passionate
about as it is not only an outlet for my own creativity and that of the other
authors and contributors, but it also strives to be a hub that can help bring
other forms of freely licensed creative content to a greater audience.

Much of my understanding of, and appreciation for, the areas of Free Software,
Free Culture and indeed the greater issues of Free Society comes from my work on
this project, so it is only natural for me to explore these issues through this
lens.

One of the basic premises behind Amarok 2 is that there is really no lack of
high quality free content out there on the web (or in ``The Cloud'' as the
fashionable term seems to be these days). The main challenge is making people
aware of its existence. Whether you are an ``up an coming'' band, radio station,
record label or indeed producer of nearly any kind of cultural content not
inside the ``mainstream media'', one of your worst enemies is obscurity. With
the vastness of the Internet, how do you get people to pay attention to you? You
have to make yourself discoverable.

Amarok tries to accomplish this by making it easy to tie content from nearly any
source into the core desktop application experience. Many of these sources will
have content licensed under Creative Commons or similar licences, but this is
not a strict requirement for inclusion of a service into Amarok. By making
content available in a consistent way, and possibly tying content from multiple
dif\hbox{}ferent sources together, the entire experience of discovering new
content is greatly simplif\hbox{}ied. With the enormous potential audience, even
the more obscure or experimental content, as long as the quality is high, is
likely to f\hbox{}ind a signif\hbox{}icant audience.

An example of a source that is now integrated, and the one that actually got
this idea started, is Magnatune.com\footnote{See \url{http://ur1.ca/fcu5}}.
Magnatune.com is a record label that tries to do ``fair trade'' music, treating
both artist and customers with respect. One of the things this means is that
customers should be able to listen, in full, to any album before deciding
whether to purchase it or not. Magnatune.com not only provides these preview
streams for all their content, but also a structured way of getting access to it
from third-party applications. So within Amarok, it is possible not only to
browse and listen to each and every album from Magnatune.com freely, as much as
you like, but also make purchases directly from within the application. Many
other Free Software applications have now included the Magnatune.com content as
well, making it a classic case of ``if you free it, they will come''.

Amarok 2 includes many other sources of content already, such as
Jamendo.com\footnote{See \url{http://ur1.ca/fcu6}},
LibriVox.org\footnote{See \url{http://ur1.ca/fcu7}} and others. So as soon as a new
user launches Amarok, these are immediately available. Perhaps much more
powerful than this however, Amarok 2 provides the ability for people to add
their own content in a relatively simple way.

One of the key issues to adoption of a scheme like the Amarok 2 service
framework is the barrier to entry. In order to spur adoption, this should
naturally be as low as possible. In an attempt to overcome this, Amarok 2 makes
it possible for third parties to add services using simple scripts. This means
that with very little knowledge of code, it is possible to add content to
Amarok. Coupled with Amarok's integrated system for downloading new ``service
scripts'', this is a potentially very powerful feature.


\section{Celebrating Diversity}
\label{s:consumer_to_creator:diversity}

To be completely honest, the possibility of adding services to Amarok using
scripts did not start out as a grand vision of empowerment. Few such things do.
But as the work progressed and interested people started contributing scripts,
even before Amarok 2 was ever of\hbox{}f\hbox{}icially released, it started to
become clear that it had great potential.

A concept that has become quite clear to me lately is that though some content
might be limited in its scope of appeal, due to language, topic, genre or a host
of other reasons, this does not make it collectively less important. In fact,
the sum of people interested in content like this might well exceed the number
of people interested in some of the services with more broad appeal that are
already integrated. This is in essence the idea of the ``long
tail''\footnote{See \url{http://ur1.ca/fcub}}.

There are however two main issues with ``narrow'' content of this kind.
F\hbox{}irst of all, it is unlikely that any of the regular contributors to a
project like Amarok will be motivated in adding sources of content far outside
their own areas of interest. Secondly, including content that is too narrow in
the default installation is not desired. 99\% of the users are not likely to
care much about Danish radio stations, and having too large a list of services
installed by default is likely to cause confusion. Also, everything that is
included in the default install will have to be maintained by the Amarok
developers, taking time away from other development work. This is where the
scripted services really show their worth.

Using the scripted service framework, people have already created a host of
services for national radio stations, access to the BBC's and NPR's archives of
freely available (but unfortunately not always freely licensed) materials, a
service for a site running a monthly vote of the best Free music, and the
aforementioned LibriVox service (which is included in the default distribution
as an example of what is possible using scripts). All of these services can be
browsed and installed from within Amarok and the content becomes instantly
available.

Having localized or niche content easily available in an integrated form is
interesting in a number of ways. Generally, in the Free Software and Free
Culture movements, we have a tendency to be very Anglocentric. That is, most
development work takes place in English, and this spills over into the kinds of
content that we generally include in the standard distribution of an application
like Amarok. For many people though, who speak poor or no English (or simply
have no interest in English language content) this makes the application less
appealing. The availability of third party scripted services providing easy
access to local content, such as local or regional radio stations, can
potentially do much to overcome this issue, making Amarok feel more ``native''
to non-English users. For instance, having the service providing a comprehensive
list of Danish radio stations would be a great selling point for my parents,
who, even though they speak perfectly f\hbox{}ine English, generally only listen
to Danish radio. And getting Amarok into the hands of more users expands the
potential audience for the other integrated services, not the least of which is
the Free Culture based ones. This example is based solely on my own work with
Amarok and the integrated services, but the underlying mechanics apply far
beyond this limited scope.

Which neatly brings me back to the Lego bricks.


\section{Empowerment}
\label{s:creator_to_consumer:empowerment}

One of the truly great things I see in the advent of Free Software and Free
Culture is that it is getting a nearly unlimited amount of interesting bricks
into the hands of creative people to build even more interesting stuf\hbox{}f.
This overcomes many of the f\hbox{}inancial and social barriers of entry that
have traditionally made it dif\hbox{}f\hbox{}icult or impossible for
``ordinary'' people to create and disseminate high quality cultural works,
software and so on, without the backing of a large corporate entity. The
f\hbox{}low of culture, traditionally one way from the few to the many, is
becoming much more many to many, peer to peer.  While this new wave of
peer-generated content might not supplant the traditional media industry any
time soon, the amount and quality of Free Culture and Software available has
long since reached the tipping point of becoming a viable alternative to many
people in many cases. You can now run your computer using only Free Software and
have a very functional setup, and you can have a life f\hbox{}illed with great
music from one of the many online sources of freely licensed music.

For most, this creation of new culture will be unpaid, but the instinct to
tinker and the gratif\hbox{}ication of being a creator and not merely a consumer
is a great motivation for many. And of course, as with all other things, the
people who are most skilled will f\hbox{}ind ways to make money from their
works, even if they are freely licensed.

I don't know what it will take to create a truly free society, but I have no
doubt that a large amount of Free Culture and Free software ``bricks'' will go a
very long way!

