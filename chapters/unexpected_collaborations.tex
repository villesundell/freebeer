\begin{savequote}
    \qauthor{\LARGE{Victor Stone}}
\end{savequote}
\chapter{Unexpected Collaborations}
\label{c:unexpected_collaborations}


\section{Introduction}
\label{s:unexpected_collaborations:introduction}

In late 2004, I started work as an independent contractor for Creative Commons
(CC)\footnote{Creative Commons is a non-prof\hbox{}it intellectual property advocacy
group that provides tools for content authors to make it easier to share their
works. Chief amongst these tools is a set of pre-authored licences that signify
to the artists' Web audience, which part(s) of their copyright they are willing
to suspend. The ccMixer project is a rare case where they actually host 3rd
party content (music) on a Web site.\\ \url{http://ur1.ca/fdui}} on a
website that would be called \textit{ccMixter.org}. I am the project lead which
means developer and site administrator and I am also a musician on the site,
with the \textit{nomme de Web} of ``fourstones''. 

The ccMixter project is not a f\hbox{}inancial enterprise. The goal of the project was
to drive adoption of the CC licences with musicians in the same way they had
been embraced in other publishing media, such as blogs and photography, and to
provide a concrete example of the benef\hbox{}its of freewheeling re-use.

Working together with WIRED Magazine, CC made a big splash into the music world
in November of 2004\footnote{Thomas Goetz ``Sample the Future'' November 2004
\url{http://ur1.ca/fduk}}. A CD featuring CC licensed music by Beastie Boys, My
Morning Jacket, David Byrne, Chuck D and others was bundled with that month's
WIRED magazine and a remix contest, hosted on the new site ccMixter, was
announced\footnote{Matt Haughey - Creative Commons blog, ``Wired CD tracks
online, and CC Mixter, our new remix community site, launched'' November 11th, 2004
\url{http://ur1.ca/fduo}}. The site outlived the contest and continues to allow uploads of CC
licensed music. The total impact is incalculable, but four years later there are
millions of pieces of audio on the Web under CC licences, so in that sense, the
project can be viewed as a success\footnote{CC Content Directories ``Audio'' section
\url{http://ur1.ca/fdup}}.


\section{On Collaboration}
\label{s:unexpected_collaborations:collaboration}

Many music collaboration sites have sprung up in the last few years, including
several that incorporate Creative Commons licences. Most employ the virtual
version of the met-at-a-bar-jammed-in-the-garage model of musicians getting
% TODO ``proffer'' ?!
together. Typically a songwriter will prof\hbox{}fer an a cappella and post a
request for collaborators with specif\hbox{}ic requests such as ``this track needs a
bass part'' or ``help me punch up the chorus''. Willing musicians will sign up to
collaborate and the group will exchange f\hbox{}iles in a project-based user model.

To be completely subjective and provocative I will say that the vast majority of
these musical projects leave much to be desired. While the social aspects are
very reassuring for many musicians, this way of working online exposes some
fundamental f\hbox{}laws:

\begin{enumerate}
    \item{Most successful collaborations are the result of musicians who have
        been playing together for many, many years, learning each others'
        musical vocabulary, making micro-corrections to their own playing in
        real-time. Other successful collaborations are based on a common
        expertise between the musicians, such as a deep knowledge and virtuosity
        within the conf\hbox{}ines of a well-understood, specif\hbox{}ic genre. F\hbox{}inally, there
        is a class of musicians who are trained in the art of accompaniment.
        They are specialists who make split second, spontaneous, ref\hbox{}lexive
        decisions based on vigorous training:  they can follow a singer deep
        into the weeds. Otherwise, face-to-face collaboration is wholly
        overrated. We think it works so well because when it works it is a
        magical experience for everybody involved. However, for every inspired
        collaboration there are literally millions that never leave the garage
        (and don't, thankfully).}
    \item{Explicit collaboration on the Web shines a glaring spotlight on any
        weakness existing between f\hbox{}irst-time collaborators. Most collaborations
        are painful, artistic disasters and taking those out of the garage and
        exposing them on the Web only makes the case. All of the mis-steps that
        are part of the natural process of an evolving collaboration, that would
        normally be hidden away in private, are exposed for everybody to see.
        It's the equivalent of putting a 24 hour web-cam into a sausage
        factory's R\&D lab. }
    \item{F\hbox{}inishing a collaboration is a serious, disciplined chore. Most of
        those in real life (and therefore on the Web) are interrupted by real
        life commitments and therefore never reach a satisfying level of
        completion.}
    \item{Collaborators regularly \textit{settle} for parts (backing tracks as
        well as vocals) because of time and closure pressures mentioned above
        but also because of social issues. How many times can you iterate with a
        bass player who is cheerfully volunteering his time and energy but who
        is, alas, continually giving you lousy bass parts? The vast majority of
        musicians I know are way too nice to be Simon Cowell about it and say,
        ``Sorry, thanks for the ef\hbox{}fort but you suck.''}
\end{enumerate}

Roughly two years after the ccMixter project got under way, several community
members put pressure on me to enable these types of explicit collaborations. I
took a survey of features at sites that specialized in such things and within a
few weeks turned on the ``Collaboration'' feature at ccMixter. Not surprisingly,
the feature suf\hbox{}fered from all the ailments I outline above. Additionally, its
presence caused confusion on the site about how to engage other musicians. A
year and a half after I had enabled the feature, the vast majority of
collaboration projects were started by newcomers who did not understand the
sample pool model of collaborating, which is primary to the site. (There was
also a fair amount of abuse of the feature: by the end, more spam type projects
were being created than legitimate ones.)

Taking luxuriant advantage of being a purist, non-prof\hbox{}it site, I f\hbox{}inally removed
the feature. With only about 20 completed collaboration projects (compared to
over 7,500 remixes) it seemed reasonable. Some consternation arose about the
method I used to discontinue the feature (I gave a few weeks' notice on the
site's forum) but no other hue and cry ensued. A commercial entity or one solely
interested in pumping up the membership numbers may have addressed any newcomer
confusion head on. They may have accepted a hit on the overall quality of music
on the site in the name of of\hbox{}fering a model of sharing that musicians already
understand. 

The idea behind ccMixter is to f\hbox{}ight through the bramble and get to a better way
to serve musicians. The model at ccMixter may have been obvious sooner to more
people (including myself) if the exchange of music was not encumbered by an
overwhelming imbalance towards ``All Rights Reserved''. In a marketplace where
every note is packaged with a price tag, creativity is locked away in that
packaging and therefore unavailable\footnote{This paragraph is a remix of a
section from \textit{The Gift: How the Creative Spirit Transforms the World}
Lewis Hyde 1979, pg 82., the key phrase of which is ``A scientist may conduct
his research in solitude, but he can not do it in isolation.''}. Thanks to the
vision of Lucas Gonze, Neeru Pahria, Mike Linksvayer and the support of Creative
Commons, we can now see an environment where creativity f\hbox{}lows unencumbered as
the currency of exchange between musicians. 


\section{The ccMixter Laboratory}
\label{s:unexpected_collaboration:ccmixter_lab}

\begin{flushright}
\textit{[Creative Commons licences] represent a visible example of a type of
creativity, of innovation, which has been around for a very long time, but which
has reached new salience on the Internet - distributed creativity based around a
shared commons of material.}\\
James Boyle, \textit{The Public Domain: Enclosing the Commons of the Mind}
\end{flushright}

On the surface, ccMixter is a music site that accepts three kinds of
submissions: samples, \textit{a cappellas} and the remixes that incorporate
them. When a remixer is uploading, he is presented with a simple interface that
helps him identify which samples, \textit{a cappellas} or other remixes he
sampled. This allows all three types of submission to link to each other,
signifying the specif\hbox{}ic relationships between them. Simplistic as the idea seems
at f\hbox{}irst glance, the freedoms f\hbox{}lowing throughout this linking relationship have
sparked an exciting set of developments.

The most rewarding aspect of the last four years has been witnessing how many
musicians relate to what is going on at ccMixter, especially those that had no
previous connection to the open music movement. In a music industry that pits
musicians against each other in a frenzy of demagoguery, here is a place for
gifts exchanged in a spirit of cooperation and kinship. It is obvious that many
musicians long for the values of the sharing economy, even when looking for
rewards from the commercial economy. For all the lecturing, vilif\hbox{}ication and
criminalization they've had to endure, maybe it is this generation that could
teach the previous one about how to avoid the need for ``reparations'' later
on\footnote{Jon Pareles ``\textit{For Old Rhythm-and-Blues, Respect and
Reparations}'' \textit{New York Times}, March 1, 1997 
\url{http://ur1.ca/fduq}}.

Philosophically, the ccMixter project is part of what Lewis Hyde calls the
``gift economy''\footnote{Hyde \textit{The Gift} 1979}, Lawrence Lessig
references as  the ``sharing economy''\footnote{Lessig \textit{REMIX Making Art
and Commerce Thrive in the Hybrid Economy} 2008} and related to what John
Buckman calls the ``Open Music'' movement\footnote{John Buckman ``What is 'Open
Music'?''\\
\url{http://ur1.ca/fdut}}. ``In a free market,'' Hyde explains,
``the people are free, the ideas are locked away\footnote{Hyde \textit{The Gift}
pg. 85}.'' Liberated from the commercial marketplace, ccMixter leverages the
Internet to its fullest by demonstrating ``distributed creativity based around a
shared commons of material''. As these authors would have predicted, but took
many of us by surprise when it actually worked, ccMixter has become an engine
for creative innovation. 


\section{The Sample Pool}
\label{s:unexpected_collaboration:sample_pool}

\begin{flushright}
\textit{We are lightened when our gifts arise from pools we cannot fathom.}\\
Lewis Hyde \textit{The Gift}
\end{flushright}

Traditionally, musicians can interact through an implicit collaboration in which
a musician's only contact with another is through a score, sheet music or audio
recording. Digital recording techniques have been a revolution for implicit
collaborations. There are countless terabytes of commercially available sample
in libraries and embedded in electronic instruments. All of those packagings
have their own custom formulated licences creating individual islands of
copyrighted material. Unlike the recording industry, sample library vendors are
much less eager to sue musicians who violate the terms of these licences.
Dangers are still there, however, and at least one popular audio tool vendor was
shaken to the point of declaring they will ``remove all melodic loops'' from
their of\hbox{}ferings\footnote{``All Fruity, No Loops: FL Studio to Remove All Melodic
Samples; Murky License, Content'' by Peter Kirn\\
\url{http://ur1.ca/fdvi}}.

CC licensed samples of\hbox{}fer a way out, but it was important that ccMixter would
not be seen as the host for CC samples. Instead, it was our hope to set an
example for commercial and amateur sample providers. So, we decided to use the
phrase ``CC Sample Pool'' to refer to the world wide collection of music
available for sharing and remixing and position ccMixter as just another player
contributing to the Pool. (If you are familiar with CC licences then you can
think of the Pool as the subset of the Commons that includes all audio samples
licensed without the NoDerivs clause.) The Pool, we tell musicians, is a safe
harbour since, by def\hbox{}inition, all the samples are provided under a well
understood, liberal, licensing scheme.

Other sites, such as the freesound project\footnote{\url{http://ur1.ca/fduv}} from the University of
Barcelona, have since sprung up providing sound designers a CC platform to share
their work.

In order to further promote the idea that ccMixter was just a small part of a
larger ecology, we published a developers' interface\footnote{``Sample Pools''
Creative Commons developer wiki.\\
\url{http://ur1.ca/fduw}} to allow
disparate Sample Pool sites to communicate with each to share their catalogues
of samples. ccMixter currently uses this API to give remixers an easy way to
attribute samples they have used from other websites such as freesound and
Magnatune.com.


\subsection{Innovation Fodder and the Unexpected Collaboration}
\label{ss:unexpected_collaboration:sample_pool:innovation_fodder}

Providing a legal safe harbour is only the f\hbox{}irst implication of an ever growing
Pool. Over the course of the project, it became clear the Pool was indicating a
special breed of creativity.

When musicians work alone they are limited by their own technical skills or
sample libraries they have purchased. When contracting musicians for a recording
session, the project is limited by budget constraints and the skills of the
hired musicians. When collaborating with friends or band mates, the results are
limited by the collective skills of the band, typically three to f\hbox{}ive people.

Compare those limitations to a pool in which millions of samples are available
for sharing and sampling. An unlimited number of genres, styles and playing
techniques. Instead of placing an advertisement in craigslist for a bass player,
musicians can now search the Sample Pool for a huge variety of bass samples. No
more worrying about being restricted by the skills of your collaborators, no
more waiting for someone else to f\hbox{}inish their parts and, best of all, no more
hurt feelings when you are not satisf\hbox{}ied with a part submission. 

By removing restrictions of skill sets, time pressure and personality, the CC
Sample Pool has enabled the most exciting development on ccMixter: the
unexpected collaboration. Consistently, a musician or singer would upload a
sample or a cappella with their own frame of reference and inspiration.
Some period of time would pass, sometimes a year or more, and a remixer would
pluck the sample or 'pell from the site and use it in a completely unexpected
context, sometimes (and this is the exhilarating part) surprising the remixer.

A work of art can be considered creative when familiar elements are combined in
an unfamiliar and therefore unanticipated context. The CC Sample Pool has turned
out to be a factory for just this kind of re-combination, because when browsing
the Sample Pool with an open mind, the remixer is bound to be inspired in ways
previously unconsidered. The remixer may have his personal history and training
to reference, the Pool has no such limitations.

I could relate to this idea when ccMixter founders Neeru Pahria and Lucas Gonze
talked me through this four years ago, but watching it happen as a matter of
course has been a revelation. 

The inspiration does not stop at the remixer. Lessig relays the story of
SilviaO\footnote{Lessig \textit{REMIX} pg. 17}, a singer who uploaded a
Spanish a cappella that I remixed. I am not f\hbox{}luent in either Spanish or
the Latin rhythms she was imagining when singing the song. When I heard the
a cappella, I was inspired by the potential for a lilting, funky jazz
accompaniment and I proceeded to mangle the vocal part into nonsensical Spanish
on my way to my arrangement. She later remarked to Lessig that she realized she
was ``just a little part of the huge process that was going on now with this
kind of creation''.


\section{Attribution Tree}
\label{s:unexpected_collaboration:attribution_tree}

In late 2008, as I was preparing to speak at FSCONS. I turned to the ccMixter
community forums to ask a question, the premise of which postulated a scenario
in which a musician would turn a sample over to the Public Domain, not expecting
any money or credit in return. This was the premise, mind you, not even the real
question. The thread was immediately derailed and got stuck, repeatedly, on the
idea of passing a creation into the PD.

I was reminded, as I had been so many times in the course of my activism for CC,
that musicians are a traumatized lot. Understandable after 100 years of taking a
beating by your own industry that holds out, as its highest attainable goal, a
Faustian ``loan sharking''\footnote{Fake Steve Jobs ``The music industry nobs
have f\hbox{}inally f\hbox{}igured out what we're doing'' July 4, 2007\\
\url{http://ur1.ca/fduy}} lottery (A.K.A. record deal) that if,
heaven forbid, you actually win, gives you the chance to relinquish all rights
to your music for life with the privilege of paying for every expense along the
way.

The idea that a musician would voluntarily give away attribution was very, very
confusing to many participating in that forum thread. Don't forget we are
talking about musicians who had each put hours of music into the Commons, hardly
neophytes to the sharing economy. But mess with attribution and a line has been
crossed. As it was later pointed out to me at the conference, this attitude is
not unlike academic publishing where credit is \textit{currency}.

Lucky for me, ccMixter has the most thorough attribution scheme we could
conjure. If it didn't, I'd be furiously coding it instead of writing this
document or risk being hung by my thumbs by the ccMixter community. Every remix
listing on the site includes a section that points to its sources.

Here's the attribution section for a song called ``Coast2Coast (We Move mix)''
by an artist named duckett:

\begin{quotation}
\textbf{Uses samples from:}\\
Coast to Coast by J.Lang\\
Mellow Dm 5ths by Caleb Charles\\
1165\_walkerbelm by dplante\\
\end{quotation}

The f\hbox{}irst listing shows that duckett used an a cappella uploaded by J.
Lang called ``Coast to Coast''. If we click on that  song title we are taken to
the details page for the a cappella. There we can see all the places
where the a cappella has been sampled:

\begin{quotation}
\textbf{Samples are used in:}\\
coast to coast-D\ldots by deutscheuns\\
Coast to coast (\ldots by alberto\\
Coast 2 Coast (j\ldots by ASHWAN\\
Coast 2 Coast (A\ldots by Dex Aquaire\ldots\\
My Name is Geof\hbox{}f by fourstones\\
Reminisce Coast by teru\\
Coast To Coast by ThomasJT\\
One Night Stand \ldots by CptCrunch\\
c2c2c by fourstones\\
Let Me Know by KatazTrophee\\
coast to coast by kristian v\ldots\\
Coast2Coast (We Move Mix) by duckett\\
\end{quotation}

We can see duckett's remix here at the bottom.

Through the use of the Sample Pool API and a blog-style trackback system we
extended these links beyond ccMixter and point to other members of the Sample
Pool, videos on hosting sites like YouTube and F\hbox{}lickr, podcasts and any other
reference to the music.

It became clear that many ccMixter musicians consider the people they sample as
benefactors and attribution as a reciprocal currency. As I learned from my
experience while preparing for the conference, the justice implied in properly
crediting your benefactors is a reactionary passion amongst ccMixter musicians.
But, I claim the attribution tree demonstrates something even more powerful.

Exposing a piece of music's roots takes the shine of\hbox{}f the ex \textit{nihilo}
mythology that fosters an image of the musician working alone in his head to
create his masterpiece without the assistance of mere mortals. This image is
what corporate marketing revels in and how many musicians, fuelled by a bubble
of sycophancy, see themselves. The ccMixter attribution scheme is a statement
about how art really works, everybody building on each other.

The attribution tree is what I mean when I say we've turned the artistic process
inside out - instead of hiding our tracks in the hopes of being considered
``great'' individual composers, we make attribution the focus of the enterprise
and build reputation on who is sampling and who has been sampled the most.
Derivation and re-use is the generous, creative spirit incarnate. The
attribution tree is the accounting book of a gift economy.


\section{A Capellas}
\label{s:unexpected_collaboration:a_capellas}

If we ever get around to making ccMixter T-shirts, they will read:
``\textit{Came for the a cappellas, stayed for the sharing economy.}''

Nothing attracts talented musicians like the chance to work with a strong
vocalist. And nothing attracts good singers like the chance to work with an
inspired producer. This mutual attraction is true for traditional recording
sessions as well as for remixing communities. When the Creative Commons staf\hbox{}f
showed me a prototype of ccMixter, my f\hbox{}irst suggestion was to add a section
specif\hbox{}ically for a cappellas. I felt very strongly that in order to bring
legitimacy to CC in the music world they would have to substantially increase
the quality of the CC music and a good crop of a cappellas was the key to make
that happen.


\subsection{Why (Free) Music Doesn't Suck Any More}
\label{ss:unexpected_collaboration:a_capellas:why_free_music}

A cappellas, indeed, have become the fuel for what makes the site work. They
ensure an overall aesthetic quality and that alone continues to make ccMixter
relevant to musicians. More than a few of the best remixers have made it clear
it was the great 'pells that attracted them in the f\hbox{}irst place.

For the rest of us, the less-than-best remixers on the site, the ef\hbox{}fect is
profound. You might enjoy a fourstones instrumental remix - or you might not.
The nice thing for me is that I can add Silvia's voice to it without taking a
chance she's having a bad day during an explicit collaboration. I can hear her
fantastic vocal performance as it sits in the Pool. Here's the real kicker: by
collaborating with Silvia in this way, you think better of fourstones music
because, in fact, my sound is better with her vocals than without. This is
important to note because it was not the cause of CC that hooked the best
musicians (who never heard of Lawrence Lessig and still have not visited the
Creative Commons Web site) into the open content world, it was the chance to
share in a pool of high quality stems\footnote{In music production a ``stem'' is
the isolated recording of a single instrument.} and 'pells, a chance to improve
their sound.

An awakening is triggered in the musician when you add frictionless access to
the 'pells, a disassociation from commercial enterprise and a model where
musicians retain ownership of their work. As their remix is picked up by a
YouTube video or podcast (both of which we track on ccMixter) more lights start
to come on. F\hbox{}inally, they start to notice a relationship between the gift
economy and their own artistic process. As I have witnessed many times in the
last four years, this relationship is what produces a fundamental shift in the
musicians' understanding of what is possible with reforms in ownership,
attribution and sharing.


\subsection{The Pros vs. The Artists}
\label{ss:unexpected_collaboration:a_capellas:pros_vs_artists}

Lessig divides the motivation of participants in a sharing economy into
``me-regarding'' and ``thee-regarding.\footnote{Lessig \textit{REMIX} pg. 151}''
Playing softball on a Saturday afternoon in Central Park against a rival law
f\hbox{}irm is a me motivation.  Ladling soup in a homeless shelter on a Sunday
afternoon is thee motivation.

The relationship I describe between the remixers and 'pells above is classic me
motivation. ccMixter provides a service to remixers by giving them access to
fantastic singers without any more ef\hbox{}fort than browsing the a cappellas section
of the site. Putting the remix into the Commons is seen as a small payback for
the chance to work with a premier vocalist that actually, you know, sings in
key.

Roughly two and a half years into the project ccMixter started attracting a new
kind of musician: the professional producer. When they f\hbox{}irst arrived, they were
far less adventurous than the remix artists we were used to, but their
productions were so well put together and slick (in a good way) that it was a
treat to have them on board. Rather than take a 'pell into a deep, personal
artistic place, they were expert at pleasing the customer. What I mean by that
is that they would create perfectly executed ``straight up'' productions around
a 'pell that succinctly matched what the singer had in mind, regardless of
genre.

Many of these producers had come from another remix site, one which operated
under an ``All Rights Reserved'' model. After a while at ccMixter however, a
transformation had been noted. More than a year after they moved over, one
long-time observer, a fellow remixer, noted in a review:

\begin{quotation}
``It's been a year of surprise from people like you and [others] who I thought I
had neatly categorized [at the other site] into a style and who have brought new
things seemingly out of the blue\footnote{ccMixter artist collab, in reply to a
review of his remix ``Beautiful People''\\ 
\url{http://ur1.ca/fduz}}.''
\end{quotation}

Out of the Pool, actually. This is a snapshot of an artist half-way through the
realization of what is enabling a newly found sense of adventure and innovation.

The surprising thing to me about the professionals was their initial attitude
toward the 'pells. It took me a while (and several Victor-schooling, pointed
email exchanges) to f\hbox{}igure out what was going on and even longer to build an
honest appreciation for it. You see, when you're a professional producer at the
top of your game the last thing you're starving for is a decent singer. Great
singers will pay you to work with them, that is how you make your living after
all. It shouldn't be surprising in this context that the pros see their remixes
as the gift. They are providing their services to these singers (and
incidentally to the Commons) \textit{pro bono}. Classic thee motivation. The rest of us
are all playing softball, these guys are handing out delicious free soup.

And thank heaven for their gifts (and their patience with me) because just by
showing up they brought more than just great music, they were giving mainstream
credibility to the entire open music movement.



\section{Licenses}
\label{s:unexpected_collaboration:licenses}

Creative Commons exists to give artists a way to signify, through a set of
ready-made licences, what can and can not be done with works posted to the
Internet. A full explanation of CC and the licences is beyond the scope of this
document but clearly it is a cause I consider worthy.

The popularity of the CC brand adds to the power of the licences - the more
people know what the brand means the less questions, the more legal sharing and
reuse, the richer the culture. The potential downside of that popularity is that
more people are likely to make bad assumptions about what the brand actually
means in legal terms. For example, there is a range of permissions between the
individual CC licences and there is a non-zero learning curve on recognizing
which of those permissions apply to a piece of art with a given CC licence.

At the risk of perpetuating the (wrong) meme that the CC brand simply means ``do
what you want'', I thought it was essential to create an environment at ccMixter
that worked within the CC domain, but still gave the remixers safe haven from
legal worries. I wanted to put the best possible face on the licences that I
could credibly get away with presenting. Is that spin? I hope not. Either way,
this goal turned out to be laced with challenges. Worth every ef\hbox{}fort, but laced
nonetheless.


\subsection{The Sampling Licences}
\label{ss:unexpected_collaboration:license:sampling_licenses}

An important element of the roll-out for the CC/WIRED contest was a new family
of CC licences aimed specif\hbox{}ically at sampling and remixers. I won't go into the
history of these licences but mistakes were made and lessons were learned.

My mistake was ignoring public calls from CC to join the discussion during the
drafting of these licences in the summer of 2004. I f\hbox{}igured this was ``legal
stuf\hbox{}f'' and everybody knew what they were doing and had the best intentions. All
that was correct but I should have made my opinions heard before and not after.
Had I been a better CC citizen, I could have avoided a lot of grief later, after
the site opened, after I realized what these licences really meant. My
involvement might not have made a whit of a dif\hbox{}ference in the drafting phase,
but at least I would have been better prepared.

A few months after the launch of ccMixter, I had come to a bitter conclusion.
The Sampling family of licences had restrictions and requirements that I came to
believe were doing more harm than good to the cause of demonstrating reuse.
Audio samples with these licences were legally incompatible with audio samples
licensed under other CC licences. Even worse, remixes with a Sampling licence
could not be used as video soundtracks - not even in amateur YouTube-style
videos. I was concerned that we could not credibly claim to be the ``sane''
alternative to an ``All Rights Reserved'' model under these conditions.

I made my case to CC staf\hbox{}f and they agreed to discontinue supporting the
Sampling licences on ccMixter and green-lit a ``re-license'' campaign on the
site that gave musicians a chance to remove the Sampling licences where legally
feasible.

Since then, CC came under f\hbox{}ire for having too many licence options, confusing
potential adopters and support was dropped for one of the lesser used Sampling
licences. The others still exist as options in the CC licence chooser but have a
much lower prof\hbox{}ile than in November 2004.


\subsection{ShareAlike}
\label{ss:unexpected_collaboration:license:sharealike}

We settled on supporting two licences commonly known as: Attribution and
NonCommercial for new uploads. That means a musician posting original samples
and a cappellas could say ``copy or remix my sample in any context, even in a
commercial project'' (Attribution) or ``copy or remix my sample, but if you use
it in a commercial project you need to contact me f\hbox{}irst so we can work something
out'' (NonCommercial). Both licences require giving credit to the musician you
sample.

If someone does use a sample with one of these licences in a remix, they are
under no obligation to license the remix under a Creative Commons licence. This
is great when it comes to choice and freedom, but it's not optimal when you're
trying to spread CC.

There is another licence feature that would force the remixer to license the
track under CC, it's called ShareAlike. We could have of\hbox{}fered ShareAlike and
NonCommercial-ShareAlike on ccMixter as two more options. The problem is that
ShareAlike is not combinable with the non-ShareAlike version of NonCommercial.

Eyes glazed over? No kidding.

Here's what that means. Joe the remixer wants to use two samples from the Pool
in his remix. One sample is licensed under NonCommercial, the other is
ShareAlike. In order to do so legally he would have to get permission from the
person that uploaded the ShareAlike sample. If he didn't get permission he would
be in exactly the same boat as if he had sampled a Michael Jackson record:
copyright violation.

At this point, I was facing a serious dilemma. On one hand, I would love to
encourage CC licence adoption by using the ShareAlike licence. On the other
hand, the last thing I want to do is enable musicians to post copyright violated
remixes to ccMixter simply by having the wrong combination of CC samples.

I didn't ruminate too long on this one because I quickly decided it was more
important to have a totally ``safe'' environment where any two samples could be
mixed together legally. I had a nightmare scenario of a producer spending weeks
on a remix using samples they had downloaded exclusively from ccMixter only to
f\hbox{}ind out they were in violation of the law. I wanted to give musicians
\textit{some} hope.

The real issue here is the NonCommercial licence which is very popular and
drives adoption of CC, but has been problematic. I can't speak for how CC deals
with the rest of the world but in my experience, when I have a problem it is met
with transparency, an appreciation for honesty and a healthy distaste for false
sacred cows. Consequently, I'm happy to report there is currently a major
re-think under way regarding the NonCommercial licences with lots of help from
the community and academia. This time, I let my feelings be known. You should
too\footnote{CC Wiki ``NonCommercial'' discussion page\\
\url{http://ur1.ca/fdv0}}.


\subsection{Licences for Remixes}
\label{ss:unexpected_collaboration:license:licenses_for_remixes}

As matter of policy on ccMixter, to simplify things for musicians, no remix can
specify a CC licence. Instead, you ``inherit'' the most restrictive licence from
the samples you use. For example, if you use two samples where one has the
Attribution licence and the other has the NonCommercial licence, then your remix
will be posted under a NonCommercial licence because that one is considered
``stricter''.


\subsection{The Heavy Breathing Factor}
\label{ss:unexpected_collaboration:license:heavy_breathing_factor}

Creative Commons attracts a lot of academics who are eager to mine ccMixter's
data that we've collected over the years. The most common things they are
looking for are patterns of behaviour with respect to the CC licences.
Understanding this behaviour and how to increase the musician's awareness of
their choices is important to the future viability of CC licences. We are happy
to oblige and make all of the internal database tables - minus user Internet
connection IDs, emails and passwords - to just about anybody that asks. And we
get asked a lot, especially around doctorate season. 

Unfortunately, decisions involved in making music are emotional, based on aural
proclivities and none of that is captured in ccMixter's internal database
tables, even as scientists do their best on semantic audio prof\hbox{}iling
tools\footnote{``Integration of Knowledge, Semantics and Digital Media
Technology, 2005. EWIMT 2005. The 2nd European Workshop''\\
\url{http://ur1.ca/fdv1}}.

For example, we don't track the gender of the singer or remixer. Yet, the
primary demographic of ccMixter remixers is a male. How do I know?  Below is a
chart of the top 12 most remixed a cappellas\footnote{As of December 28th, 2008
and excluding those related to remix contests.}. Note the gender proclivity (I
added the last column manually):

\begin{table}[h]
\label{t:unexpected_collaboration:license:heavy_breathing_factor:remixes}
\begin{tabular}{|l|l|r|l|}

\hline
upload & artist & \#remixed & gender\\

\hline
Ophelia's Song & musetta & 64 & F\\
Sunrise & shannonsongs & 63 & F\\ 
Lies & trifonic & 54 & F\\
Matter of Time & shannonsongs & 49 & F\\
Girl and Superg & lisadb & 48 & F\\
Sooner Or Later & trifonic & 46 & F\\
Magic In Your E & Songboy3 & 43 & M\\
Whatever(acappe & Tru\_ski & 42 & M\\
September & calendargirl & 42 & F\\
Broken & trifonic & 40 & F\\
Freedom & snowf\hbox{}lake & 36 & F\\
We Are In Love & shannonsongs & 36 & F\\
\hline
\end{tabular}
\end{table}

A further look at the data reveals that it typically takes a male singer or
rapper roughly twice as long, at twice the uploading pace, to reach the same
number of remixes as his female counterpart.

The preference seems to go further than mere gender, and this is where simply
mining the data as numeric values completely breaks down. All of the female a
cappellas in that chart can be said to share the same vocal style. The
performances could be called laid-back, cool, breathy. If I were a less
enlightened person I would say they sound, in a word: sexy.

We have had uploads by a few women that have a stronger, more dramatic vocal
style. These are fantastic singers who could really belt out a melody, American
Idol-style. Yet, they completely f\hbox{}izzled on ccMixter, with barely a remix, and
of those, many were pretty terrible. This is not a ref\hbox{}lection on the singer.
Again, these are truly gifted vocalists who simply are not to the personal taste
or don't f\hbox{}it the harmonic prof\hbox{}ile of the better remixers on our
site\footnote{Victor Stone - Virtual Turntable blog ``My (Throwing) Muse'' Blog
entry in which I discuss a kind of mismatch between a remixer and singer that
may be attributed to clashes in the harmonics of a singer's voice and bedding
the remixer typically users.\\
\url{http://ur1.ca/fdv3}}.

Regarding which source material to use, the conclusion I've come to is that
liberal licences are less about choice and more about enabling. The decision
whether to use a specif\hbox{}ic piece of music or not is based on the content. If it's
available without legal strings attached all the better - but the decision
rarely starts with a licence agreement. This is clearly the case in a
non-commercial environment like ccMixter, but art is what comes f\hbox{}irst to an
artist - the rest is back-f\hbox{}ill.


\section{What's Missing: Open Payment Protocol}
\label{s:unexpected_collaboration:open_payment_protocol}

More crossover between the sharing economy and the commercial economy, as in a
list of Hollywood credits, would certainly provide potential business partners
with the ``recognition of success''\footnote{Lessig \textit{REMIX} pg. 221}.
Allowing contact information to atrophy, as so often happens on the Web, and
thereby ignoring email inquiries to license music for money, is not optimal for
achieving that end.

One possibility would be to create a mechanism to funnel money to the artist
(and all the artists that artist sampled) cleanly and automatically. If I post a
remix that gets licensed for money, I expect everybody I sampled would get paid
automatically, even when the sample was posted on another site.

Personally I would hate to see the actual royalty payment system turn into a
proprietary, competitive marketplace. From a musician's perspective I want music
hosting sites to add value on top of an established, open protocol between
sites. 

The ccMixter attribution tree and the Sample Pool API serves as a non-commercial
skeleton today but could be expanded, perhaps with CC+ technology\footnote{CC
Wiki ``CCPlus''\\ 
\url{http://ur1.ca/fdv4}},
to include a royalty pipeline between artists, even when they host music on
dif\hbox{}ferent sites. The tools for royalty payments can be made as transparent as
simple attribution - in the case of ccMixter that's done by picking the sources
from a search result list.

The type of features that would be needed on all commercial music hosting sites
includes: 

\begin{enumerate}
    \item{A way to automate payment to an artist such as a PayPal(tm) account.}
    \item{A choice of pricing schemes that allows someone posting an a cappella
        or sample to set a price for dif\hbox{}ferent scenarios of usage. For example:
        Free for schools, \$10 for short videos, \$100 for f\hbox{}ilms, etc. I would
        even be interested in an ``expiration price''. This says: if you can't
        reach me through the means I supply within XX days, then the price is XX
        amount (including zero).}
    \item{A marking on every a cappella or sample that signif\hbox{}ied it has been
        ``cleared'' - meaning it is either free to use in a commercial context
        through an Attribution licence or there is a clearly marked price
        (depending on scenario) and a way to make payment on it.}
    \item{A remixer can set the price(s) for his own remix but the total fee for
        the remix will include royalty payments for the artists he sampled.}
    \item{Payment would be posted to the site and distributed automatically to
        the remixer and everybody sampled including, through the royalty
        pipeline, artists on other sites.}
\end{enumerate}

Again, it would be a mistake to make this payment system part of a proprietary
competition between businesses. Music hosting has plenty of areas to compete in
for value-added services. Like ef\hbox{}fectively soliciting for licences.


























































































