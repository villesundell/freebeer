\begin{savequote}
    \qauthor{\LARGE{Smári McCarthy}}
\end{savequote}
\chapter{The End of (Artif\hbox{}icial) Scarcity}
\label{c:articifial_scarcity}

The modern materials economy has been marked by an unwillingness to face the
subtle repercussions of the industrial revolution. In this essay I intend to
play out this future drama of mankind in three parts. F\hbox{}irst, I will set
the stage by showing that we have perhaps unknowingly built several political
assumptions into our society in such a way that we cannot see these 
foundations, let alone replace them when they are sinking into the mire. 
Second, I will show that the failure of these foundations is not merely
inevitable, but that it has already happened. F\hbox{}inally I intend to try to
describe a couple of methods we can use to build new egalitarian foundations 
for our societies.


\section{Act 1. Our Unspoken Mythology}
\label{s:artificial_scarcity:unspoken_mythology}

A myth is a powerful thing. The power of a story, an epic or a tale is 
formative to a culture, from the epic of Gilgamesh to the stories collected by
the Brothers Grimm and onwards to \textit{Star Wars} or \textit{Harry Potter}.
The stories of our time give us the context by which we live our lives – the
stock phrases, the iconography, even, nowadays, styles and variations. Every 
era has its heroes, and the narratives they follow from are strongly woven into
the mood of the era, as both reality and f\hbox{}iction move forward in a
powerful symbiosis – who would Beowulf have been without the conception of evil
hidden in the darkness personif\hbox{}ied by Grendel? Would James Bond have 
been interesting if not for the Cold War and subsequent hiccups and hijinx in
global politics?

Before the advent of writing, stories were transmitted from person to person by
word of mouth. Until the printing press came to be they continued to go by word
of mouth primarily but were also preserved for posterity in a slightly more
permanent and immutable form. The printing press changed all that, it provided 
a platform by which two things could be achieved. F\hbox{}irst, the
formalization of myths – no longer would they be subject to faulty memory or
creative manipulation, embellishment or subjugation. Second, the elimination of
scarcity – the printed myths in their more immutable form could be reproduced
almost indef\hbox{}initely, allowing the ideas presented to reach an almost
inf\hbox{}initely larger audience, given time.

Our stories have captured well the struggle for freedom. The premise of Arabian
nights is the thousand and one nights in which the sultan is told a fascinating
tale by his harem-bound storyteller who yearns for freedom from captivity.
Dickens's stories often featured themes of freedom, from \textit{The Tale of 
Two Cities} to the \textit{Christmas Carol}, the protagonists seek freedom of
some kind. \textit{Oliver Twist} told of a boy wishing for freedom from poverty
that was unjustly assigned to him as an unwanted birthright. Even Shakespeare
put his f\hbox{}inger on the topic every now and then; Romeo and Juliet's 
desire to be free from the constraints of their social situation, feeling that
the battles on the streets of Verona weren't necessarily what they signed up
for.  Some are more blatant than others in this, Orwell's \textit{1984} and
\textit{Animal Farm} notwithstanding.

All of the above can be studied in a number of ways, and is. While folklorists
may refer to the Aarne-Thompson system\footnote{A system which enumerates
roughly 2,500 basic plots that manage to encompass most stories. See Antti
Aarne, \textit{The Types of the Folktale: A Classif\hbox{}ication and
Bibliography}, The F\hbox{}innish Academy of Science and Letters, Helsinki,
1961, for Aarne's original system which was later expanded by Thompson.} as a
way of understanding the stories' structure, and semioticians may consider the
symbolism within a tale or the meaningful patterns that emerge in collections 
of stories\footnote{A fairly benign guide to Semiotics for people unfamiliar
with the term is Daniel Chandler's Semiotics for Beginners,
\url{http://ur1.ca/f6ro}}, there may be a better f\hbox{}ield to use in our
exploration of the theme which interests us the most in this instance, namely
freedom.


\subsection{Formative myths}
\label{s:artificial_scarcity:unspoken_mythology:formative_myths}

The f\hbox{}ield of memetics came out of Richard Dawkins' book \textit{The
Self\hbox{}ish Gene}\footnote{Richard Dawkins, \textit{The Self\hbox{}ish 
Gene}, 1976}, which applied the phraseology of epidemiology and genetics to the
concepts of ideas.  Memetics studies evolutionary\footnote{It's worth 
mentioning that not all evolution needs to be Darwinian evolution; I think 
ideas are more of a Lamarckian type, if any model of ``evolution'' (as opposed
to emergence) applies at all here.} models in the transmission of ideas, and is
as such as much born out of information theory on the one hand and cybernetics
on the other as much as it is from genetics. In fact I generally consider
memetics to be a sub-f\hbox{}ield of cybernetics, which I'll come to later.

The meme (or possibly meme-complex) of freedom is very popular and very
powerful, being transmitted from an ardent believer (memoid) to a potential 
host through various means. Indoctrination generally begins young as with any
potent idea, like language or property or respect for elders. Freedom also 
seems to be a meme that people are prone to reinvent if they aren't infected
with it and they f\hbox{}ind it might be useful. Freedom, as a meme, has 
several f\hbox{}laws though. It is largely undefended against 
misrepresentation, it has inconsistent sociotypes (or social expressions of the
meme), and it appears quite prone to memetic drift, or the idea becoming 
watered down as time progresses, until such a time that it snaps back into full
force, creating a sawtooth-wave of sorts.

All myths are not f\hbox{}iction. Some myths are portrayed not as stories for
campf\hbox{}ire sittings or late night movies, but rather as if they were the
truth. These are generally called lies, but only after they have been 
discovered to be untruthful. Until such a discovery is made, these
\textit{f\hbox{}ictitious} myths are quite as formative as their
f\hbox{}ictional counterparts to our society. A statement regarding some well
respected businessman's deviant sexual behaviour can damage his reputation, 
even if it is a lie. And even after such a lie has been discovered, much
irrevocable damage may have been done.

An example of such a formative lie would be McCarthyism in the 1950s. It was a
widely held belief of the time that communists were a purposefully destructive
force, acting in unity within US borders in an attempt to destroy democracy and
freedom and all that. This belief was strengthened by the will of uncle
Joe\footnote{I am in no way related to former senator Joseph McCarthy, but I
sure like to make that joke. Apparently, so does the Icelandic media, as can be
seen in a late June 2008 edition of Fréttabla{\dh}i{\dh}, where I am likened to the
senator.} and others who used the myth to push forth their political agenda.
Perhaps they believed in the myth, perhaps they didn't. It doesn't matter. The
meme of anti-communist sentiment f\hbox{}lourished under these circumstances,
the cognitive image was strengthened, and society changed because of it.

Granted that we know that myths and lies can be formative to our society, and
our keen interest in this meme called freedom, the central theme of our
movement\footnote{This would be the \textit{Free Society Movement}, and it's
sub-classif\hbox{}ications far and wide, reaching the shores of the Free
Software Foundation, the Electronic Frontier Foundation, Creative Commons, and
so on.}, it is self-evident that we would benef\hbox{}it our choice meme 
greatly if we were to discover lies which have a negative ef\hbox{}fect on it.
There are two in particular that are worth mentioning in this context for their
profound ef\hbox{}fect on our civilization over the past two hundred years and
the astoundingly small amount of scrutiny they have received. 


\subsection{Centralization culture}
\label{s:artificial_scarcity:unspoken_mythology:centralization_culture}

Modern political science narrowly and crudely separates all modes of thought
into the socialist and individualist movements with few exceptions. Whilst most
political scientists will agree that there is more to the world than exists in
the capitalist and communist philosophies, they tend in general to sit on 
either side of that particular fence and toss faeces thence without regarding
other pastures. But deep within both political theories lie two assumptions 
that are held up high. The Marxists may disagree with the Smithists on the
issues of who should own what and who should rule over whom, but despite all
their diatribes they are dear buddies when it comes to the questions of whether
anybody should rule anybody and whether anybody need own anything.

In 1651 Thomas Hobbes published his \textit{magnum opus Leviathan}, a thickset
tome using complex language to explain a set of ideas regarding the nature of
control in man and animal, the essence of authority and the purpose and correct
modes of civilization. In it, he makes certain statements as to the nature of
government in particular, easily stating that in lieu of a strong centralized
government, human civilization will dissolve into chaos\footnote{``The only way
to erect such a common power, as may be able to defend them from the invasion 
of foreigners, and the injuries of one another, and thereby to secure them in
such sort as that by their own industry and by the fruits of the earth they may
nourish themselves and live contentedly, is to confer all their power and
strength upon one man, or upon one assembly of men, that may reduce all their
wills, by plurality of voices, unto one will [\ldots]'', Thomas Hobbes,
Leviathan, chapter XVII (Of the Causes, Generation, and Def\hbox{}inition of a
Commonwealth)}.

The reason given for this is that man is, in his own right, a haphazard beast
and completely incapable of making rational decisions, and thus it is only
natural that his welfare be put into the hands of inf\hbox{}initely more 
capable people such as, say, kings.

Does that sound a little bit odd? Consider this assumption in the context of
capitalism. Very few capitalists entirely reject the notion of
government\footnote{I could point at Milton Friedman and Friedrich von Hayek,
but I'm not going to for reasons that will become apparent.}, most saying 
rather that the government should stay out of the way of the natural behaviour
of the market, which is busy doing its thing. A government has very few tools
with which to sway the behaviour of a community, the f\hbox{}irst and foremost
being the legal system, which provides a system of restrictions (or
\textit{boundary conditions}), which act as parameters within which everybody 
is bound to act.  Restrictions, the capitalists note, put limits on the growth
of an economy.  Rejecting government altogether would be to reject restrictions
altogether, but most capitalists feel strongly about keeping government handy 
in case they screw up.

I mentioned that \textit{Leviathan} addressed ``nature of control in man and
animal.'' This wording is not accidental. In the early 1950s they were used by
mathematician Norbert Weiner in his description of a new f\hbox{}ield of study
with which he had become infatuated, which he verily named 
\textit{cybernetics}, or ``control theory''\footnote{In Lawrence Lessig's
\textit{Code v2.0}, cybernetics is misrepresented as a study of ``control at a
distance through devices,'' missing by far the subtlety of actually studying 
the nature of control itself and the way it behaves in systems.  }. The purpose
of cybernetics was to explore how authority propagates through systems, and it
has alarmingly deep things to say about such things as computers and tribes and
economies and so on. Nowadays cybernetics is rather unpopular, with one of the
world's largest cybernetics faculties having recently been merged with a 
faculty of computer science, as if it were so narrowly def\hbox{}ined. 

In previous decades cybernetics had glorious times, like when Staf\hbox{}ford
Beer spent time in Chile helping Salvador Allende's government install a
computer-controlled network of sensors and transducers, connected upstream
through statistical software, that gave a simple method of reacting to
situations at the factory, district, county, or national level\footnote{See
\textit{Fanfare for Ef\hbox{}fective Freedom}, by Staf\hbox{}ford Beer.}. The
idea was to use a network of teletype terminals running through the phone
system, a precursor to the Internet, to maintain complete information about the
status of the nation's economy; the Marxist government having the ability to do
without the capitalist theme of withholding information that may benef\hbox{}it
competitors. 

The project was killed along with Allende himself when the CIA sponsored
\textit{coup d'etat} organized and enacted by General Augusto Pinochet shocked
the Chileans into submission\footnote{See \textit{The Shock Doctrine}, by Naomi
Klein.}. It is unsure to what degree the CyberSyn project, as it was called,
af\hbox{}fected the CIA's decision to sponsor the coup, but it is clear that 
one of the key motivations for replacing Allende's Marxist government was to
temper the rising prices of copper, Chile's main export, which was required for
the growing information infrastructure throughout the west: CyberSyn, by
heightening the f\hbox{}low of information through the industrial sectors in
Chile and af\hbox{}fording the workers a more egalitarian method of industrial
organization, was threatening to make the adoption of information technology 
too expensive in the western world at a pivotal point in time.  Perhaps one
could argue that Pinochet saved the Internet by enslaving an entire nation, but
in doing so set information technology back by decades.


\subsection{Building the System}
\label{s:artificial_scarcity:unspoken_mythology:building_system}

In cybernetics, you consider a \textit{system} to be a \textit{state space} 
upon which a set of \textit{transformations} may act\footnote{See \textit{An
Introduction to Cybernetics}, by W. Ross Ashby.}, and by mapping all possible
transformations on the state space you can f\hbox{}ind contextually congruent
states and possible paths that the system can take. To visualize this, take a
piece of paper and draw a circle on it. The paper is the system, the circle
represents the desired operational boundary of the system. Now place a point
randomly inside the circle. This is the system's state. Now without lifting the
pencil, go back and forth within the circle, making scribbles.

A number of interesting questions arise. What happens if you keep going back 
and forth between the same places? This is called homoeostasis, and is 
generally considered a good thing, albeit somewhat unexciting. It occurs when
you have a harmonic oscillation between states. Call it harmony if you will.
Don't call it Utopia, please.

Does distance traversed within the circle matter? It does. If you go too far
your system is very unstable, and is likely to explode. If you don't go far
enough the system may grow ``cold'' and die out, being replaced by something
else entirely\footnote{A Douglas Adams quote comes to mind: ``There is a theory
which states that if anybody ever f\hbox{}igures out what the Universe is and
what it's for, it will immediately by destroyed and replaced with something
dif\hbox{}ferent.  There is another theory which states that this has already
happened.''}. What is an explosion? That's when you leave the circle. That's
when you enter uncharted waters. It shouldn't really happen, but let's remember
that this is a large and complex chaotic system where we are faced with any
number of situations such as global warming, \textit{coups d'etat} and
f\hbox{}inancial meltdown. Not everything that can happen exists within the
circle; rather, we def\hbox{}ine our circle in terms of what kind of behaviour
we deem acceptable.

Government then, is the device that draws the circle, that sets the rate of
change in the states, or at least installs speed bumps and so forth to keep
things in check and balance. If they draw the circle too tight – limiting
freedoms too severely – they risk explosion. If they put in too many speed
bumps, they risk cooling out and being replaced by something stronger.

And that's why the capitalists like to keep the government around, because they
control the lasso, they can make sudden changes to the playing f\hbox{}ield.
This can prove useful, they believe.

Consider now the implications of the \textit{Leviathan statement} on communism.
Marx \& Engels noted the importance of the control of the means of production 
to be in the hands of the producers themselves, which sounds quite reasonable.
The idea being that nobody has a say in how and when things are produced unless
they are actually going to be doing the work. They wrote of ownership by the
proletariat, rather than ownership by the bourgeois\footnote{A term which has 
no relevance any more, since industrialization and destruction of natural
habitats have forced the majority of humanity to now live in cities. Now it
would be more correct to speak of \textit{owners of capital}, or, erm,
\textit{capitalists}.}.  So that was theoretical communism, drunken deeply from
tankards forged in the anarchist tradition. But in applied communism we have
seen all over the world a tendency towards drawing ever tighter concentric
circles, building a centralist government which tells people what the plan is
and how it shall be accomplished by way of bureaucratic output in industrial
dimensions.

Verily has a Leviathan been pulled from a hat, and the assumption of strong
centralized government has been abjured into reality. The result is that most
modern local or municipal level government activity is applied to jumping
through hoops manufactured by authorities higher up in the chain. My local town
government has employees writing reports for the ministries of industry and
education and environment, and they in turn have employees writing even larger
reports for the European Union and the United Nations and so on. The power base
has even become so diluted that it is no longer clear exactly on whose 
authority many things are being performed.


\subsection{Scarcity set in Stone}
\label{s:artificial_scarcity:unspoken_mythology:scarcity_in_stone}

More than a century after Hobbes, an awkward man named William Godwin wrote a
book named \textit{An Inquiry Concerning Political Justice}. In this book he
argued against the \textit{Leviathan statement}, insisting that it was a myth, 
a lie, something that might not actually be right and that somebody should
check.  The book sold well at f\hbox{}irst, attracting the attention of many
famous people such as the feminist Mary Wollstonecraft (who later became
Godwin's wife), the romance poet Percy Shelley (who later ran away with 
Godwin's daughter Mary) and former US vice president Aaron Burr (who later
killed Alexander Hamilton because of a silly dispute\footnote{In \textit{The
Federalist Papers} as published by Bantam Classics, Burr is spoken of as
``volatile'' in defence of Hamilton, who wrote of freedom and traded in slaves.
The entire Burr-Hamilton incident is a fascinating one but outside the scope of
this essay.}). But amongst Godwin's erstwhile readers was at least one who
didn't take the meme of political justice without a grain of salt. Thomas
Malthus, being well versed in the \textit{Leviathan statement}, wrote in
response to Godwin a vast tract, \textit{An Essay on the Principle of
Population}.

In his essay, Malthus pointed out that without a strong centralized government
(without using those words) imposing arbitrary restrictions on resource
allocation to the proletariat (without using that word), human population would
continue to increase exponentially until such a time that all the resources
available to man would be depleted and we would all die of starvation and chaos
would ensue\footnote{``Population, when unchecked, increases in a geometrical
ratio. Subsistence increases only in an arithmetical ratio. A slight
acquaintance with numbers will show the immensity of the f\hbox{}irst power in
comparison of the second.  By that law of our nature which makes food necessary
to the life of man, the ef\hbox{}fects of these two unequal powers must be kept
equal.\\ This implies a strong and constantly operating check on population 
from the dif\hbox{}f\hbox{}iculty of subsistence. This dif\hbox{}f\hbox{}iculty
must fall somewhere and must necessarily be severely felt by a large portion of
mankind.'', Thomas Malthus, \textit{An Essay on the Principle of Population},
Chapter 1.}. This was a commonly held belief at the time, but Malthus gained
notoriety for putting it in words and justifying it with graphs.
Suf\hbox{}f\hbox{}ice to say Thomas and William\footnote{And others, including
Nicholas de Caritat, marquis de Condorcet, who developed the \textit{Condorcet}
voting scheme.} argued about this for several decades, and Thomas won hands
down. As in any philosophical debate, the validity of the arguments hinged not
on their truthfulness, but on their memetic infectiousness, which in Thomas'
case was severely augmented by support from the governmental powers in Britain,
desperate to hold on. The Malthusian myth was forged and is still being
reinforced to this day, yet depressingly few Malthusians go out of their way to
read the works of Godwin and Condorcet which are heavily referenced in his
\textit{Essay}.

Consider our circle. In the cybernetic, this means that there exist innumerable
paths from our current state that lead to states wherein we all die from
starvation. I'll assume this lies outside of the circle since we deem that an
unacceptable result. Malthus' claim was that it was government's job to prevent
society from applying certain transformations that would lead to an exhaustion
of resources.

Remember that this is all happening just as the industrial revolution was 
taking its f\hbox{}irst steps, tumbling awkwardly over itself, making silly
mistakes and not really getting very far. Machines, back then, were a joke,
despite Watt and Carnot and the others. So little could Malthus know (although
Godwin predicted) that industry would alter the entire materials economy to a
point where resources were the least of our problems\footnote{For a couple of
hundred years, at least.}, so it's fair to forgive him. What cannot be forgiven
is how this assumption of \textit{scarcity}, the meme of \textit{poverty}, has
managed to survive the industrialization of the western world without being
attacked or scrutinized too deeply.

Look at the f\hbox{}igures. Agriculture in the western world now produces more
food than would be needed for a humanity twice the size\footnote{Statistics
available at \url{http://ur1.ca/f6rp}; for example, 784.786.580 tonnes of maize
were produced worldwide in 2007, 651.742.616 tonnes of rice, 216.144.262 tonnes
of soybeans, 1.557.664.978 tonnes of sugar cane, and so on. That year
6.186.041.997 tonnes of \textit{vegetables} were produced worldwide, which is
roughly a tonne of food per person per year. The US Department of Agriculture
states at \url{http://ur1.ca/f6rr} that the average person consumed 884.52 kg
of food per year, and that statistic includes meat and dairy products.}. About
half of this food is thrown away\footnote{See Timothy Jones;
\url{http://ur1.ca/f6rt}}, and yet about 800 million people are
starving\footnote{According to FAO, 852 million people, about 13\% of the
world's population. ``Of this, about 815 million people live in developing
countries, 28 million in ``transition'' countries of the former Eastern Europe
and ex-Soviet republics, and about nine million in the industrialised world.''
\url{http://ur1.ca/f6ru}} and in the west millions of people are obese. Does
this make sense? Does poverty make sense?

Industry was supposed to remedy this. Wasn't it? Was industry not intended to
replace the human hand with machines, transforming hard labour into a
caretaker's af\hbox{}fair of relative ease, letting machines fulf\hbox{}il our
every want and desire in plenty, letting us all lead comfortable lives of
af\hbox{}f\hbox{}luence? Or was the industrial revolution a purely technical
issue, hackers of yore making things that did suave stuf\hbox{}f just because
they had a strong desire to solve technical problems? Doubtful. As 
technocentric as hacker\footnote{I use the term \textit{hacker} in the sense 
``A person who delights in having an intimate understanding of the internal
workings of a system, computers and computer networks in particular,'' as
def\hbox{}ined in RFC1392 and echoed in senses 1-7 in the Jargon f\hbox{}ile.
\url{http://ur1.ca/f6rv}} culture tends to be, hackers have politics up to 
here. Look at the free software movement, look at Wikipedia.  When technically
minded individuals come together to address problems, be they technical or
political or social, they do so with a fervour that makes people's heads spin.

Nobody is going to convince me that Alessandro Volta didn't think electricity
wasn't going to tip the game slightly in favour of the peasants. Nobody is 
going to tell me that Robert Fulton wasn't acting in what he believed were the
interests of mankind. ``Oh, look,'' I can't imagine him saying. ``there's an
opportunity for further oppression of the working classes by making them not
only have to work, but have to f\hbox{}ight for the right to work too by making
them have to compete on an open market against machines capable of working
tirelessly with arbitrary accuracy!'' Nobody is that stupid. Or are they?

Let's fast forward a bit. In 1968, whilst student uprisings were happening in
Paris, Milan and San Francisco, to name a few of the more important
battlegrounds, a professor of biology at University of California at Santa
Barbara, Garrett Hardin, crawls out of the woodwork of relative obscurity and
writes of the \textit{Tragedy of the Commons}\footnote{Originally printed in
Science magazine with the introductory line: ``The population problem has no
technical solution; it requires a fundamental extension in morality''. See
\url{http://ur1.ca/f6rw}.}, a thought based very deeply on the
\textit{Malthusian statement}. Here he claims that common ownership (or 
rather – stewardship) will end in tears when the resources run out. But Hardin
is a post-industrial person saying that the existence of a commons was
contradictory to the assumption of scarcity. That with anything in common or
communal ownership, be it works in the public domain or resources not
specif\hbox{}ically allocated, there was a threat that the commons would wipe
themselves out. Given scarcity, people would take and take and never give.

Hardin, in making this statement, was doing game theory a big favour. Game
theory was a relatively fresh branch of mathematics made famous by Nobel
laureate John Nash, that inspected strategies and situations in terms of
\textit{games} played by \textit{players}. Examples of strategies developed
under game theory were minimax (commercialism: maximize the ef\hbox{}fect of
your actions and minimize the ef\hbox{}fect of those of your opponent) and
tit-for-tat (the cold war: if you launch nukes, so will we). Hardin produced a
strategy that was widely adopted, and it is known as the CC-PP game. CC-PP
stands for “Communize Costs-Privatize Prof\hbox{}its.” In this strategy you
leech of\hbox{}f the investments of your competitors, making the community as a
whole pay for as much of your own expansion as is possible, but at the same 
time making sure to keep all prof\hbox{}its for yourself by not divvying out
your booty to the rest of the pirates.

Exploring this within our system-circle (which has now admittedly become
something of a mess), what we're doing is pushing the system in directions that
will make others pay for our prof\hbox{}its. Who better to do this but the
government, which already has the legislative authority to do so?


\subsection{The Best Insurance Policy Ever}
\label{s:artificial_scarcity:unspoken_mythology:insurance_policy}

Say what you will about Friedman and co, but at least they were
honest\footnote{Well, no. But it's a good argument to make nevertheless.}. The
rest of the capitalists are playing the CC-PP game. Consider a few examples:
after the great depression John Maynard Keynes suggested ideas that became
rolled into Franklin D. Roosevelt's New Deal, which was accepted and performed
quite altruistically. But if we look at the situation, what was being done was
huge debts were being forgiven towards the people who caused the depression to
begin with and society as a whole was being made to pay. In Iceland in 2008, as
soon as the f\hbox{}inancial situation of the banks was regarded as ominous, 
the banks were – and get this – \textit{nationalized}\footnote{For more details
on this, see \url{http://ur1.ca/f6rx} and it's many references.}. The assets of
the banks were seized and the government put in direct control of the daily
operations of the bank.

The owners were magically freed from their already non-existent obligations
towards the f\hbox{}inancial stability of the bank, losing a pile of money that
didn't exist either anyway, and the full brunt of the debt that the owners had
created within the bank pushed onto the nation.

The exact same story happened with Fannie Mae and Freddie Mac, and any number 
of other examples come to mind. Would a bank ever be nationalized if it were
doing well? Not at all. Indeed, as was seen in Bolivia in 2001\footnote{See
\textit{¡Cochabamba!: Water War in Bolivia}, by Oscar Olivera and Tom Lewis.}
the obverse is true. Prof\hbox{}itable ventures, such as selling water to
peasants, tend towards privatization in any system that assumes scarcity of the
same.  Instant prof\hbox{}it!

The net result of the CC-PP game, in this instance, is the production of a
situation where the rich play by the Marxian rules and the poor play by the
Smithian rules: Socialism for the Rich, Capitalism for the Poor. If you just
happen to be one of the unlucky sods who doesn't own stocks and wear a \$5,000
suit to work, you're in a dog-eat-dog world and getting beyond that point will
always be problematic at best. Indeed, our cybernetic circle diverges into two
circles at an ever-accelerating rate, where one of the circles is a game plan
for the wealthy and the other is a game plan for the poor.

The government, then, is a tool being used by two factions to preserve their 
own dominance. For those who strive to increase their inf\hbox{}luence, a
government is a way to satisfy their egotistical yearnings. For the 
capitalists, a government is the best insurance policy other people's money can
buy.


\subsection{Manufactured Scarcity}
\label{s:artificial_scarcity:unspoken_mythology:manufactured_scarcity}

And all of this comes back to the underlying principles of the political
doctrines of Smith and Marx: Hobbes' Lie and Malthus' Lie. There are other 
lies, but these are the core, as far as I can tell. No other elementary
assumptions built into the system are as well def\hbox{}ined and as thoroughly
cherished by all parties.

In fact, government has been very busy enforcing these lies, upholding the 
myth.  Scarcity is the tool they use in conjunction with the owners as a method
for ensuring the subservience and subjugation of those not indoctrinated in
their world\footnote{I almost wrote \textit{of the working classes} here, but I
fear instigating a class war is a perfect way to maintain the \textit{status
quo}.  See any class war in history for examples of this.}. Scarcity in food 
and commodities by an inherently faulty distribution network, implicitly 
limited by people's lack of regard for one another and explicitly limited by
trade barriers, tolls, taxes and tarif\hbox{}fs. Scarcity in culture by the
conf\hbox{}inement of \textit{f\hbox{}ine art} and cultural events within the
lucrative boundaries of the cityscapes, as well as the projection of knowledge
into books – immutable and easily scarcif\hbox{}ied by the producers, who sell
at whichever price f\hbox{}its their fancy. 

Everywhere in the system, scarcity is being manufactured to insure the
prof\hbox{}iteers against the dangers of abundance. Working from Malthus' Lie,
the myth of scarcity is being upheld quite vigorously as a fundamental truth
about the nature of the universe, while elsewhere in the system people are hard
at work disposing of excess production and obstinate themes, colour schemes and
styles in favour of new.

An example of this is the production of academic textbooks. When a professor of
some f\hbox{}ield appears at the publishers with a manuscript for a new 
textbook on whichever subject, the publisher will explore the availability of
other similar textbooks, the originality, the readability and the depth of the
manuscript, and the statistics on how many people are likely to study such a
subject. After which they will decide on the price of each copy of the book in
such a way that they are destined to make a prof\hbox{}it. Quite reasonable,
assuming scarcity, but the idea of publishing the manuscript in a readily
copyable way has not caught on.

Why? Copyright.

Back in the time of Hobbes, copyright law did not exist\footnote{The
f\hbox{}irst example of copyright law in the modern sense being the Statute of
Anne from 1710.}. Mapmakers toiling day and night to copy out maps by hand for
ships to sail by and people to travel by were extremely jealous of their
property, and went to great extents to maintain their unequivocal right to
produce maps based on their particular data set, and as a copy-protection
measure they would mark in false roads, so called trap streets, or mangle names
of places, so that if another were to copy their maps they would be easily 
found out. Back in those days illegal copying wasn't a large problem, but
despite this the producers of the maps were damaging their products by
decreasing their accuracy in order to foil people who wish to mimic that
(in)accuracy.

This kind of early DRM\footnote{Digital Restrictions Management, or Digital
Rights Management, depending on who you ask. Generally speaking a technological
method intended to enforce copyright. These invariably fail for numerous
reasons. See \textit{Microsoft Research DRM} talk by Cory Doctorow,
\url{http://ur1.ca/f6s0}}, along with monopolies in the publishing
business\footnote{Held originally in Britain by the Worshipful Company of
Stationers and Newspaper Makers.} and later a succession of laws starting with
the Statute of Anne and the Berne Convention and moving through to legislations
such as the Sonny Bono act in the United States, copyright has been transformed
into a means of production, not of works of art, but of scarcity. Scarcity of
the very works of art it claims to protect. Before the advent of the printing
press and the phonograph, this was almost cute, since it was rarely worth the
hassle of copying data by illegal means anyway because of the shortcomings in
the technology. But with the further digitization of society, copying became
easier and easier, and the scarcity was upheld increasingly vigorously by the
lawmakers.

Imagine you live in a far away land where the penalty for stealing bread is
quite severe. You are starving, and so you attempt to steal a loaf, but are
caught bread-handed. This poor judgement on your part provides you with a ten
year prison sentence. Fair enough, 'tis the law of the land. 

But let's imagine that the day after you are incarcerated, a new technology is
invented. This new technology produces bread out of thin air at no cost to
anybody, in virtually inf\hbox{}inite quantities, and nobody need starve ever
again.  How just, then, is your incarceration? You stole the bread while bread
was still scarce, and there was no way of knowing that this technology was just
around the corner, so perhaps it is still fair; but obversely, if a law were
passed making it no longer criminal to steal the bread, would you not wish to 
be released?

No such law is passed, and a few years pass as you mull over these details in
your stinky cell, when suddenly a new prisoner appears. It is your brother, and
he has just been convicted of stealing bread. Outraged, you ask how can that 
be, since bread now exists in such plethora that nobody needs to steal bread?

Ah, your brother replies, it may well be that the technology exists to produce
bread at no cost to anybody, but it is still criminal to steal bread, and not
everybody owns a breadulator to make bread with. In fact, the bakeries that
produced the bread before have bought up all the breadulators and have claimed 
a patent for their design, so they can now prevent anybody from building their
own breadulator. Now bread costs the same as it did before, and it is of course
illegal to steal something that is scarce, be it from your neighbour or from 
the bakery.

This inane example illustrates in very silly terms how copyright works in the
digital age, and highlights one important aspect of it: that not only is our
sociopolitical system thoroughly dependent on the concept of scarcity, but the
producers who control the means of production will use their means to produce
scarcity as well as products, in order to maintain their worth in the system.

With each producer doing this, including the producers of money itself, the
system hangs in a balance where producers attempt to scarcify their produce to
maintain their worth relative to the prices of everything they themselves
require from other producers to survive. If anybody over-scarcif\hbox{}ies or
under-scarcif\hbox{}ies, there is chance of a crisis emerging. If it's food 
that is over-scarcif\hbox{}ied, people starve. If it's oil that's
under-scarcif\hbox{}ied, middle-eastern nations get invaded. If it's money
that's over scarcif\hbox{}ied, people stop trusting each other to maintain the
scarcity-equilibrium and the entire economy explodes.


\subsection{A Recipe for World War}
\label{s:artificial_scarcity:unspoken_mythology:world_war}

We're in our circle again, this time we draw a line against our will to the
point where we get a deep f\hbox{}inancial recession, just like in the 1930s,
just like in 2008. Then something weird occurs. In the cybernetic, this is
called a backlash. This is when a large and sudden change in the system causes
another sudden change in the system. A domino ef\hbox{}fect. Probability
theorists call these Markov explosions\footnote{Markov explosions occur in
stochastic processes when an inf\hbox{}inity of events occur simultaneously and
the system resets itself to a random state. There is a lot of deep literature 
on the subject that warrants scrutiny, but as an introduction for the
mathematically minded, I suggest \textit{Markov Chains} by J.R. Norris}. An
inf\hbox{}inite amount of events occur in the same instant, an apocalyptic
causality that devours every aspect of the system, and then, suddenly, it's
over. The world has changed.

In a post-depression world, a lot of people have a hard time getting their
bearings. Confused, people lash out against whatever they can f\hbox{}ind to
fault, be it the government, the owners of the means of production, or even
people from outside of their tribe, city, nation or other demographic group.
Increased nationalism is quite a typical result of f\hbox{}inancial crisis, 
look at World War I, World War II. Look at the Napoleonic wars. Each was
preceded by a spike in nationalism, which in turn was preceded by a
f\hbox{}inancial collapse of some type\footnote{The historical
justif\hbox{}ication for this claim is complicated. The Great Depression is
easy, but see also the implications of the 1873 panic following the crash of 
the Vienna Stock Exchange on Eastern Europe, and the ef\hbox{}fects of the
collapse of London banking house \textit{Neal, James, Fordyce and Down} in 1772
on Western-European trade, which led directly to the Boston Tea Party. Consider
Kondratiev waves in this regard.}.

The Napoleonic wars followed immediately from the French revolution, which in
turn followed bankruptcy in the French state. Simultaneously in the American
colonies f\hbox{}inancial instability was also a hot topic, which led to demand
for taxation with representation or no taxation at all. These events and others
like it culminated in extreme nationalism – the Americans wanted to be
Americans, the French wanted to rule everybody, the British wanted to rule
everybody, the Danish and Norwegians had problems f\hbox{}ighting of\hbox{}f 
the British while the Swedish and Russians and Prussians tried to f\hbox{}ight
of\hbox{}f the French. F\hbox{}inancial instability led to nationalism led to
world war. Is this not avoidable?


\section{Act 2. Burning the bridges when we get to them}
\label{s:artificial_scarcity:burning_bridges}

From the preceding pages we can learn a few things. The most important lesson 
is that the paradigms that form the basis of our mental models of reality can 
be built upon assumptions that are neither intended, apparent, nor correct. A
second is that all current forms of society and government are built around the
assumption of scarcity, and that scarcity can be shown not to exist any
more\footnote{Or at least be insignif\hbox{}icant. Further details of remaining
scarcity follows.}. The third is that because of these assumptions, all higher
dynamics within our system are fraught with terrible inequalities and
eventualities, namely poverty, famine, oppression, bankruptcy, prejudice and
war.


\subsection{Homogeneity and Censorship}
\label{s:artificial_scarcity:burning_bridges:homogeneity}

At the outset I made f\hbox{}leeting mention that increasingly potent copying
technologies had made creativity harder to accomplish, since accurate copying
leaves little room for embellishment. Constant and well-def\hbox{}ined data,
such as the text of the Constitution of the Swiss Confederation or the original
manuscript of a Harry Potter book is fairly resilient to \textit{ad-hoc}
editing, whether for creative or malicious reasons. In Orwell's \textit{1984}
the protagonist's occupation was to be a historical revisionist, altering all
distributed accounts of the past to meet the goals of the present.

Such alterations of available information cause people to be less able to
gingerly estimate their situation, especially if given evidence contradictory 
to what they know. Revisionism contaminates the state-space we live in and
ef\hbox{}fects our path through it like walls raised around us blocking other
exits. Governmental speed-bumps have been transformed into causeways, designed
to keep us forever within their boundaries at a speed that they can very easily
control.

In less abstract terms, this is the purpose of the Great F\hbox{}irewall of
China\footnote{A computer f\hbox{}irewall that f\hbox{}ilters all Internet
traf\hbox{}f\hbox{}ic passing within Chinese borders, allowing arbitrary and
even asymmetrical censorship by the government.} and other censorship tools,
including the less well known Swedish law that allows censorship of websites
considered to contain child pornography. The danger of such systems is that
there is no way to know what has been placed on such blacklists without
bypassing the censorship. Perhaps somebody has maliciously censored information
that could af\hbox{}fect the direction taken by the society with regard to
certain issues.

Censorship need not be absolute to be ef\hbox{}fective. Western governments 
have in recent decades realized that by applying knowledge of trends and
emotional reactions, they can avoid the need for censorship by simply placing
information out of sight. Press conferences confronting uncomfortable issues 
can be pushed to times of the day where they're unlikely to be televised, or if
televised not watched by many. Unpopular results, such as dioxin output from
industry, can be drowned in bureaucratic noise, such as measurements of other
less damaging chemicals, so that very few would be willing or able to plough
through the data looking for the bad results. In legislation unpopular motions
can be stacked up with more popular issues in sets, to hide them from scrutiny.

The point of this tangential discussion is that not only the mythology upon
which the system is built af\hbox{}fects the way we behave, but also the 
quality of the information available to us.

Memetics and indeed cybernetics is a dangerous f\hbox{}ield because of the
danger of misunderstanding. Faulty data can be worse than no data at all, as 
our credence for getting some output is generally high; it's only when we get
nothing – like those living behind the Great F\hbox{}irewall of 
China\footnote{A stunning feature of the Great F\hbox{}irewall of China is how
it feigns non-censorship. The HTTP protocol def\hbox{}ines error codes such as
200 (everything is okay), 500 (internal server error), 404 (f\hbox{}ile not
found) and 403 (unauthorized to access). When a censored page is accessed from
within the f\hbox{}irewall, instead of reporting 403, clearly stating that the
page has been censored, the f\hbox{}irewall reports 404, as if the censored
article did not exist at all.} – that we start to raise our eyebrows.

In our journey through the state-space of our reality, being pushed this way 
and that by cybernetic inf\hbox{}luences that we may or may not be aware of, we
are seldom aware of where we are going or what we will f\hbox{}ind when we get
there. A well drawn circle will allow people within to believe themselves to be
completely free whilst imposing fairly rigorous boundaries on what paths can be
taken.


\subsection{The Dance F\hbox{}loor}
\label{s:artificial_scarcity:burning_bridges:dance_floor}

An important feature of authority or control is that everything and everybody
has it, and it cannot be entirely eliminated. Authority will always necessarily
exist and cannot be done away with entirely\footnote{This may seem a
self-contradictory statement from somebody f\hbox{}lying the f\hbox{}lag of
anarchism, but it doesn't trouble me and if you understand where I'm going with
all this cybernetics talk, it won't trouble you either.}.

Consider a dance f\hbox{}loor. The dancers on this dance f\hbox{}loor are when
we gaze upon them paired up, one as the \textit{lead}, the other as a
\textit{follow}. Sometimes the couples break apart and singularly dance
freestyle, and sometimes dancers \textit{steal} partners from one another. The
objective shared by each of them is to solve a particular task, dancing, and
they do this by submitting control to others or taking control of\hbox{}f
others, but no single dancer can at any given time have complete knowledge of
the status of the entire dance f\hbox{}loor. Their knowledge is limited by 
their perception at any given point, but a dancer who perceives a potential
problem arising (such as a collision between two couples) or a solution (such 
as a fancy move) will take control of the vicinity momentarily to produce
results.

In this example – and it is a realistic one – although no individual or group 
of individuals has been designated as rulers over the others, authority still
exists. Each individual has complete authority over herself to begin with, but
as the dance progresses individuals may temporarily cede their authority to a
\textit{trusted interlocutor} in order to maximize gain.

The key here is that authority f\hbox{}lows between individuals in the system,
and manipulations of that authority can alter our collective path through the
system. Imagine a dance f\hbox{}loor where one person stood in the middle
yelling out orders, trying to micromanage the crowd. It would not function, as
even if we were to grant this single person the unlikely talent of complete
oversight, he would not be able to holler orders out fast enough. And if this
person were a choreographer who plotted all the movements beforehand, there
would be no spontaneity, and the dancing would have to stop intermittently to
allow for more choreography. Authority must exist, yes, but like any resource 
it must be well spent and fairly distributed. \textit{Ad-hoc} authority appears
to allow for the highest synergistic benef\hbox{}its, as the natural agreement
of all parties to the temporary authority will requisite the mutual
benef\hbox{}it of all parties.

This understanding of the nature of authority is a valuable tool to aid our
understanding of cybernetics: with this, we have not only established a model
for understanding peer-to-peer behaviour, but have also highlighted that any
stable system is necessarily and inherently creative. This will be important.


\subsection{Non-Rival Scarcity}
\label{s:artificial_scarcity:burning_bridges:non-rival}

A lot of what has been said can be traced back to a few people. Identifying the
villains of this story early on as Hobbes, Malthus and Hardin, the heroes
already mentioned are Godwin, Weiner and Beer, and now two more members of our
cast shall appear: George Pask and Richard Buckminster-Fuller.

Fuller is well known for his contributions to architecture and engineering, 
most notably the geodesic dome, but in his less well known book \textit{Nine
Chains to the Moon} he wrote of a process he dubbed ephemeralization, by which
he meant the way in which advances in technology would allow us to do more with
less.  Industrialization was exactly that: the advent of machines allowed 
people to produce more goods with less workforce behind the production; 
assembly lines allowed for more rapid assembly with less waste of time. 
Advances in materials science have given us carbon f\hbox{}ibre strengthened
plastics (CFSPs) that are both stronger and lighter than metals.

The Internet is the hallmark of ephemeralization: it allows us to perform
mind-boggling amounts of direct telecommunications and distributed computation
using a very elementary method of sending electrical or optical pulses through
copper and glass f\hbox{}ibre. More with less.

Malthus could not have imagined the industrial revolution, but he could have
paid attention to the trend of ephemeralization that Godwin appeared aware of,
even if he didn't have quite such a fancy word for it. Ephemeralization alone
kills the Malthusian argument entirely. We will be able to sustain an
increasingly large population by applying advances of our understanding of the
nature of reality to the aim of sustainability. Less will give us more, and
chaos is not a given.

This requires some hefty proof. Thankfully it is ample\footnote{See \textit{The
Wealth of Networks} by Yochai Benkler and \textit{The Democratization of
Innovation} by Eric von Hippel for much more proof than I shall provide here.}.

Things can be categorized into two categories: rival goods and non-rival goods.
Non-rival goods are not scarce by def\hbox{}inition, giving of them will not
diminish one's own supply. This applies to software and mp3s, but not to CDs 
and concert tickets. The latter are rival goods, but rival goods can be either
scarce or abundant, where we def\hbox{}ine abundance of a rival good not by
there being more than we need, but that the function of availability grows
faster than the function of need.


\subsection{Food}
\label{s:artificial_scarcity:burning_bridges:food}

One of the most profound examples of this comes from a research paper by
Perfecto, \textit{et al}\footnote{\textit{Organic agriculture and the global
food supply} , Ivette Perfecto, \textit{et al.}}, where it is shown that by
exchanging manufactured fertilizer with organic fertilizer, for certain crops 
it would be a simple matter to quadruple the annual yield, with multiplicative
results across the board. Add this to the earlier statement that we already
produce enough food even discounting meat, f\hbox{}ish and dairy products to
sustain humanity at its current level and still have leftovers, and it is clear
that we are not destined to starve to death any time soon. Food, our most basic
need, is a rival good, but can be considered abundant because it is currently
available in much greater quantities than is required, and because it appears
that technological advances will maintain this superiority in the food supply. 

The beauty of the food discussion is that it is so long since invalid. Peter
Kropotkin wrote in 1892 \textit{The Conquest of Bread}, wherein he points out
fallacies in feudal and capitalist economical systems in part by showing the
global abundance of food indisputably.


\subsection{Shelter}
\label{s:artificial_scarcity:burning_bridges:shelter}

Another of our basic needs is shelter. Globally we are faced with a housing
crisis, with an estimated 100 million homeless in highly developed
areas\footnote{See \textit{HUMAN RIGHTS: More Than 100 Million Homeless
Worldwide}, Gustavo Capdevilla,
\url{http://ur1.ca/f6s1}} and a further 600 million in
developing countries. Note here two things. F\hbox{}irst, there is 
approximately one starving person for each homeless person worldwide, but in
developed countries homelessness is disparate to hunger. Second, the Geneva
Convention grants prisoners of war rights to shelter, food and a blanket, 
whilst not a single government in the world has granted homeless people the 
same rights although they are granted by the Universal Declaration of Human
Rights\footnote{``Everyone has the right to a standard of living adequate for
the health and well-being of himself and of his family, including food,
clothing, housing and medical care and necessary social services, and the right
to security in the event of unemployment, sickness, disability, widowhood, old
age or other lack of livelihood in circumstances beyond his control.'',
Universal Declaration of Human Rights, Article 25.1.}. With the size of homes
having grown substantially in the western world over the last f\hbox{}ifty
years, there is absolutely no reason why there should be prevailing
homelessness. 

The argument made for homelessness is generally a lack or high cost of 
materials for building construction. One cause of this is the high standards
maintained by legislation in the form of building codes in some countries, 
where many forms of af\hbox{}fordable housing have been simply made illegal,
such as the Hexayurt infrastructure package\footnote{See Vinay Gupta's
\url{http://ur1.ca/f6s2}} and many other comparable projects\footnote{See
\textit{Architecture for Humanity} by Cameron Sinclair.}. Another cause is
luxuriation. In the city of Malmö, Sweden, authorities faced with a large 
number of lower and middle class people without adequate housing started a huge
project building expensive luxury homes along the southern waterfront. The 
logic was that with luxury homes available, upper class citizens would move to
these, freeing up cheaper homes elsewhere in the city for the lower and middle
class citizens. This is generally referred to as ``trickle-down'' economics,
where raising the standards for the uppermost echelons is expected to raise the
overall average to acceptable levels. 

The real result was that many of these luxury homes still stand vacant and most
of those which have been purchased were bought by upper class people from other
cities looking to own a second home. The housing problem was in no way averted
by these ef\hbox{}forts, but rather compounded as it resulted in less viable
land for development. If the issue had been dealt with directly the result 
might have been dif\hbox{}ferent.

Regarding material costs of housing, these can be severely reduced in a number
of ways. Jökull Jónsson \textit{et al} have shown that improvements to the
accuracy of the application of the Navier-Stokes  equations to structural
integrity estimation of concrete can yield signif\hbox{}icant strength
improvements with reduced materials volume and cost. Wallewik \textit{et al}
have shown that modif\hbox{}ications of concrete viscosity can increase spread
speed, allowing for much faster concrete pouring and setting. This could allow
for layered 3D printing of buildings in the future, but for the near term 
allows for much faster modular housing construction. Buckminster-Fuller showed
the feasibility of tensigrity structures in housing, which distribute 
structural load over the entire structure rather than on few key points, which
lowers the requirements for overall material strength. Vinay Gupta has 
developed a \$300 infrastructure package for temperate and tropic climates that
can house a small family in close quarters with acceptable living conditions.
Marcin Jakubowski \textit{et al} have shown that it is entirely possible to
build a single storey 100m$^2$ building from compacted earth blocks for less
than \$400 in materials costs in the American Midwest. Cameron Sinclair and his
Architecture for Humanity project have collected hundreds of examples of
ephemeralization in building construction and provided ample proof that current
methods of housing construction is both overly expensive and poorly organized.

Long story short, housing is not a problem any more than food. But what of 
other things? 


\subsection{Electronics}
\label{s:artificial_scarcity:burning_bridges:electronics}

Consumer electronics are an example of a f\hbox{}ield where decentralization is
currently extremely dif\hbox{}f\hbox{}icult, and yet profoundly simple.

The dif\hbox{}f\hbox{}iculty here lies in chip fabrication: the arrangement and
casting of specialized integrated circuits is a process that, by way of Moore's
law, requires increasing amounts of specialization each year. Current
microprocessors have circuit pitches of around 3$\mu$m in some cases, and this
is expected to decrease even more. Each order of magnitude reduction in circuit
pitch within ICs increases the complexity further as far as fabrication goes, as
they require increasingly pristine manufacturing conditions, including clean
rooms, high accuracy machine tools, and so on. However, three things may change
that.

The f\hbox{}irst is that with increasingly fast FPGAs, or F\hbox{}ield
Programmable Gate Arrays, unspecialised integrated circuits made in bulk can be
specialized \textit{in the f\hbox{}ield}, meaning that whichever specialization
is required can be def\hbox{}ined by the end user rather than it needing to be
def\hbox{}ined during the fabrication process. While FPGAs remain by far
inferior to specialized chips, they are already eating away at the second
factor, which is that hardware-level specialization is increasing overall 
whilst demand increase for generalized computing devices is slowing. This is 
due to desktop computing slowly losing out to laptop computers, and the 
ubiquity of hand-held devices such as mobile phones, music players and other
such gizmos.  All of these call for integrated circuits of a kind where one 
size does not f\hbox{}it all, which pressures the chip producers to develop
FPGAs even further or to develop smaller scale fabrication techniques. 

The third point is that current 3D printing technologies are already lending
ef\hbox{}fort towards arbitrary fabrication of circuits, and as this technology
develops it is inevitable that accuracy will increase, eventually to such a
level that printing out ICs may become feasible.

At any rate, the assembly of the end products has never been a problem in the
consumer electronics industry. The original personal computer was developed in 
a garage by Steve Wozniak and Steve Jobs, and this trend has held throughout 
the decades, albeit with some f\hbox{}luctuation, with a recent explosion in 
the hobby electronics industry giving new strength to user groups such as NYC
Resistor, magazines and e-zines such as \textit{Make Magazine} and
\textit{Instructibles}, and to open hardware projects such as the
Arduino\footnote{See \url{http://ur1.ca/f6s4}}. A lack of strict
regulations on electronics production has helped this a lot, although there is
signif\hbox{}icant barrier to entry into commercial production of consumer
electronics through safety regulations such as CE.


\subsection{Transportation}
\label{s:artificial_scarcity:burning_bridges:transportation}

Even the titanic automotive and aeronautic industries are starting to buckle
under stress from the decentralization movement, as open source cars, airplanes
and even tractors are seeing the light of day. As with housing, here regulations
are impeding progress. As Burt Rutan has commented\footnote{See
\url{http://ur1.ca/f6s5}}, increasing safety regulations in the aeronautics
industry have all but extinguished aircraft development, making progress
insanely slow even for large companies such as Boeing and Airbus. For small
groups aiming to build manned aircraft, secrecy is just about the only way to
avoid the transactional overhead put in placed by aviation authorities.
    
Automotive regulations are nowhere near as stringent, but in many countries
regulations for road safety are impeding reasonable developments. For example,
in many Asian countries such as India the auto-rickshaw is a very common mode 
of transportation, but it is almost inconceivable that such a device would be
allowed to drive on British roads.

With corporations such as General Motors having collapsed and the entire
ecosystem of transportation being overturned by smaller units like the C,mm,n
project and companies like Tesla, what is inevitable is the future realization
that these things can be done dif\hbox{}ferently.


\subsection{Exotic Objects and Real Scarcity}
\label{s:artificial_scarcity:burning_bridges:exotic_objects}

It's worth noting that there will always be scarcity for some things. I call
them \textit{exotic objects}. One example is the Eif\hbox{}fel Tower. You can
copy the Eif\hbox{}fel Tower exactly atom for atom, but it won't be the
Eif\hbox{}fel Tower, it'll just be a copy. Anybody who's been to Las Vegas 
knows that it isn't quite the same.  There's lots of things like that: Mona
Lisa, the Statue of Liberty \ldots more or less anything that is what it is for
cultural or historical reasons rather than physical reasons. My friend Olle
Jonsson called this \textit{aura}, which is neat: \textit{aura} can't be 
copied, although it can be manifested symbolically.

Scarce things versus abundant is a very important point. We tend to treat
everything as scarce and that's a very bad thing, but as we stop treating
abundant things as scarce things, we should also take note of which things
really are scarce and f\hbox{}igure out how we're going to treat them. Food
isn't scarce, but there's a limited amount of bauxite in the world and thus a
limited amount of aluminium. Likewise, things can be abundant globally but
scarce locally. Either way, taking stock of the exotic objects and the scarce
goods is important if we want to make the most of them and benef\hbox{}it those
who need them to the greatest degree.

But while we think of everything as scarce, we're going to waste a lot of
ef\hbox{}fort on trying to overcome scarcity that has been artif\hbox{}icially
generated, which is stupid.

The lesson to take from this is that we've been doing things in a way that is
manifestly stupid and there are innumerable examples in existence of how to do
things better. Conservatism will only bring a people so far, and we're past 
that point already. We've been crossing increasingly rickety bridges as we get
to them for far too long, and it's about time we burned them down and built new
ones to better places.


\section{Act 3. F\hbox{}ive steps, a spin, and a new tomorrow}
\label{s:artificial_scarcity:five_steps}

The foundations for the current society are the myths that underlie our entire
economy, the lies that structure our mental models, that guide us through the
state space. That without a centralized government our civilization will
fragment into particles and humanity will devour itself in a war of all against
all, and that without regulations on the distribution of goods we will consume
faster than we can produce and exterminate ourselves.

These myths have been compounded, mostly in good faith, by consolidation of
power and legislative systems that diminish people's ability to self-governance
on the one hand and ef\hbox{}fective utilization of resources on the other,
ef\hbox{}fectively the opposite of what these systems were meant to prevent.

The system we live by has f\hbox{}ive core institutions that I'd like to address
here brief\hbox{}ly.

The f\hbox{}irst of these is the monetary system. We live by a monetary system
that has, as Bernard Liataer pointed out\footnote{See \textit{The Future of
Money} by Bernard Liataer.}, four core features: money is created out of 
nothing and has no material backing, money is created as a result of loans
between banks, currencies are def\hbox{}ined geographically, and interest is
paid on loans. These features mean that the sum of the entire monetary system
(all debit plus all credit) is much less than zero, and it grows smaller
constantly. There is no way to repay all the debt in the system, and as a 
result money itself becomes a rival good – we are playing a game where the goal
is to pay all debts.  In this game, to lose is to go bankrupt. If many
bankruptcies occur simultaneously we suf\hbox{}fer a Markovian explosion of
sorts, called a depression or crisis.

The second of these institutions is our economy. This is dif\hbox{}ferent from
the monetary system: the monetary system is the means for exchange, while the
economy is the exchange itself. Because the means for exchange are rival goods,
the economy adapts by assuming rivalry and scarcity in all goods even when 
there is abundance. Competition replaces cooperation as each strives to pay
of\hbox{}f his debts, and companies and individuals use missing information –
that is to say, secrecy – to their advantage, to increase their chances of
winning, to get the competitive edge. Secrecy causes an inability to accurately
measure the state of the economy, an inability to relatively estimate demand 
and supply, so all companies guesstimate their production requirements and
invariably squander resources as a result. Companies are then punished for this
by the legislative system for certain types of waste while other types of waste
are not punished.

The third system is the legislative system itself: Small groups of people make
decisions about a set of rules that guide societies through the state space, 
and all are made to comply. The law represents the needs of the most
inf\hbox{}luential persons in the economy and legislation is guided by their
need to not go bankrupt. With every law which is passed, the Hobbesian lie is
strengthened, and the capitalists reinforce their insurance policy at the cost
of the poor.  Instead of the legal system being a small set of simple rules 
that everybody can agree to, it has become a behemothic beast, our very own
Grendel.

The fourth system is the executive authority system. A small group of people is
selected to make decisions about the execution of all the ideas they have about
how society as a whole ought to be run, and this authority reaches to every
niche of society. With regulations and exact control individuals are made to
suf\hbox{}fer their own individuality, trapped within a vicious cycle produced
for that very purpose in concordance with the Malthusian and Hobbesian
principles.

F\hbox{}inally, the judicial system has been erected to divvy out punishments 
to those who act against society, even in some cases for its own good. The
executive authorities select judges who make decisions about how arguments
should be resolved and these decisions, in many countries, become quite as
authoritative for future discourse as the law itself. Judges have become monks
who none may question.

This may be done dif\hbox{}ferently.


\subsection{Identity infrastructure}
\label{s:artificial_scarcity:five_steps:identity_infrastructure}

For our future society we must recognize that at our civilization's core are
individuals, not rules or money. People are the most important aspect of our
reality and everything should be based upon our needs.

The cornerstone of being attributed to the ``people'' group is currently the
acknowledgement of the government and the owners of banks and corporations of
one's existence, which is frequently circularly dependent, which gives one
access to the institutions listed above. A national census, a registration
of\hbox{}f\hbox{}ice, the publishers of bank accounts, birth
certif\hbox{}icates, passports and drivers licences, these are the
identity-management organizations of our society.

Understanding that identity underlies everything we are and everything we do is
paramount, without that understanding we are bound to remain in the current
system indef\hbox{}initely.

So I suggest a new system, one in which the individual is the alpha and the
omega, and greed and the production of artif\hbox{}icial scarcity is not
rewarded.

Step one is to alter the identif\hbox{}ication system. Rather than being
identif\hbox{}ied as members of society by a centralized institution, embroiled
in bureaucracy and haphazardly associated with the truth, we can use friendships
as def\hbox{}initions of identity. One's identity can be def\hbox{}ined by one's
friends more accurately than it can be def\hbox{}ined by an institution. This is
the philosophy of Ubuntu: ``I am who I am because of who we all are''. To
accomplish this we are going to need a bit of mathematics and a bit of
anthropology.

Michael Gurevich, Stanley Milgram, Benoit Mandelbrot and others\footnote{See
\textit{The Small World Problem} by Stanley Milgram. It should be noted that 
the idea has been largely debunked in its original form, but the level of
interconnectivity between people is still very high.} have suggested that in
human society connections between people are so dense that the longest path
between people is six steps. Malcolm Gladwell\footnote{See \textit{The Tipping
Point} by Malcolm Gladwell} has expanded on the \textit{six degrees of
separation} idea by identifying certain individuals as connectors – socialites
who are more accomplished than others in creating and maintaining connections
between people and who act as social hubs. Although the idea has been largely
debunked it still remains true that the maximum number of connections between
people appears to be a relatively low number. This matters when we consider the
social network.

A graph is def\hbox{}ined mathematically as a collection of vertices and edges.
If we let the vertices be people and the edges be friendships or acquaintances
between people, we call it a social network. The maximum number of connections
in a graph is def\hbox{}ined by the formula n(n-1)/2 for a graph of n vertices,
which basically means that for a graph of two vertices the maximum is one
connection, for three vertices the maximum is three, for four vertices the
maximum is six, and so on. For 150 vertices you have a maximum of 11,175
connections, for 300,000 vertices there are roughly 45 billion connections at
maximum.

The value of a network is def\hbox{}ined by Metcalfe's law as the ratio between
the number of connections and the maximum number of connections – how close are
you to a perfectly connected network. It is obvious that one person could not
have 300,000 friends, but if 300,000 people all had 300,000 friends, we would
have so many pairwise connections that it would be mind-boggling. This gives us
that in small cities (or countries such as Iceland) it is nonsensical to assume
that everybody will know each other. In fact, even in a town of 5,000 people
there would be twelve and a half million pairwise connections at maximum, which
is realistically unattainable.

The anthropologist Robin Dunbar found\footnote{See \textit{Neocortex size as a
constraint on group size in primates} by Robin Dunbar} a correlation between the
average number of members in a tribe of primates and the size of the brain.
Extrapolating from his acquired data, human tribes should have a weighted mean
size of 148 individuals\footnote{150 is frequently quoted as Dunbar's number.}.
Comparing this to real data of primitive tribes has shown this to be fairly
accurate in general, with tribes being known to split after having reached a
certain ``supercritical'' size. Applying technological mechanisms such as legal
and monetary systems, and even communications technology such as telephones and
the Internet has the potential to artif\hbox{}icially augment this
f\hbox{}igure, but hardly beyond a certain degree. The average number of friends
on Facebook is signif\hbox{}icantly higher than Dunbar's number\footnote{See
\textit{Facebook study reveals users 'trophy friends'} by Roger Highf\hbox{}ield
and Nic F\hbox{}leming, \textit{Daily Telegraph}.  \url{http://ur1.ca/f6s7}},
but the availability of telecommunications people more f\hbox{}lagrantly
befriend people, using assistive technology to maintain more friendships than
was previously possible; some have called this \textit{trophying}, but the truth
might simply be that we are far more socially motivated than our brains can keep
up with without assistance.

The point here is that our world is fairly small because of our ``limited''
cognitive capacity, and a perfectly isolated tribe of 150 may have 11,175
connections internally but in reality it is more likely that people will be
meshed globally, with relatively few connection steps between any given pair.

Let's make use of this, but before we do, let's do some cryptography. The RSA
algorithm\footnote{See \textit{A Method for obtaining Digital Signatures and
Public-Key Cryptosystems} by Ron Rivest, Adi Shamir and Leonard Adleman.} uses a
mathematical trapdoor function – something that is easy to do but very hard to
undo – to perform asymmetric encryption. Instead of a pair of individuals
sharing a secret they use to exchange other secrets, each publishes a public key
and maintains his own secret private key. The asymmetry can be used in many
ways. For encrypting, you apply the recipient's public key to a message, and to
decrypt the recipient applies his private key to the cipher text. For digital
signatures one applies one's private key to a message and to verify it one
checks against the public key.

If people in the social network generate key pairs and digitally sign public
keys belonging to their friends as a method both of verif\hbox{}ication of the
validity of the public key and to ``formalize'' the friendship (or
acquaintance). This way, your identity is established by your friends as you
establish theirs, in a peer-to-peer fashion, without any central authority.
This allows us to proceed with changing the world.

From this simple feature we get f\hbox{}ive results: A monetary system without
central banking, an economy without secrets, a legislative system without
elitism, an executive authority model without a government, and a judicial
system without courts.

I shall explain these results individually.


\subsection{Monetary system}
\label{s:artificial_scarcity:five_steps:monetary_system}

By utilizing the trusted network in a particular way we can construct mutual
credit currencies where business transactions happen like so: Alice wishes to
purchase a product from Bob. They decide on a price. Alice digitally signs the
invoice, and Bob then does the same. Each takes a copy and encrypts it to
themselves. This process can be simply obscured behind the ``put credit card in
card reader'' praxis we are all familiar with, or placed into cellphones or
other equipment.

What is happening when this occurs is quite technical, and yet it is quite as
simple if not simpler than our current monetary system. Essentially in every
transaction money is created by the parties to the agreement and debited to one
while being credited to the other, a loan. The sum of each transaction is thus
zero, and therefore the sum of the entire system is zero. Because the
transactions are small, frequent and symmetrical, it is nonsensical to resort 
to usury.

The idea that every single person in the system can create money appears weird
to people used to our current system. Today banks create money by lending money
they don't have to each other, which is an act of trust. In this suggested
system, if Bob does not trust Alice personally for the loan of this amount of
money, he can either deny her the transaction, or, more sensibly, traverse the
trusted network in search of a trusted connection that would allow for that
large a transaction. Some sequence of friends connect the two of them together,
and based on the amount of trust available between them, they can agree on the
debt. Bob trusts Carl who trusts Damien who trusts Eve who trusts Alice, and
through this sequence of friendships the business is conducted. Trust becomes
the backbone of the f\hbox{}inancial system – he who has many friends is a rich
man.

This is not much dif\hbox{}ferent from our current system, but it is stronger 
in that the failure of one node (a bank) is far less likely to disrupt the 
whole system.  Furthermore nobody need ever lose this game – the sum is zero,
and thus nobody will ever go bankrupt. Some may misuse other people's trust and
f\hbox{}ind it hard to f\hbox{}ind goodwill and credit, but notice that in this
system people are under pressure not to be untrustworthy!

At any given point in time the monetary system can be resolved, meaning that
circular debts can be nullif\hbox{}ied. If Alice owes Bob and Bob owes Carl and
Carl owes Alice, the smallest common value can be zeroed out. By traversing the
entire network every transaction can be nullif\hbox{}ied to some extent, and 
the result will show how far from the average each individual is (and at least
one person in the system can be at zero). This can be looked on as a measure of
how much a person has contributed to society. Furthermore, for simplicity it is
useful to resolve the system frequently, although resolutions may not be useful
if too frequent; this hinges on the level of activity in the economy.

Whilst remaining a f\hbox{}iat monetary system, this idea removes interest,
centralization and geographical restriction from the monetary system in one go,
and it does so simply by utilizing the trust af\hbox{}forded by our personal
relationships already.


\subsection{Economic system}
\label{s:artificial_scarcity:five_steps:economic_system}

One of the more destructive features of the economy as it is today is a result
of the monetary system. Our collective drive to repay our debts causes us to
attempt increasingly larger business transactions due to the time-ef\hbox{}fort
overhead of conducting any given transaction – maximizing the mark-up is
essential. Large sums are unlikely to be the norm in business in this system as
they are in our current system. For distribution purposes end-buyers are both
capable and incentivized to link up with producers directly. Middlemen serve
less of a purpose except as glorif\hbox{}ied stockpilers, who can be paid by 
the producers rather than the consumers to maintain a more localized cache of
goods. This would make sense for things such as tantalum, which is mainly mined
in the Congo, and may be scarce elsewhere, but would make less sense for things
such as capacitors, which, while made of tantalum, could essentially be made
anywhere.

Consumption in the economy is stabilized by this kind of ``bottom up'' rather
than ``top down'' transaction sequence. ``The rich \ldots consume little more
than the poor,''\footnote{See \textit{The Wealth of Nations}, Adam Smith} and
what little they do consume beyond the poor is a function of the opportunity
cost of consumption. Access to radically decentralized production and high
availability of skilled craft industries\footnote{See \textit{The Second
Industrial Divide}, Michael Piore \& Charles Sabel} can of\hbox{}fset that
opportunity cost by reducing the importance of the distribution subsystem.

Because it is no longer important for middlemen to compete for market dominance
and producers to worry about their market share of the demand curve (due to the
free availability of \textit{trust dollars}), not only can they strive to 
create better products that last longer, but they can also freely share
information amongst themselves about their production output, methods, and
demand; in fact it may even be favourable for them to gloat. This would provide
data for a readily available \textit{ad-hoc} worldwide information system
regarding the state of the economy as a whole, making futures markets more
prof\hbox{}itable, commodities markets less wasteful, and business in general
move faster and with less impedance. This is Staf\hbox{}ford Beer's CyberSyn:
predicting and resolving market-level and production-level problems before they
occur.


\subsection{Legislative system}
\label{s:artificial_scarcity:five_steps:legislative_system}

For this to work we need radical changes to the legislative system. By 
utilizing the trusted network we can build a form of direct democracy that does
not suf\hbox{}fer from the shortcomings of direct democracy that its opponents
will gladly point out. 

Granting everybody the ability to submit legislative proposals to the trusted
network, legislature itself can be crowd-sourced. Bills can be prioritized by
popularity (vote up/down) or reference counts (Pagerank) as a measure of
importance, and likewise bills can be altered and ``forked'' to create
derivative bills that can compete. This way anybody can contribute to the
options available to voters, for example ``yes'', ``no'' and ``broccoli'', with
the last of these being obviously silly and likely to be revised out in
subsequent edits.

Voters can choose the options on the bill, and when enough people have voted it
becomes \textit{validated}, meaning that the result of the popularity contest
between the available options is law. By allowing voters to change their vote 
at any time, law can change dynamically over time, perhaps with a mandated time
lag or signif\hbox{}icance factor put into the legal framework to cull
instability, which serves as a method to clean out laws that do not serve their
purpose or are obsolete. 

Similarly, when voters die their vote is discarded, and new voters also get to
have their say on any given bill. This causes the society at any given time to
be in agreement on the current state of legislature, at least to a
signif\hbox{}icant degree, rather than people being bound by historical
legislation that may now be counterproductive.

Elections on a given bill are performed by the vote being digitally signed and
encrypted to counting parties, which may be one or many, in the form of 
``double envelopes''. The signature identif\hbox{}ies the voter but by way of
encryption it is segregated from the vote itself, which protects vote secrecy.

Since votes can be changed at any time, election theft is almost impossible, as
voters can be asked to ``check their votes'' and people can not be violently
caused to vote a certain way as they can change them after the vote is 
complete, and killing people after they have voted will lead to the vote being
discarded.

This also means that there is no reason to impose arbitrary restrictions on
voter age: any born human can have a vote, and even if the parents use the 
votes of their children in any which way, the children can change their votes
whenever they have asserted their independence or come of age. Disparity 
created by families having more votes is minimal, as family sizes tend to 
reduce as prosperity increases, and in fact this provides families with 
children with a better footing in terms of social welfare and so on.

Here comes the smart part: not everybody, say the naysayers, is interested in
participating in all votes and claim to be apolitical. Traditional voting
systems provide for two exposed functions for interacting with ballots:
abstaining (or voting blank, which for our purposes can be considered the 
same), or selecting an option.

The third option, that eliminates much abstinence from apolitical people, is to
allow voters to proxy their votes, essentially selecting any third party to 
cast a vote on their behalf. This type of representation can be on a per-bill
basis, categorical, or total, and it can be revoked at any time.

Giving people the ability to defer to their peers in this way creates a highly
dynamic system in which every single organizational structure ever seen in 
human history exists as a state: parliamentary governments are a state in which
a small f\hbox{}ixed number of people get votes proxied to them in equal
measure; dictatorships or monarchies are the state in which all people grant 
one person with their vote (either directly or indirectly), and direct 
democracy is where nobody grants anybody their vote. None of these situations 
is incredibly likely, as the number of possible states within this system are
approximately two to the power of the number of voters.


\subsection{Executive system}
\label{s:artificial_scarcity:five_steps:executive_system}

Since the economical system has been restructured in such a way that personal
gain need not be enacted by way of greed, it is perfectly reasonable to remove
the concept of government entirely. Private entrepreneurship can be trusted to
fulf\hbox{}il all the roles of government without fear of there being
inequality; as long as private individuals and collectives thereof operate in
accordance to the law which they themselves have created, and conduct their
af\hbox{}fairs in whichever way will garner them the most trust outwardly, all
traditional functions of government are void save for a few.

The purposes of police and military can be replaced by private security
contractors, the purpose of foreign af\hbox{}fairs ministries can be replaced
with trade agreements enacted by syndicates, embassies operated as social
centres, and so on. 

Such ``privatization'' must not be misconstrued as the same kind of
privatization we've seen in propertarian governments in previous decades, where
banks, telephone companies and television networks have been placed wholesale
into the hands of prof\hbox{}iteering individuals for a fraction of their 
value, but rather, it is closer to the ideas of the anarcho-syndicalist ideas 
of free association and collective ef\hbox{}fort to solve problems facing
society or individuals within it. 


\subsection{Judicial system}
\label{s:artificial_scarcity:five_steps:judicial_system}

There not being any government poses a problem to all the lawyers and judges 
out there: without there being an executive authority to decide who they deem 
is capable of being impartial in every possible dispute, the entire system of
jurisprudence may falter. Nobody has the authority to select a judge – or,
perhaps it is everybody who has that authority.

Social contract or law may cause disputing factions to elect judges to try 
their case. An example of a method of electing judges would be that the
disagreeing parties would f\hbox{}ind the subset of the trusted network wherein
all members are four (to pick a number) or more steps from themselves, and six
(to pick a number) randomly selected members from that set are asked to act as
judges.  These people need not be lawyers, rather they would pass judgement
based on their convictions in light of the law, perhaps enlisting lawyers they
would hire to be \textit{their} legal counsels: the disputing parties would 
pool to pay for the proceedings.

With these changes it is not hard to envision an equally networked model for
education, health care, and so on. By utilizing the nature of the trusted 
social network we can ef\hbox{}fectively build a system that makes no
assumptions about the correct structure of society, allowing natural structure
to emerge. It may, at the end of the day, be similar or identical to our 
current system, but at least then we'll know.


\subsection{The Curtain drops}
\label{s:artificial_scarcity:five_steps:curtain}

Let's be clear: These are not idle thoughts. Many of these systems are being
tried, none of these ideas are new. It is the context that they are given that
provides them with novelty. The software required to enact these changes is
rapidly coming into existence, there are social movements popping up all over 
to enact these changes. They're not inevitable, but it'd take a \textit{force
majeure} to derail this train.

And it is here that the narrator leaves the stage and takes a seat amongst the
audience, and the audience becomes the stage, as the interactions of the actors
become the deepest plot of the most amazing drama, the most horrible tragedy,
the most delightful comedy, the best story ever. And this is no myth: this is
humanity, we are here, now, doing our thing, dancing to our tune, together.

I write these f\hbox{}inal words from the trenches of a complex network of
revolutions where our only opponents are our own broken assumptions and the
horrifying systems that run on them. But rather than being muddy and stinky and
littered with our fallen comrades, these trenches are digital landscapes of
unending variety, a tribute to human creativity. They are the hallmark of all 
we have accomplished.

All around us the ancient strongholds of broken systems are falling. In 
Iceland, where I live, our government just crumbled and a new one has taken its
place, a left wing liberal environmentalist government headed by a lesbian
socialist, and it looks like a few months down the road we may start drafting a
new constitution, where direct democracy might be the result.

In Belgium, yet another government has failed; in the United States a liberal
black progressive president just took of\hbox{}f\hbox{}ice in the middle of a
f\hbox{}inancial crisis that may dwarf the Great Depression. In Thailand people
have taken matters into their own hands, in India there are calls for general
strikes. In Sweden, youth movements are squatting empty buildings in the middle
of a housing crisis. In Afghanistan people are fabricating equipment to mesh
together wireless networks, unleashing the power of the Internet. In Zimbabwe
the currency has become so devalued that all currencies have been made equally
valid, in neighbouring Malawi the government has decided to ignore the World
Bank's demand that agriculture not be subsidized, and have surplus yield for 
the f\hbox{}irst time in decades.

Throughout the world the story is the same: our capacity for self-governance is
being uncovered, in part due to lessons learned from the Internet and the 
social movement that runs it. Hackerdom and its particular kind of meritocratic
anarchism, having birthed the free software movement, the free hardware
movement, and the free culture movement, having liberated technologies, built
the largest encyclopaedia ever seen, and revolutionized communications and
computation in every way – having done all that, our movement is now moving 
into wider pastures and tackling the broken foundations of our society itself.
And it's about time.

We're here to change the world, nothing more. This is how it starts. Good luck.
