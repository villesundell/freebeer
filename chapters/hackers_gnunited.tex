% Copyright 2009 FSCONS, Superflex and the individual authors.
% This entire book and all its source files is licenced under Creative Commons Attribution-ShareAlike 2.5
\begin{savequote}
    \qauthor{\LARGE{Johan Söderberg}}
\end{savequote}
\chapter{Hackers GNUnited!}
\label{c:hackers_gnunited}

\section{The political left and the politics of hackers}
\label{s:hackers_gnunited-political_left}

In this article I will look at hacking from a trade union perspective. The
political signif\hbox{}icance of computer hacking has puzzled the old left,
though there are some communicating bodies between the hacker movement and
traditional, social movements. Most noticeable are those groups within the
computer underground calling themselves 'hacktivists'. They want to apply their
computer skills in furthering an already established political agenda, such as
feminism or environmentalism\cite{gnunited-jordan02}. More challenging is making
sense of the political agenda of the mainstream of the hacker movement. One
immediately comes up against the question of does the computer underground
qualify as a social movement at all. Many hackers, perhaps the majority, would
say that this is not the case. At best, politics is held to be secondary to the
joy of playing with computer technology\cite{gnunited-torvalds_diamond01}. Even
so, out of this passionate af\hbox{}f\hbox{}irmation of computers have grown
ideas with political ramif\hbox{}ications. For instance, hackers who otherwise
do not consider themselves as 'political' tend nevertheless to be opposed to
software patents and state surveillance on the Internet, to mention just two
examples.  Indeed, these viewpoints are so widely shared in the computer
underground that they look more like commonsense than political stances. Some
issues, such as campaigns against the expansion of intellectual property laws
and the defence of freedom of speech, have been added to political agendas and
are actively promoted by hacker lobby groups, two examples of which are the Free
Software Foundation and the Electronic Frontier Foundation. These organisations
are clearly involved in politics, though they claim that these interests cut
along dif\hbox{}ferent axes than the traditional right-left divide. When social
scientists have analysed the assumptions which lay behind the public statements
of these hacker lobby groups however, they have usually found a close
af\hbox{}f\hbox{}inity with liberalism\cite{gnunited-coleman08}. 

A couple of leftist writers have broken ranks in that they do not interpret
hacking as a liberal ideology. Quite to the contrary, they believe that the
hacker movement could revitalise the old struggles of the left, not just for
individual freedom but also against injustice and inequality. The most renowned
insider who has voiced such opinions about hacking is Eben Moglen. He is a law
professor and was for a long time a senior f\hbox{}igure in the Free Software
Foundation. Moglen is also the author of \textit{The DotCommunism Manifesto},
where he predicted that the anarchism of free software development would replace
capitalist f\hbox{}irms as the most ef\hbox{}ficient mode for organising
production in the future\cite{gnunited-moglen99}. The media scholar Richard
Barbrook reasoned in a similar way when he was debunking the hype about 'free
markets in cyberspace' which was touted in the 1990s. Instead he presented his
own vision of a high-tech, anarchistic gift economy. The impulse to give would
follow automatically from the fact that people on the Internet had a
self-interest in sharing information freely rather than trading it on a
market\cite{gnunited-barbrook02}. Arguably, the rise of Napster and later
generations of f\hbox{}ile-sharing technologies could be said to have proven
Barbrook right. Even more iconoclastic in his embrace of socialist rhetoric is
the Slovenian philosopher Slavoj Zizek. He has paraphrased Lenin's endorsement
of electricity by stating, tongue-in-cheek, that 'socialism equals free access
to the Internet plus power to the Soviets'\cite{gnunited-zizek02}. At least a
few old-time communists are taking this idea seriously. They believe that
computer technology has provided the missing link which at last could make a
planned economy a viable alternative to the market
economy\cite{gnunited-pollack98}.

But these positive af\hbox{}f\hbox{}irmations of hacking and computer technology
are probably minority opinions within the traditional left. There is a deeply
rooted suspicion among leftist intellectuals towards computer technology and, by
extension, its most zealot users, i.e. hackers. The Internet's origin in
American cold war institutions is suf\hbox{}f\hbox{}icient to put of\hbox{}f
many progressive thinkers\cite{gnunited-edwards96, gnunited-shiller99}. Add to
that the hype surrounding the Internet in the mid-1990s. It gave new lease to
the old chestnut about the 'Information Age'. This notion dates back to the
1950s and conservative American sociologists who set out to disprove the
continued relevance of class conf\hbox{}licts.  By announcing an end to
industrial society, they wanted to prove that tensions between the classes had
been dissolved and the ideological struggle between liberalism and socialism was
becoming obsolete.  Consequently, left-leaning scholars have protested against
notions about the rise of an Information Age and insisted on the continued
existence of industrialism, capitalism, and class
conf\hbox{}lict\cite{gnunited-webster02}. To make this point they have only to
call attention to the inhuman conditions under which computer electronics are
manufactured in export zones in third world
countries\cite{gnunited-sussman_lent98}. A report from 2008 has documented how
girls in China as young as 16 years old are working twelve to f\hbox{}ifteen
hours a day, six or seven days a week, and barely earning a
living\cite{gnunited-weed08}. These f\hbox{}indings resonate with the historical
circumstance that punched cards, numerical control machinery, mainframes, and
other embryos of modern computers were instrumental in making blue-collar
workers redundant and degrading craft skills at the point of
production\cite{gnunited-braverman74, gnunited-kraft77}.

Now, having brief\hbox{}ly outlined the perplexed relation between the
traditional left and the political thrust of hackers, this article will proceed
by examining the political signif\hbox{}icance of hackers in the light of an old
debate about factory machinery and labour. The Braverman Debate, as it is known
after the author who started the controversy, harks back to the 1970s. Harry
Braverman published a book where he argued that the deskilling of labour was an
inherent quality of capitalism. The reason was that managers strove to become
independent of highly skilled workers in order to keep wages down and unions
politically weak.  Braverman found support for his hypothesis in the writings of
the pioneers of management philosophy. The pivotal f\hbox{}igure among them,
Winston Taylor, had laid the foundation of what is now known as
'scientif\hbox{}ic management' or 'Taylorism'. A central idea of
scientif\hbox{}ic management is that the shop-f\hbox{}loor ought to be
restructured in such a way that tasks can be done with simple routines requiring
a minimum of skills from employees. Taylor argued that this could be done
through the introduction of factory machinery. Braverman showed how this
strategy was being deployed in heavy industry during the mid twentieth century.

This insight can serve as a lens for looking at the political
signif\hbox{}icance of computer machinery and the hacking of it. The novelty of
this argument is that its analysis of hackers is formulated from a
production-oriented perspective, as opposed to a consumer rights perspective. It
will be argued that the rise of Free and Open Source Software (FOSS) can be
traced back to the industrial conf\hbox{}lict between managers and workers.
Furthermore, the similarity between the struggle of workers against factory
machinery and the struggle of the hacker movement against proprietary software
will be highlighted. Free access to source code, a key concern of hackers,
contradicts the factory system and the logic of scientif\hbox{}ic management in
computer programming\cite{gnunited-hannemyr99}.  Though the situation of
programmers compared to blue-collar workers is very dif\hbox{}ferent in many
respects, the article notes that both groups are preoccupied with the goal of
preserving skills and worker autonomy in the face of rapid technological change.
Hackers' demand that source code should be freely accessible can be interpreted
as part of a strategy which is aimed at preserving the programmer's know-how and
his control over the tools of his trade. 


\section{The machine at work}
\label{s:hackers_gnunited-machine_at_work}

The ambivalent feelings of enthusiasm and fear which computer technology often
evokes among people have a historical precedent. At the dawn of the industrial
revolution, it was hotly debated in all quarters of society what mechanisation
would do to the human being, both socially and
spiritually\cite{gnunited-berg80}. Even some of the forerunners of liberal
economic theory, such as David Riccardo, admitted that the working class had
good reasons for being resentful of factory
machinery\cite{gnunited-riccardo1821}. The wretchedness which befell workers who
were subjugated under machinery and factory discipline was vividly described by
James Kay, a social reformer who worked as a doctor in the slums:

\begin{quote}
``While the engine runs the people must work – men, women and children are yoked
together with iron and steam. The animal machine – breakable in the best case,
subject to a thousand sources of suf\hbox{}fering – is chained to the iron
machine, which knows no suf\hbox{}fering and no
weariness.''\cite{gnunited-kay1832}
\end{quote}

Early management writers like Andrew Ure and Charles Babbage welcomed this
opportunity and advised factory owners how to design machinery in order to keep
workers docile and industrious\cite{gnunited-ure1835, gnunited-babbage71}. Their
testimonies informed Karl Marx's analysis of capitalism. He denounced factory
machinery as 'capital's material mode of existence'. But he also qualif\hbox{}ied his
critique against technology by adding that: ``It took time and experience before
the workers learned to distinguish between machinery and its employment by
capital, and therefore to transfer their attacks from the material instruments
of production to the form of society which utilises those
instruments.''\cite{gnunited-marx76}. Thus Marx renounced the strategy of
machine breaking which had been the hallmark of the Luddites. The Luddites
consisted of combers, weavers, and artisans who felt that their trade was
threatened by the introduction of new looms and a subsequent reorganisation of
the textile industry. Nightly raids were conducted to smash wool mills and
weaving frames owned by 'master weavers'. These activities culminated in
1811-1813 and at one time the English Crown had to deploy 14,400 soldiers in the
region to crush the nightly insurgencies. Quite remarkably, more English
soldiers were mobilised against the Luddites than had been sent to Portugal four
years earlier to face Napoleon's army\cite{gnunited-sale95}. In his classic
re-examination of the Luddite uprising, Eric Hobsbawm showed that the breaking
of machines was not a futile resistance against technology and progress, as it
was later made out to have been. Instead he interpreted it as a method of
'collective bargaining by riot'. Breaking the machinery was one option, but
workers could also put pressure on their employers by setting f\hbox{}ire to the
warehouse or sending anonymous threats. Hobsbawm concluded that, if judged by
the ability of workers to preserve their wages and working conditions, they had
been moderately successful\cite{gnunited-hobsbawm52}.

The misreading of the Luddite rebellion as deranged, irresponsible, and, most
importantly, as having nothing at all to do with politics, resembles the
portrayal of hackers in news media today. Andrew Ross has protested against the
image of the hacker as a petty criminal, a juvenile prankster, or,
alternatively, a yuppie of the Information Age. He stresses that spontaneous
sabotages by employees contributes to most of the computer downtime in
of\hbox{}f\hbox{}ices.  These attacks often go unreported since managers prefer to blame
external adversaries. With this observation in the back of his mind, he suggests
a much broader def\hbox{}inition of hacking: 

\begin{quote}
``While only a small number of computer users would categorize themselves as
'hackers', there are defensible reasons for extending the restricted
def\hbox{}inition of \textit{hacking} down and across the case hierarchy of
systems analysts, designers, programmers, and operators to include all high-tech
workers – no matter how inexpert – who can interrupt, upset, and redirect the
smooth f\hbox{}low of structured communications that dictates their position in
the social networks of exchange and determines the pace of their work
schedules.''\cite{gnunited-ross91}
\end{quote}

Andrew Ross' suspicion is conf\hbox{}irmed by studies conducted by employers'
organisations. Personnel crashing the computer equipment of their employers is a
more common, more costly, and more dreaded scenario for f\hbox{}irms than the
intrusion by external computer users. According to a survey in 1998 conducted
jointly by Computer Security Initiative and the FBI, the average cost of a
successful computer attack in the U.S. by an outsider was \$56,000. In
comparison, the average cost of malicious acts by insiders (i.e. employees) was
estimated to \$2.7 million\cite{gnunited-shell_dodge02}. The fondness of
employees for attacking the computer systems of their employers underlines the
role of computerisation in transforming the working conditions of white-collar
of\hbox{}f\hbox{}ice workers. Ross' comparison with sabotage will certainly
raise some objections among 'real' hackers. Those of the hacker movement who
want to be 'f\hbox{}it for the drawing room' try to counter the negative media
stereotype of hackers by dif\hbox{}ferentiating between original hackers and
so-called crackers. The former name is reserved for creative uses of technology
which contributes to socially useful software projects. The negative
connotations of computer crime are reserved for the latter group\footnote{For
instance, the Jargon f\hbox{}ile, which is considered to be the authoritative
source on hacker slang, goes out of its way to distinguish between crackers and
'real' hackers: \url{http://ur1.ca/f6o3} (accessed: 27-05-2009)}.

These ef\hbox{}forts at improving the public relations of hackers merely underline
the historical parallel with labour militancy suggested above. The trade union
movement too has rewritten its own history so that sabotage, wildcat  strikes
and acts of violence are left out of the picture. Indeed, unions have been very
successful in formalising the conf\hbox{}lict between labour and capital into a
matter of institutionalised bargaining. The case could be made, nonetheless,
that the collective bargaining position of labour still relies on the unspoken
threat of sabotage, strikes and riots\cite{gnunited-brown77}. In the same way, I
understand the distinction between hackers and crackers to be a discursive
construction that does not accurately portray the historical roots and the
actual overlapping of the subculture. Rather, it seeks to redef\hbox{}ine the meaning
of hacking and steer it in one particular direction. In spite of the success of
this rhetoric, it is nevertheless the case that the release of warez, the
breaking of encryptions, and the cracking of corporate servers play a part in
the larger struggle to keep information free.

Having said this, the reader would be right in objecting that the motivation of
Luddites and workers for rejecting factory and of\hbox{}f\hbox{}ice machinery is very
dif\hbox{}ferent from the motivation of hackers who are f\hbox{}ighting against
proprietary software. For the latter group, computers reveal themselves as
consumer goods and sources of stimulus. Arguably, their relation to technology
is one of passion rather than hostility. Even when hackers (crackers) sabotage
corporate servers, it is an act out of joy. Discontented of\hbox{}f\hbox{}ice workers
might also take some pleasure in destroying the computer of their employer, but
it is still meaningful to say that their act springs from resentment against
their situation. This dif\hbox{}ference in motivation does not, however, rule out the
possibility that hackers share some common ground with machine breakers of old.
Both are caught up in a struggle which is fought out on the terrain of
technological development. It might even be that the passionate af\hbox{}f\hbox{}irmation
of technology by hackers of\hbox{}fers a more subversive line of attack, in
comparison to, for instance, the insurgency of Luddites. Though it is incorrect
to say that Luddites were against technology \textit{per se}, it is true that
they defended an outdated technology against a new, scaled-up factory system.
Thus it appears in hindsight as if their cause was doomed from the start.
Hackers, in contrast, have a technology of their own to draw on. They can make a
plausible claim that their model for writing code is more advanced than the
'factory model' of developing proprietary software.


\section{Deskilling of workers, reskilling of users}
\label{s:hackers_gnunited-deskilling}

It is a strange dialectic which has led up to the current situation where
hackers might reclaim computer technology from companies and government
institutions. Clues as to how this situation came about can be sought in a
retrospective of the so-called Braverman Debate. The controversy took place
against the backdrop of the idea about the coming of a post-industrial
age\cite{gnunited-bell73}. Two decades later, the same idea was repackaged as
the 'rise of the Information Age' or the 'Network Society'. This notion has come
in many hues but invariably paints a bright future where capitalism will advance
beyond class conf\hbox{}licts and monotonous work. Crucially, this transition has not
been brought about through social struggle but owes exclusively to the inner
trajectory of technological development. Harry Braverman targeted one of its key
assumptions, namely that the skills of workers would be upgraded when
blue-collar jobs were replaced with white-collar jobs. He insisted that the
logic of capital is to deskill the workforce, irrespectively whether they are
employed in a factory or in an of\hbox{}f\hbox{}ice. Instead of a general upgrading of
skills in society, he predicted that the growth of the so-called 'service
economy' would result in white-collar of\hbox{}f\hbox{}ice workers soon confronting
routinisation and deskilling just as the blue-collar factory workers had done
before.

\begin{quote}
``By far the most important in modern production is the breakdown of complex
processes into simple tasks that are performed by workers whose knowledge is
virtually nil, whose so-called training is brief, and who may thereby be treated
as interchangeable parts.''\cite{gnunited-braverman98-318}
\end{quote}

His statement was rebutted by industrial sociologists. They acknowledged that
deskilling of work is present in mature industries, but argued that this trend
was counterbalanced by the establishment of new job positions with higher
qualif\hbox{}ications elsewhere in the economy. At first sight, the emergence of the
programming profession seems to have proven the critics right. One of the
critics, Stephen Wood, reproached Braverman for idealising the nineteenth
century craft worker. Wood pointed at the spread of literacy to prove that
skills have also increased in modern society\cite{gnunited-wood82}. His comment
is intriguing since it brings into relief a subtlety that was lost in the heated
exchange. It is not deskilling \textit{per se} that is the object of capital,
but to make workers replaceable. When tasks and qualif\hbox{}ications are
standardised, labour will be cheap in supply and lack political strength. From
this point of view, it doesn't really matter if skills of workers level out at a
lower or higher equilibrium.  Universal literacy is an example of the latter. 

Literacy in this regard can be said to be analogous to present-day campaigns for
computer literacy and calls for closing the 'digital gap'. In a trivial sense,
skills have increased in society when more people know how to use computers. One
might suspect that a strong impetus for this, however, is that computer literacy
reduces a major inertia in the scheme of 'lifelong learning', that is, the time
it takes for humans to learn new skills. Once workers have acquired basic skills
in navigating in a digital environment, it takes less ef\hbox{}fort to learn a new
occupation when their old trade has become redundant. This somewhat cynical
interpretation of computer literacy can be illustrated with a reference to the
printing industry. The traditional crafts of typesetting and printmaking took
many years to master and it required large and expensive facilities. The union
militancy which characterised the printing industry was founded upon this
knowledge monopoly of the workers. The introduction of computer-aided processes
was decisive for breaking the strength of typographic
workers\cite{gnunited-zimbalist79}. Personal computers can be seen as an
extension of this development. Software mediation allows the single skill of
navigating in a graphical interface to translate into multiple other skills.
With a computer running GNU/Linux and Scribus, for instance, the user is able to
command the machine-language of the computer and can imitate the crafts of
printmaking and typesetting. Very little training is required to use these
programs compared to the time which it took for a graphical worker to master his
trade. This suggests how computer literacy reduces the inertia of human learning
and makes the skills of workers more interchangeable. Liberal writers interpret
this development as an example of linear growth of learning and education
corresponding with the so-called 'knowledge society'. From the perspective of
labour process theory, quite to the contrary, the same development is seen as a
degradation of the skills of workers and ultimately aimed at weakening the
bargain position of trade unions. 

David Noble's classic study of the introduction of numerical control machinery
in heavy industry in the mid twentieth century provides the missing link between
Braverman's argument about deskilling and the current discussion about computers
and hackers. One thing which his study sheds light on is how the universality of
the computer tool was meant to work to the advantage of managers. Their hope was
that it would weaken the position of all-round, skilled machinists.
Special-purpose machinery had failed to replace these labourers, since
initiatives had still to be taken at the shop-f\hbox{}loor to integrate the separate
stages of specialised production. In contrast, general-purpose machines
simulated the versatility of human beings, thus it was better f\hbox{}itted to
replace them\cite{gnunited-noble84}. This historical connection is important to
stress because it is now commonplace that the universality of computer tools is
assumed to be an inherent quality of information technology itself. Thus the
trajectory towards universal tools has been detached from its embeddings in
struggle and is instead attributed to the grace of technological development. 

Saying that does not oblige us to condemn the trend towards a levelling out of
productive skills and the growth of universal tools such as computers. On the
contrary, in sharp contrast to the negative portrayal of Harry Braverman as a
neo-Luddite, Braverman reckoned that the unif\hbox{}ication of labour power caused by
machinery carried a positive potential. 

\begin{quote}
``The re-unif\hbox{}ied process in which the execution of all the steps is built
into the working mechanism of a single machine would seem now to render it
suitable for a collective of associated producers, none of whom need spend all
of their lives at any single function and all whom can participate in the
engineering, design, improvement, repair and operation of these ever more
productive machines.''\cite{gnunited-braverman98-320}
\end{quote}

With a universal tool, the computer, and the near-universal skill of using the
computer, the public can engage in any, and several, productive activities. It
is from this angle we can start to make sense of the current trend of 'user
empowerment'. In other words: Displacement of organised labour from strongholds
within the capitalist production apparatus, through a combination of deskilling
and reskilling, has prepared the ground for computer-aided, user-centred
innovation schemes. Because programs like \textit{Inkscape} and
\textit{Scribus}, and their proprietary equivalents, are substituting for
traditional forms of typesetting and printmaking, a multitude of people can
produce posters and pamphlets, instantly applicable to their local struggles.
Companies have a much harder time controlling the productive activity now than
when the instruments of labour were concentrated in the hands of a few, though
relatively powerful, employees. What is true for graphic design equally applies
to the writing of software code and the development of computer technology. Here
the Janus face of software comes to the fore: the very f\hbox{}lexibility and
precision by which software code can be designed to control subordinated workers
the same ease allows many more to partake in the process of writing it. Though
embryonic forms of computer technology, such as numerical control machinery,
were introduced at workplaces by managers in order to free them from their
dependency on unionised and skilled workers; as a side-ef\hbox{}fect, computer
technology has contributed to the establishment of user-centred production
processes partially independent of managers and factories. The free software
development community can be taken as an illustration of this.


\section{Free software as a trade union strategy}
\label{s:hackers_gnunited:fs_trade_union}

The corporate backing of the Free and Open Source Software (FOSS) development
community must be seen against the background of a restructured labour market.
During the last few decades, industrial sociologists have documented a trend
where the factory is losing its former status as the role model of production.
The point of production has become increasingly decentralised and spread out in
a network of subcontractors, freelancers, work-at-home schemes, and
franchisees\cite{gnunited-mcchesney_wood_foster98}. Companies can now add
volunteer development communities to the list of heterogeneous forms for
contracting labour. Or, saying it with a catchphrase, labour is outsourced and
open sourced. The opportunity to drastically cut labour costs for software
maintenance has attracted government institutions, vendors, service providers,
and hardware manufacturers to FOSS. The savings that are made by giants such as
IBM, the U.S. Army, and Munich city, to mention a few high-prof\hbox{}ile cases, has
created the space for specialised software f\hbox{}irms to sell free software
products and services. This analysis is consistent with Tiziana Terranova's
critical remark that the engagement of free labour has become structural in the
cultural economy. She protested against the many hopes and claims made about the
trend of active media consumption, f\hbox{}irst celebrated in the cultural studies
discipline from the 1980s and onwards and most recently updated with the hype
around Web 2.0. In response to these often unfounded claims, Terranova responded
that capital has always-already anticipated the active consumer in its business
strategies\cite{gnunited-terranova00} (2000). Her argument provides a corrective
to the uncritical appraisals of the fan f\hbox{}iction subculture, the creative
commons licence, and other expressions of 'participatory media'. Nevertheless,
in my opinion, left-leaning critics like Terranova have been too eager to cry
out against the economic exploitation of volunteer labour and have thus failed
to see the potential for political change which also exists in some of these
cases.

The relevance of my objection has to be decided on a case-by-case basis. While I
concede that the interactivity of video games and the volunteer ef\hbox{}forts of fan
f\hbox{}iction writers is unlikely to result in any substantial political change, the
interactivity and the gift-giving of free software developers cannot be tarred
with the same brush. Here it must be taken into account that the software code
is given away together with a clearly articulated, political goal: to make free
software the standard in computing. It is true that this standpoint is not
anti-commercial in a straightforward sense. As is probably known to the reader,
the General Public Licence (GPL) protects the right of the user to run software
for any purpose, including commercial purposes\cite{gnunited-gay02}. In
practice, of course, this option is limited by the fact that GPL also allows
sold copies to be copied and given away for free. While the free licence resides
perfectly within an idealised free market, it is ungainly within the actually
existing market which always presupposes quasi-monopolies and state
regulations\cite{gnunited-polanyi01}.

This goes some way to explain why the political right is in two minds about free
software licences. Self-acclaimed libertarians, such as Eric Raymond, see the
growth of open source business models as a better approximation of the free
market. Behind this assessment lies an understanding of capitalism as basically
identical with its institutions, i.e. private property, free markets and
contracts. But that outlook disregards another possible def\hbox{}inition of
capitalism which puts stress on capital as self-expansion of money, or, in other
words, accumulation. The latter viewpoint is central to Marx's analysis of
capitalism, but it is also closer to the concerns of the 'captains of industry'.
With that in mind, it can be interesting to take notice of market research which
\textit{claims that the adoption of FOSS} applications by businesses are eating
into the annual revenues of proprietary software vendors by \$60 billion per
year. Crucially, the losses to proprietary software companies are
disproportionate to the size of new FOSS markets, for the simple reason that a
lot of it is not paid for.\footnote{The market research rapport referred to is
called Trends in Open Source and has been published by the Standish Group.
Because access to the material is restricted, information about it comes from
news media\cite{gnunited-broersma08}}. Hence, the opposition against FOSS from
parts of the industry is not necessarily as misplaced as it has often been made
out to be. This opposition reached a climax in the court case between the SCO
Group and corporate vendors of GNU/Linux which came to an end in 2007.  During
the court case, the executive of\hbox{}f\hbox{}icer of the SCO Group, Darl
McBride, wrote an open letter to the American Congress where he accused his
competitors of being naïve in supporting FOSS licences: 'Despite this, we are
determined to see these legal cases through to the end because we are
f\hbox{}irm in our belief that the unchecked spread of Open Source software,
under the GPL, is a much more serious threat to our capitalist system than U.S.
corporations realize.'\footnote{\url{http://ur1.ca/f6o4} (accessed:
01-11-2009)}.

At the very least, these worries among some parts of the computer industry show
that free software developers cannot be written of\hbox{}f as mere unsuspecting
victims of commercial exploitation. Perhaps it would be more justif\hbox{}ied to say
that hackers, by freely of\hbox{}fering up their labour, are blackmailing
corporations into adopting and spreading the FOSS development model. No company
answering to the market imperative of lowest costs can af\hbox{}ford to argue against
free (as in free beer) labour. My hypothesis is that advocacy for free licences
can be interpreted in the light of an emerging profession of computer
programmers. This suggestion is far from obvious since the identity of the
hacker is tied up with the notion of being a hobbyist, or, in other words, a
non-professional, non-employee. Contradicting this self-image, however, numbers
have it that the majority of the people contributing to free software projects
are either working in the computer industry or are in training to become
computer professionals\cite{gnunited-lakhani_wolf05}. Hence, it is not so
far-fetched to connect the dots between hackers and the labour market that
awaits them. Indeed, this line of reasoning has already been attempted in Josh
Lerner and Jean Tirole's famous article\cite{gnunited-lerner_tirole02}. They
wanted to square the supposed altruism of free software developers with the
assumption in neo-classical economic theory about the 'rational economic man'.
The two authors concluded that hackers are giving away code for nothing in order
to create a reputation for themselves and improve their chances for employment
at a later date. Without denying that such cases may exist, I disagree with the
assumption of methodological individualism that underpins their thinking. When I
say that free software licences might be benef\hbox{}icial to the labour interests of
computer programmers, I do not mean that this is a rationally calculated
strategy or that it is an exhaustive explanation as to why hackers license their
software under GPL. Furthermore, in contrast to Lerner and Tirole, I do not
think that those labour interests are pursued exclusively through individual
strategies. In addition to improving their own reputation, individual hackers
are contributing to changing the labour market for programmers as a collective. 

It sounds counter-intuitive that programmers would improve their bargaining
strength vis-a-vis f\hbox{}irms by giving away their work to potential
employers. Let me start by returning to an insight of Harry Braverman. He
stressed that the very outlay of the factory put the machine operator at a
disadvantage. The worker could only employ skills when given access to the
machinery.  Unfortunately, the scale and mode of organisation of the factory was
already biased towards hierarchy. The capitalist had an advantage due to the
ownership of the machines and buildings, without which the workers could not
employ their abilities. The only bargain chips that the workers had were their
skills and intimate knowledge of the production process. This was also how
Braverman explained the tendency that capitalists are pushing for technologies
which reduce skilled labour. What has happened since Harry Braverman made his
analysis in the 1970s is that the large-scale Fordist machine park has grown
obsolete in many sectors of the economy. This is particularly true in the
computer industry. Productive tools (computers, communication networks, software
algorithms, and information content) are available in such quantities that they
have become a common standard instead of being a competitive edge against other
proprietors (capitalists) and a threshold towards non-possessors (workers). A
horde of industrial sociologists and management philosophers have written about
this trend since the early 1980s\cite{gnunited-zuboff88}. It is a truism in this
body of literature to claim that the employees, not the machine park, are
nowadays the most valuable resource of the modern corporation. The claim is
clouded in rhetoric, but the validity of the statement can be tested against the
adoption of 'non-disclosure agreements' within the computer industry. It is here
stated that the employee is not allowed to pass on sensitive information about
the f\hbox{}irm. Another kind of clauses which are sometimes included in the
employment contract to much the same ef\hbox{}fect, i.e. to prevent leakages,
forbid the programmer from working with similar tasks for a competitor after
having left his current employer. These agreements can be taken as testimonies
that the knowledge and skills of the programmers have indeed become increasingly
precious to the f\hbox{}irm to exercise control over. I will argue that these
practices, though they formally have very little to do with copyright law,
nevertheless brace up my claim that proprietary and free licences af\hbox{}fect
the bargaining position of software developers.

The justif\hbox{}ication for these dif\hbox{}ferent kind of contractual agreements is the
necessity of preventing trade secrets from leaking to competitors. However, as a
side-ef\hbox{}fect, the programmers are prevented from moving freely to similar
positions in their trade. Since the programmer becomes a specialist in the
f\hbox{}ield in which he has been working, he might have dif\hbox{}f\hbox{}iculties in finding
a job in a dif\hbox{}ferent position. The signif\hbox{}icance of this observation becomes
clearer against the background of Sean O'Riain's ethnographic study of a group
of software technicians working in a computer f\hbox{}irm in Ireland. It has proved
to be very dif\hbox{}f\hbox{}icult for trade unions to organise these workers. Since jobs
are provided on a work-for-hire basis, the collective strategies of unions lack
purchase. One of O'Riain's conclusions is that mobility has instead become the
chief means by which the employees negotiate their working conditions and
salaries\cite{gnunited-oriain04}. With awareness of this fact, the
signif\hbox{}icance of the contractual agreements mentioned above must be
reconsidered. The limitations which they put on the ability of employees to
'vote with their feet' means that the f\hbox{}irms get the advantage back. As to what
extent non-disclosure agreements and other clauses are actually used in the
Machiavellian way sketched out here is something which remains to be
investigated empirically. What interests me in this article, however, is that
the very same argument can be applied to proprietary software licences more
generally. 

Intellectual property\footnote{Many critics of copyright and patent law reject
the words 'intellectual property'. In their opinion, the words are loaded with
connotations that mislead the public. Instead they advocate the words
'intellectual monopoly'. I am unconvinced by this argument though there is no
space to develop my counter-position here. It suf\hbox{}f\hbox{}ices to say that I will
use the words 'intellectual property' in the article as I think that the
association with other kinds of property is entirely justif\hbox{}ied} too is
justified by the necessity of f\hbox{}irms to protect their knowledge from
competitors. A complementary justif\hbox{}ication is that intellectual property is
required so that producers can charge for information from consumer markets. But
intellectual property is also likely to af\hbox{}fect the relation between the f\hbox{}irm
and its employees, a subject which is less often discussed. A case can be made
that proprietary licenses prevents the mobility of employees. It ensures that
the knowledge of employed programmers is locked up in a proprietary standard
owned by the f\hbox{}irm. A parallel can be drawn with how the blue-collar worker
depends on the machine park owned by the industrialist.  Without access to the
factory the worker cannot employ his skills productively.  In the computer
industry, as was mentioned before, most of the tools that the programmer is
working with are available as cheap consumer goods (computers, etc.). Hence, the
company holds no advantage over the worker by providing these facilities. But
when the source code is locked up behind copyrights and software patents, large
amounts of capital are required to access the programming tools.  As a
consequence, the software licence grants the f\hbox{}irm an edge over the
labourer/programmer. This theoretical reasoning is harder to prove empirically
than the claim made before that clauses in the employment contract might be used
to restrict the mobility of programmers. Even so, it might be of an order of
magnitude greater in importance to the working conditions in the computer
sector. Indeed, this production-oriented aspect of proprietary licences might be
as signif\hbox{}icant as the of\hbox{}ficially touted justif\hbox{}ications for intellectual
property law, i.e. to regulate the relation between the f\hbox{}irm and its customers
and competitors. If I am correct in my reasoning so far, then the General Public
Licence should be read in the same light. I was led to this thought when reading
Glyn Moody's authoritative study of the FOSS development model. He makes the
following observation concerning the exceptional conditions for f\hbox{}irms
specialised in selling services in connection to free software: 

\begin{quote}
``Because the 'product' is open source, and freely available, businesses must
necessarily be based around a dif\hbox{}ferent kind of scarcity: the skills of
the people who write and service that software.''\cite{gnunited-moody01}
\end{quote}

In other words, when the source code has been made publicly available to
everyone under the GPL, the only things which remain scarce on the market are
the skills required to employ the software tools productively. And this resource
is inevitably the faculty of 'living labour', to follow Karl Marx's terminology.
It is thus that the programmers can get an edge over the employer when they are
bargaining over salary and working conditions. The free licence levels the
playing f\hbox{}ield by ensuring that everyone has equal access to the source code.
Terranova and like-minded scholars are correct in pointing out that
multinational companies have a much better starting position when exploiting the
commercial value of free software applications than any individual programmer.
The savings that IBM makes from running Apache on its servers are, measured in
absolute numbers, many times greater than the windfalls bestowed on any
programmer who has contributed to the project. Still, at a second reading, the
programmer might be better of\hbox{}f if there exists a labour market for free
software developers, compared to there being no such occupation available. By
publishing software under free licences, the individual hacker is not merely
improving his own reputation and employment prospects, a point which has
previously been stressed by Lerner and Tirole. He also contributes to the
establishment of a labour market where the rules of the game are rewritten, for
him and for everyone else, in his trade. It can be interpreted as a kind of
collective action adapted to a time of rampant individualism.

It remains to be seen if the establishment of a labour market in free software
development translates into better working conditions, higher salaries and other
benef\hbox{}its otherwise associated with trade union activism. Such a hypothesis
needs to be substantiated with empirical data. Comparative research of people
freelancing as free software programmers and those who work with proprietary
software is much wanted. Such a comparison must not, however, focus exclusively
on monetary aspects. As important is the subjective side of programming. An
example hereof is the consistent f\hbox{}inding that hackers report that it is more
fun to participate in free software projects than it is to work with proprietary
software code\cite{gnunited-lakhani_wolf05}. Neither do I believe that stealth
union strategies are the sole explanation as to why hackers publish under GPL.
Quite possibly, concerns about civil liberties and the anti-authoritarian ethos
within the hacker subculture are more important factors. Hackers are a much too
heterogeneous bunch for them all to be included under a single explanation. But
I dare to say that the labour perspective deserves more attention than it has
been given in popular press and academic literature until now. Though there is
no lack of critiques against intellectual property law, these objections tend to
be formulated as a defence of consumer rights and draw on a liberal, political
tradition. 

There are, of course, some noteworthy exceptions. People like Eben Moglen,
Slavoj Zizek and Richard Barbrook have reacted against the liberal ideology
implicit in much talk about the Internet and related issues. They have done so
by courting the revolutionary rhetoric of the Second International. Their ideas
are original and eye-catching and often rich with insight. Nevertheless, the
revolutionary rhetoric sounds oddly out of place when applied to pragmatic
hackers. Advocates of free software might do better if they look for a
counterweight to the hegemony of liberalism in the reformist branch of the
labour movement, i.e. in trade unionism. I believe that such a strategy will
make more sense the more the computer industry matures. In accordance with Harry
Braverman's general line of argument, the profession of software engineering has
already been deprived of much of its former status. Indeed, from the early 1960s
and onwards, writers in management journals have repeatedly been calling for the
subjugation of programmers under the same factory regime which had previously,
and partly through the introduction of computer machinery, been imposed on
blue-collar workers\cite{gnunited-dafermos_soderberg09}. With this history in
the back of the mind, I would like to propose that the advocacy of free
software, instead of falling back on the free speech amendment in the American
Constitution, could take its creed from the 'Technology Bill of Rights'. This
statement was written in 1981 by the International Association of Machinists in
the midst of a raging industrial conf\hbox{}lict:

\begin{quote}
``The new automation technologies and the sciences that underlie them are the
product of a world-wide, centuries-long accumulation of knowledge. Accordingly,
working people and their communities have a right to share in the decisions
about, and the gains from, new technology.''\cite{gnunited-shaiken86}
\end{quote}


\section{Acknowledgements}
\label{s:hackers_gnunited:acknowledgements}

The author would like to thank the editor, Stian Rødven Eide, as well as Michael
Widerkrantz and Don Williams, for constructive comments on earlier drafts of
this paper.
