\begin{savequote}
    \qauthor{\LARGE{Stefan Larsson}}
\end{savequote}
\chapter[The darling conceptions of your time]{The darling conceptions of your
time,\\ \Large{or: Why Galileo Galilei sings so badly in the chorus}}
\label{c:darling_conceptions}


\section{Law, social change and conceptions}
\label{s:darling_conceptions:law_social_change_concetions}

``People in power get to impose their metaphors'', wrote Lakof\hbox{}f and Johnson in
their ground-breaking work Metaphors we live by, on structures of metaphors and
concepts and the manifest part in human thinking and communication that
metaphors and concepts play. They strengthened the idea that human thought
processes are mainly metaphorical and said that the ``human conceptual system is
metaphorically structured and def\hbox{}ined''. By ``metaphor'' they actually meant
``metaphorical concept''\cite{darling-lakoff_johnson80}. Their work inspired
many disciplines to develop in this direction. 

Conceptions, like metaphors, carry with them a heritage of the context from
which they were derived. They are not always easily translated from one context
to another without some kind of distortion. One can go even further: conceptions
and metaphors are ways of thinking. They describe the way we understand life,
our world and our place in it. The problem is that metaphors and conceptions can
be both informative and deceptive. They can be taken from a context where they
function well, to be used in a context where they deceive and distort (see for
instance \cite{darling-morgan99}). The starting point of this article is that
conceptions can be tied to a specif\hbox{}ic world order, to a way in which a society
is organized: in its politics, administration, government and, very importantly,
its regulation. This leads to what the title asserts: societies change and the
conceptions that have been more or less deeply founded in them can face problems
when translated into a new context. This article uses the examples of f\hbox{}ile
sharing and Internet and copyright legislation to show the clashes of such a
societal transition and the conceptions embedded. And it does this via the
lyrics of a song about the astronomer Galileo Galilei. Before I go into detail
on this perhaps unexpected diversion I want to elaborate the role of technology
in relation to social norms and legal regulations. 

This article is about metaphors, or rather conceptions, and about law and social
change connected with technology. Technology often has an important role in
social and normative transitions\cite{darling-vago09}. Digital technology has
changed the conditions of communication and has therefore caused a changed
behaviour in society in connection to what can be perceived as normative change,
for instance regarding f\hbox{}ile sharing of media content. To illustrate the battle
of conceptions tied to this I use the example of stealing/sharing. What from an
analogue perspective is seen as theft, an action with highly negative
connotations, is from a digital perspective seen as something else, with less or
no negative connotations. Normatively, one could say that these actions are not
comparable. Technology can be seen as the prime mover of the social changes
creating the contemporary copyright dilemma. I am focusing on technology in the
sense that other parallel processes that are part of the paradigmatic transition
are neglected (for a grander picture, see \cite{darling-castells96,
darling-castells97, darling-castells98}, and for a stronger focus on law and
legislative paradigmatic change in a global perspective, see
\cite{darling-santos95a, darling-santos95b}), but I am still interested in the
consequences of how technology rearranges society and creates various conditions
for norms. 

Each society regulates dif\hbox{}ferently. One can here talk about rules of the game.
Every society, like every game, has its own set of rules that def\hbox{}ine that
society or that game. Historically, social evolution has often been connected to
technological innovations. The combustion engine took a central position in what
later became known as the industrialized society, an urbanizing era of factories
and production, following the rural society tied to agriculture and trade (see
\cite{darling-fridholm84, darling-sundqvist01, darling-ewerman_hyden97}). With
each type of society comes a specif\hbox{}ic type of legal ``darling'' conceptions tied
to the patterns of behaviour relevant for this type. Some conceptions are in
conf\hbox{}lict when society changes, some new conceptions emerge. 

In general, some of the conceptions embedded in law and the debate around, for
instance, f\hbox{}ile sharing are dependant on the preconditions of reality, which also
form the conceptions that are used in legal regulations. The aim of this article
is to highlight and describe a few of the conceptions that have been developed
under conditions for communication and media distribution other than what
prevails today. A fact that creates a tension between regulation and reality.
But, what has the song I mentioned about Galileo Galilei to do with this?

When working on an article in Swedish for an anthology published in the fall of
2008, I decided, being both a socio-legal scholar and a musician, to write a
song that pedagogically illustrated the problem both in its lyrics and in the
fact that it was to be released under the Creative Commons Licence Attribution,
non-commercial. Both the book, \textit{FRAMTIDSBOKEN: vol
1.0}\cite{darling-hyden08}, and the song were released online and could be
downloaded freely. It meant that the song was neither buyable nor sellable
(according to the licence). It could not be used for commercial activities
without my consent. You could say that the song embraced the power of the f\hbox{}low,
rather than the f\hbox{}low of power. It was, and of course still is, shareable,
searchable and downloadable.

A couple of principally very interesting conceptions that create a high amount
of tension in society today are tied to online behaviour, content distribution
and legal regulation. The idea of letting a song display the issue is
pedagogically of double interest. I use a song because it is a question of
transition and the music medium will here illustrate change. It also illustrates
the search for darling conceptions of our time, by revealing, discussing and
challenging them. It is also a test. To practically look to the ideas of
creative commons licences as a way for creators to make the rights granted by
law – copyright law – a little less protective by the consent of the creators,
and likely a little more adapted to the practice of Internet, f\hbox{}ile sharing and
f\hbox{}low of media. You could say that the song forms a meta-pedagogical display: it
both tells the story of societal transition in terms of a battle of conceptions,
as well as in itself exemplifying a contemporary issue regarding legal
regulations and social change when released for free sharing online. The song is
about Galileo Galilei and is called \textit{The darling conceptions of your
time}.


\section{Galileo Galilei and the Darling conceptions of your time}
\label{s:darling_conceptions:galileo_galilei_darling_conceptions}

Conceptions and metaphors are ways of understanding things. They can be the
results of a social construction, meaning that it is not a matter of true or
false. It is a construction made to serve a purpose. A metaphor, for example, is
not necessarily more true because it has been around for a longer time than a
newer one.

Let us turn to the f\hbox{}irst two verses of the song that will continually (and
f\hbox{}ictitiously) play along while the reader reads the article. Picture a three man
combo playing in the corner of a bar. Every now and then a few lines of what
they are singing are heard through the murmur of the crowd scattered throughout
the room. You see a double bass, hear the soft snare drum and suddenly a voice
starts to sing: 

\begin{quote}
\textit{I see a learned man watching the sky}\newline
\textit{His mind is forming a question}\newline
\textit{He trembles when he starts to realize}\newline
\textit{There is something wrong with how the sun passes the sky}\newline
\textit{There is something wrong with how the sun passes the sky}\newline
\newline
\textit{The court declared the conviction}\newline
\textit{and the mumbling crowd awaited no reply}\newline
\textit{It expected no contradictory claims}\newline
\textit{There is nothing wrong with how the sun passes the sky}\newline
\textit{There is nothing wrong with how the sun passes the sky}
\end{quote}

These are the two opening verses of the song ``The darling conceptions of your
time''. Think of the famous astronomer Galileo Galilei as the ``learned man
watching the sky''. Galileo Galilei found out something that clearly challenged
a darling conception of his time. Earth was not central in the planetary system
surrounding us in space, the sun was. In addition to this, he proved this bold
statement empirically. He constructed a pair of binoculars, made the
mathematical calculations, and concluded that he had a new truth to reveal. The
earth was not in the centre of the universe as we know it. The planets can not
be revolving around the earth: ``Earth is revolving around the sun, and I have
seen it!'' The Church was outraged (on Galilei, see for instance
\cite{darling-naess07}).

A remarkable fact is that he was not even the f\hbox{}irst one to make the claim.
Copernicus had mathematically come to the same conclusion a couple of years
earlier. That is why it is called the Copernican view. He did not however look,
empirically measure and see that the sun could not be rotating around earth. He
was also not punished as harshly by the Church, which also acted as a court, as
was Galileo. Galileo came to a cross roads where he had to choose between the
truth, as he had investigated it empirically, and the law, which found his deeds
to be wrong. To challenge some of the darling conceptions can be experienced as
a challenge to the system, which was likely in this case. It was not merely
about the planetary organization in space, it also questioned who should be the
true interpreter of the order of things. It was about who should have power over
the conceptions that should rule as truth. Galileo challenged this and as a
result had to choose between standing by his f\hbox{}indings and risking his life or to
deny what he regarded as true and staying alive.

He chose life. Maybe truth seemed a little less important when faced with the
risk of being burned on a pile of wood. Maybe truth even seemed a little less
right. ``And still it is moving'', he allegedly said very quietly, sitting on
his chair on a podium, surrounded by a hostile and mumbling mob on either side
and behind him. In front of him sat the tribunal, which is the court of the
Church, and the very same court that had accused him. Galilei spent his
remaining days in house arrest.

As indicated by the very f\hbox{}irst sentence in this article, the one from Lakof\hbox{}f and
Johnson, the conceptions that prevail have some kind of connection to power. The
law is a commonly used instrument of control by the State. A successful law not
only imposes behaviour, but also often conceptions of how the world is and
should be arranged. However, in a connected world the centralised power is
challenged in some aspects. The social norms that control behaviour on the
Internet do not necessarily apply to a legislation that functioned well in a
pre-digital era. As put by Castells:

\begin{quote}
``\ldots the power of f\hbox{}lows take precedence over the f\hbox{}lows of
power.''\cite{darling-castells00}
\end{quote}

It has to do with a transition, the view of the world, and what the
prerequisites are when it comes to communication between peers and distribution
of media content. One could express it as if earth is the natural scientif\hbox{}ic
depiction of our planet and the world is the social construction that social
science deals with. There are structures in society –  legal, economic and
social – that interact and depend on each other. When prerequisites drastically
change, there is a need for a new balance in these structures. F\hbox{}inding this
balance takes time, and will create winners and losers along the way. This
applies, for instance, to the structures of news and media production in a
centralised society, as it shifts towards a more decentralised version of
possibilities in f\hbox{}inding alternative media, alternative broadcasts, alternative
methods of production, or even co-production of media content. This rips the
keys out of the hands of the former key holders within news organisations,
governments and media producers. Social science has to deal with the conceptions
embedded in the conf\hbox{}lict, to sort out the old and describe the new that may take
its place, just like Galileo. Over time, the strong inf\hbox{}luence of the Church
declined and its role as the interpreter of truth regarding earth's place in
space was lost. The scientif\hbox{}ic approach evolved, a school of reason and
empirical sciences took a greater place in society.


\section{The battle of what the Internet should be}
\label{s:darling_conceptions:battle_internet}

In a historical sense, the Internet is very new. The impact of digitalisation
has however in a short time led to what Castells describes as the Network
Society. How the Internet was designed in terms of what type of information that
would be embedded in the communication was paradigmatically dif\hbox{}ferent from how
most legal regulation and legal systems have been constructed. Legal systems
generally operate in a national domain, relying on information regarding where
an action has taken place geographically, as well as the age of a person if
there is a special relation between involved individuals etc., in order to f\hbox{}ind
out if the action was criminalised or not, as well as how hard the actions
should be penalised within given restrictions. The Internet lets people act
across national borders without revealing their ages, whereabouts or what
relationships people have. The communication is, or at least has been, this
free. This type of freedom, or lack of control, is under attack from strong
legislators throughout the world, where the traditional media industry is a
heavily investing instigator and lobbyist. More layers of control over the f\hbox{}lows
of the Internet mean that existing analogically preconditioned models for the
market can survive. On the other side stand the critics claiming that the
control needed for these models to still function is such an utterly
over-dimensioned control that it threatens grand values such as privacy and free
speech. Questions that need to be addressed here are what balance should we
strive for, what is lost and what is gained when more aspects of control are
added to the layers of the Internet? And in the case of copyright, is this for
the sake of creativity or for the sake of an industry with an aged market model?
In order to understand this we need to take a brief look into the copyright
construction.


\section{Copyright}
\label{s:darling_conceptions:copyright}

The origin and growth of copyright as a legal concept is intertwined with the
technical development in regards to the conditions for storing and distributing
the created media; the melody one wrote and recorded, the book, the photograph
and so on. If we focus on music, we will see how copyright and technology have
developed side by side. But also, which is interesting to note, how creativity
itself is inf\hbox{}luenced by the preconditions in technology. One purpose of
copyright is the creation and development of culture (if we want to dig into
Swedish law-making history, the preparatory work for the Swedish copyright law
states this, SOU 1956:25 s 487). The legal regulation in itself has no
justif\hbox{}ication in addition to stating systemic conditions that are culturally
stimulating and ensuring future innovations.

Copyright law is amazingly homogeneous throughout the globe as a result of
international co-operation with treaties and conventions. Both the European
Union and the U.S. have added to a strong and homogeneous copyright throughout
major parts of the world. A few of the characteristics that can be found in most
national copyright legislations are that:
\begin{itemize}
    \item{the period of protection lasts the life of the copyright holder + 70
        years (sometimes 50, see the Berne Convention and the TRIPS
        Agreement\footnote{Berne Convention for the Protection for Literary and
        Artistic Works, last amended at Paris on 28 September, 1979. Sweden
        signed on 1 August 1904 and has adopted all the amendments of the
        Convention after that. Agreement on Trade-Related Aspects of
        Intellectual Property Rights signed in Marrakech, Morocco on 15 April
        1994.})}
    \item{the period of protection for those companies who own the recordings
        (related rights) are mostly 50 years (see the Rome
        Convention\footnote{The International Convention for the Protection of
        Performers, Producers of Phonograms and Broadcasting Organizations.})}
    \item{no registration is needed to achieve copyright when something is
        created (disputes will be settled in court. The U.S. used to have some
        demands – the year and the \copyright{} symbol, but that is less
        important these days when everyone has signed the same treaties)}
    \item{copyright means exclusive rights to the created for the creator or the
        holder of these rights (which is a very important distinction) that are
        economic – for instance control over the copies and to sell them – and
        moral – that is to be attributed (mentioned) and not have the work
        ridiculed, for instance}
    \item{the exceptions from these exclusive rights are for ``fair'' use in the
        U.S., which is the sharing of copies to \textit{a few} friends, like in
        the Swedish regulation, within the private sphere. All depending on what
        type of creation and for what circumstance. The line is drawn a little
        dif\hbox{}ferently in dif\hbox{}ferent countries}
\end{itemize}

These characteristics have mainly been developed during the twentieth century
and are very much tied to a technological development that has allowed
distribution of content\footnote{Of course, printed material reached a
distribution revolution after the Gutenberg press and legal protection and the
ideas of copyright has been around before the twentieth century. But it was the
1886 Berne Convention that set out the scope for copyright protection which
originally meant maps and books but today has grown to become a signif\hbox{}icant
regulated conception in relation to sound recordings, f\hbox{}ilms, photographs,
software etc.}. These characteristics have been developed in an analogue setting
where heavy investments were needed for most of the production, reproduction and
distribution. Some of the characteristics show examples of being darling
conceptions of an industrialized society which has been embedded in incredibly
well-spread, global and strong regulations. At the same time, some of these
characteristics are now challenged due to the changes in preconditions for
production, reproduction and distribution that the digitalisation and rise of a
network society contributes to. 

An example: the concepts and specif\hbox{}ic terminology of Swedish copyright stems to
some extent from the preparatory works of 1956, prior to the Copyright Act from
1960 (it speaks of the expanding possibilities of reproducing sound with
innovations such as the magnetophon – basically an early and huge tape
recorder). Of course, the act has continuously been changed over the years, but
many of the terms are still used. This development has led to a legal regulation
that is so complex that even legal experts think it is complex. In fact, when
some additions were made to the law in 2005 (to harmonize with the INFOSOC EU
directive) the real experts on legal construction in Sweden, the Council on
Legislation (Lagrådet), concluded that it had been desirable to do a complete
editorial review of the Copyright Act instead of implementing the ``patchwork''
that the changes in the law now meant. The Council however stated that it
understood the hurry to implement the directive (Prop 2004/05:110, appendix 8, p
558). Sweden had already received a remark from the EG Court for a
delay\cite{darling-larsson05}.

This shows two things. It shows that the architects behind the legal
construction thought analogically, and it shows the strong interconnection that
the many national legislations have \textit{via} international treaties as well as the
European Union. The freedom to rethink copyright law is limited, or at least not
easily made, seen in the international perspective. Still, the regulating
process seems to lack a critical element in the legislative trend so far. The
policy makers seem to be beyond all doubt that the legislative tradition on
copyright is not only to be followed but the protection should also be expanded.
A strong and unif\hbox{}ied copyright (see for instance the INFOSOC
directive\footnote{Directive 2001/29/EC of the European Parliament and of the
Council of 22 May 2001 on the harmonisation of certain aspects of copyright and
related rights in the information society.} in the EU) and a strong enforcement
of this copyright (for instance the IPRED\footnote{DIRECTIVE 2004/48/EC OF THE
EUROPEAN PARLIAMENT AND OF THE COUNCIL OF 29 APRIL 2004 ON THE ENFORCEMENT OF
INTELLECTUAL PROPERTY RIGHTS.}) are in this perspective seen as the only
measures that will ensure innovation and creativity in society. There seems to
be no room for doubt here. If copyright protection is failing, the only answer
to be reached in this way of thinking is to enhance the enforcement, the control
of data streams and all online behaviour.

Another example from Sweden would be the so called Rehnfors investigation from
2007. The investigation regarded music and movies on the Internet and was
conducted by the governmentally appointed Cecilia Rehnfors (Ds 2007:29). The
investigation concluded that the legal services on the Internet often had an
unsatisfactory range of content to of\hbox{}fer, but also launched the idea that the
Internet operators should be given a responsibility to control that their
subscribers did not participate in copyright infringements. This proposal was of
course met with great opposition from the operators (Dagens Nyheter 3 September
2007). The increased operator responsibilities had been proposed by copyright
organizations, such as IFPI (Ds 2007:29, p 207). The development of technical
safety measures was seen as a key issue (Ds 2007:29, p 16).

The issue of f\hbox{}ile sharing and media content was up for a hearing in the Swedish
Parliament in April 2008. However, even the setting can be questioned from a
society in transition perspective: only legal alternatives were allowed to
present their case. No advocates of f\hbox{}ile sharing were invited to the hearing. It
was stated by a spokesperson for the hearing that:

\begin{quote}
``Several people can bring forward the arguments that for instance the Pirate
Bay has, such as the secretary of the Rehnfors investigation [see Ds 2007:29
above] Johan Axhamn. He knows most of the arguments''
(\url{http://ur1.ca/f6pd} 12 Mar 2008, author's translation).
\end{quote}

There was no one representing the f\hbox{}ile sharing community, even though the
purpose of the hearing was to speak about and to collect knowledge regarding how
the issue of f\hbox{}ile sharing and copyright issues should be handled. This is an
unbalanced approach that is problematic if one attempts to understand the
dilemmas of modern copyright, to say the least. It also illustrates how
conceptions legally formalised can blind real attempts to solve problems
connected to societal transition.


\section{A legal trend}
\label{s:darling_conceptions:legal_trend}

The development towards an increased protectionism in copyright, and the
proposals of how this protection should be undertaken, is part of a legislative
trend seeking to take control over the Internet and its communication. The
exceptionally stormy debate regarding increased governmental signals
intelligence (scanning internet traf\hbox{}f\hbox{}ic) is a national Swedish example (Ds
2005:30, prop. 2006/07:63) from the Summer of 2008. The new law was heavily
questioned, resulting in the forming of interest groups to stop it. A wave of
bloggers protested, and members of Parliament received lots of e-mails and
letters begging them to vote no. 

To describe the European legal trend I start at 2001 when the European Community
Directive on Copyright in the Information Society, \textit{the INFOSOC
Directive}, was passed which included narrow exemptions to the exclusive rights
of the rights holder as well as protection for ``technological measures'' (art
6). This meant that more actions were criminalized and that the copyright
regulations around Europe generally expanded and became stronger. In April 2004
the EU passed the Directive on Enforcement of Intellectual Property Rights, the
so called \textit{IPRED directive}, following what has been called ``a
heavy-handed inf\hbox{}luence of the American entertainment
industry''\cite{darling-kierkegaard05}. It had been set up as it is ``necessary
to ensure that the substantive law on intellectual property, which is nowadays
largely part of the \textit{acquis communautaire}, is applied ef\hbox{}fectively in the
Community. In this respect, the means of enforcing intellectual property rights
are of paramount importance for the success of the Internal Market.'' (Recital
3). The IPRED directive also states that all Member States are bound by the
Agreement on Trade Related Aspects of Intellectual Property (TRIPS Agreement),
which aligns the global regulatory connection on copyright between nations, the
EU as well as international treaties. After the bombings in Madrid in March 2004
the work started on what later became the so called \textit{Data retention
directive} in order to force Internet service providers and mobile operators to
store data in order to f\hbox{}ight ``serious crime''\footnote{DIRECTIVE 2006/24/EC OF
THE EUROPEAN PARLIAMENT AND OF THE COUNCIL of 15 March 2006 on the retention of
data generated or processed in connection with the provision of publicly
available electronic communications services or of public communications
networks and amending Directive 2002/58/EC.}. This was heavily criticized by
both the Article 29 Data Protection Working Party as well as the European Data
Protection Supervisor for lacking respect for fundamental human rights. The
question still remains in the Swedish implementation whether or not this can or
will be attached to copyright crimes and be used in connection to the IPRED
legislation, depending on how ``serious crimes'' will be def\hbox{}ined in national law
in relation to copyright crimes. Recently it is \textit{the European Telecoms
Reform Package} that has been heavily debated. It was presented to the European
Parliament in Strasbourg 13 November 2007 but voted upon 6 May 2009.

This cluster of legislation seeking to harmonize the national legislations of
the European Union all points to the obvious trend of adding control over the
f\hbox{}lows of the Internet. 


\section{Darling conceptions}
\label{s:darling_conceptions:darling_conceptions}

What are the darling conceptions tied to the legal order that creates the
tension in relation to the digital practice of today? There are a few
conceptions that are problematic in the transition to a digitalised society.
Legitimacy is a key question here. However, before we are even able to discuss
questions of legitimacy, we need to sort out a few things regarding the ideas
and the meaning of both law and the debate around copyright and legislation. 


\subsection{Theft}
\label{ss:darling_conceptions:darling_conceptions:theft}

When the idea of property rights are formed in an analogue reality and
transferred to a digital one, certain problems occur. An obvious problem, which
has shown the two sides of viewing the handling of media content in the debate,
is the sharing and copying of internet communication on one side and the
``theft'' on the other side. When seen from a traditional point of view, the
illegal f\hbox{}ile sharing of copyrighted content has been called theft. However, the
metaphor is problematic in the sense that a key element of stealing is that the
one stolen from loses the object, which is not the case in f\hbox{}ile sharing, since
it is copied. The Swedish Penal Code expresses this as ``A person who unlawfully
takes what belongs to another with intent to acquire it, shall, if the
appropriation involves loss, be sentenced for theft to imprisonment for at the
most two years'' (Penal Code Chapter 8, section 1, translation in Ds 1999:36).
To be specif\hbox{}ic, the problem of arguing that f\hbox{}ile sharing is theft lies in the
aspect of ``if the appropriation involves loss''. There is no loss when
something is copied, or the loss is radically dif\hbox{}ferent from losing, say for
instance your bike. The loss lies in that you are likely to lose someone as a
\textit{potential} buyer of your product. The ``theft'' argument is an example
of how an idea or conception tied to a traditional analogue context is
transferred to a newer, digital context.  Something is, however, lost in the
translation.


\subsection{Control over copies}
\label{ss:darling_conceptions:darling_conceptions:control}

The global construction of copyright has resulted in fairly homogeneous
copyright laws throughout the world. This has been done \textit{via} international
agreements (such as the Berne Convention and the TRIPS agreement), harmonisation
within the European Union (such as the INFOSOC directive of 2001), and copyright
cooperation amongst for instance the Nordic countries in Europe. A part of this
construction is the control of copies that the rights holders are granted. As
mentioned above, this can be seen as a logic and conception that was born and
functioned well in an analogue reality. Control was still possible, unlike
today's enormous task to control all online activities for all people,
regardless, if the behaviour has to do with illegal f\hbox{}ile sharing or not. In a
time where production, reproduction and distribution of each copy demanded an
investment that was not ignorable, the legal protection of the control over
copies makes sense. On the other hand, in a time where reproduction and
distribution costs are ignorable the legal protection of the control over copies
does not make the same self-evident sense. The development is probably that the
market is moving from being product based to being service based. You deliver
access to media rather than selling it in pieces. The control of copies, and the
idea that it is the copies that need to be controlled in order to have a
functioning market, is a darling conception of analogue times.


\subsection{Private/public relationship}
\label{ss:darling_conceptions:darling_conceptions:priv_pub_rels}

Generally, in Swedish legal tradition, the private sphere has been left
unregulated. The copyright legislation has followed this logic, such as section
12 in the Copyright Act above. With digitalisation and organisation in networks,
this private-public dichotomy has become a regulatory conception that has less
and less value in society. The private is not so private and the public is not
so public any more, in a sense. It is a regulatory method that functions less
and less well, at least in the f\hbox{}ield of copyright. The item-based reality of an
analogue production has now become digital and copy-based. Behaviour and
societal norms change in accordance with how the conditions for them change. As
the user generated web (2.0, as some call it) arises, many industries go from
being producer driven to consumer driven, and copyright is unavoidably af\hbox{}fected
by the introduction and distribution of new information technology. This leads
to questions about integrity and what type of society we want.


\subsection{Creativity of the few produces for the consumption of the many}
\label{ss:darling_conceptions:darling_conceptions:creativity}

Behind this conception lies the idea of an investment demanding production and
distribution, mentioned above. This conception stems from the idea that a few
key persons decide what the masses will need and like. Think about the few big
record companies or the old state owned TV channels in Sweden. It also applies
to the traditional logic of news reporting. What is regarded as news was a
centralised decision to make. ``Democratize democracy'' said the socio-legal
scholar Boaventura de Sousa Santos when speaking of the empowerment of the third
world at a conference in Milan in the Summer of 2008. Let us think about that
quote for a moment. It is about a model for decision-making. The Internet stands
for a widespread decision-making of content. It is the many who decide what is
interesting, not the few key persons. The quote could be used for saying: do not
construct systems around a few key persons of power when it comes to the
potential creativity of the masses. Democratize creativity in the system,
because creativity should not be decided over by the few. Let the many decide.
Democratize democracy.

The ``democratic culture'' is an expression used by John
Holden\cite{darling-holden08} to describe what in some areas of the industry is
called Web 2.0, meaning that content in online products is to a large extent
created and driven by the users. It is as a peer-to-peer product rather than an
ever so smart product originating from the wits of one genius. Compare a
traditional centrally produced encyclopaedia to the collectively produced
Wikipedia. Some solutions can not be thought out centrally, and nothing singular
can replace the social web. This is a benef\hbox{}iciary aspect of ``the f\hbox{}low'' of
media content that the digitalisation brings with it. 


\subsection{Ownership and property}
\label{ss:darling_conceptions:darling_conceptions:ownership}

The Swedish legal scholar, Dennis Töllborg, regards the introduction of the
Internet as a hegemonic revolution, similar to those earlier in history when our
view on society and ourselves were radically changed. Creation is still central
and imitation is always strong as a model for norm-building, but there is a
dif\hbox{}ference, and that is the value-base. The idea is still free, but when ideas
materialize in a digital way and leave their mechanical existence, the material
relation to physical control over what you consider as your property, is
missing. When the idea loses its reference to the physical world, the value the
usage brings once again becomes dominating for what we regard as legitimate and
fair. The exchange value, coupled with exclusive intellectual property rights
for the owner, cannot and should not be protected, since the idea behind the
Internet is, according to Töllborg, at stake in the example of f\hbox{}ile-sharing. In
this situation the former legal understanding of property rights will be
invalid. Töllborg argues that you cannot claim ownership to something which is
not possible to transform into something material, to a physical object. This
will be the understanding of ownership, according to Töllborg, in the new
hegemonic era\cite{darling-tollborg08}. The fact that there are a lot of people
arguing for old solutions, does not change Töllborg's prediction. It is only a
sign of the inevitable f\hbox{}ight between dif\hbox{}ferent darling conceptions of your time,
taking place when a society is in a phase of transition, and the idea of
property in a digital context is part of the battle.

So, to f\hbox{}inish the f\hbox{}ive examples of problematic darling conceptions in relation
to digitalisation the three man combo is suddenly heard from the corner, singing
something about a battle between the old and the new:

\begin{quote}
\textit{Can you feel it too?}\newline
\textit{The old world measuring the new}\newline
\textit{Can you feel it too?}\newline
\textit{The old world claiming the truth}\newline
\newline
\textit{I know you've heard it too}\newline
\textit{That the questions that we ask ourselves}\newline
\textit{in the passed way of thinking}\newline
\textit{won't solve the problems of the new}
\end{quote}


\subsection{Conclusions: the battle of conceptions}
\label{ss:darling_conceptions:darling_conceptions:conclusions}

There seems to be a battle not only over how to organize society but also about
conceptions. The analogically based conceptions regarding the importance of the
control over the reproduction of copies battles with the digitally based
conceptions regarding f\hbox{}low of media where copies in themselves are not of the
same importance. This leads to an interesting counter factual question that we
can use to activate our minds. How would copyright laws have been designed had
media distribution been digital from the beginning? That is, if we had skipped
the step of a demanding distribution and reproduction \textit{via} plastic and
physical artefacts, how would we have designed the legal setting that would
ensure creativity in society? 

This question aims at unlocking conceptions that are embedded in copyright
legislation that may not be in accordance with the digital practice of today.
There are parts of copyright legislation of today that probably would have
survived and parts that would have looked dif\hbox{}ferent. If we at the same time look
at the creators (and creativity stimulation) on one side and copyright as a
market security for copyright holders on the other, we could nuance the
discussion of copyright a bit. The much discussed protection of rights for
seventy years after the creators' death is aiming at the copyright holders
rather than at the creators and creativity stimulation. 

Let me also address the scholars and the law-makers: legal science must
understand how society changes. Otherwise, there is a high risk that the legal
system could turn into an institution that uses its powers to support the
parties that act and are coming from the traditional order in society, meaning
an institution that distorts the societal development to f\hbox{}it some interests
before others. And this is the consequence of that the legal regulations has
f\hbox{}irst appeared in the same time as the old structures and parties
emerged(mixed-up syntax). These ageing parties will receive support, not because
they represent something more true or more just, but simply because they are the
next to kin of the emperor, so to speak. The legal order then becomes a tool for
power in a struggle between the old and the new, rather than a democratically
legitimate interpreter of what is right and just.

In using the above mentioned work of Lakof\hbox{}f and Johnson on metaphors, applied on
the grand context of this article, conceptions are unavoidably attached to
discourses, and although they may have a very specif\hbox{}ic meaning in the discourse
their meanings can change, and their uses can be altered. This implies that
conceptions can be tied to an arranging order, an administrative pattern, in
itself stemming from, for instance, analogue conditions of distributing media.
These conceptions are likely to stand in the way when the administrative system
is in need of a revision due to a change in the conditions. In short, the
digitalization changes the conditions for distribution of media, and the
conceptions tied to copyright are standing in the way of the needed revision of
copyright legislation.

Let me get back to the initial quote from Lakof\hbox{}f and Johnson (``People in power
get to impose their metaphors''\cite{darling-lakoff_johnson80}), and state that
even though the research on metaphors of Lakof\hbox{}f and Johnson had nothing to do
with law or regulatory language, the quote can be used in this context. Law
relies on metaphors and conceptions that have been discussed above, when it
comes to copyright and the various legal constructions that for instance have
been implemented within the European Union in order to enforce copyright more
easily, these conceptions rely on a metaphorical use of the language that
incorporates ideas of how the world is constructed as well as what the legal
regulations should say. Those who control the laws and the legislative process
can also, to a large extent, control what conceptions and metaphors should
remain therein.  This is why the battle of the Internet to a large extent has to
do with controlling the conceptions that construct how we regulate the internet,
and controlling those conceptions having to do with power. 

When the idea of property rights are formed in an analogue reality and
transferred to a digital, certain problems occur. An obvious problem, which has
shown the two sides of viewing the handling of media content in the debate, is
the sharing ideal of internet communication on one side and the ``theft'' on the
other side. It is a battle of ideas, but also of conceptions of reality. 

There is a risk that copyright goes from being a stimulator of creativity to a
conservator of rights holders. It sort of implies that the most important media
content is already created. ``Now let's protect those who did it (or rather,
hold the rights for those who did it)'', which is a sad implication. It is
conservative and will more likely stif\hbox{}le innovation, which is the direct
opposite to the rhetoric that surrounds the law and its enforcement. This leads
to an aim to control and to over-regulate protection of copyrighted content. It
misses the point that \textit{all} creativity is born out of a context, out of a
culture, and that too much regulated protection will be \textit{bad} for
creativity\footnote{Even legal scholars have referred to this as \textit{lex
continui}. See \cite{darling-karnell70}. See also the preparatory works for the
Swedish Copyright Act, SOU 1956:25 s 66 f.}.

The copyright regulation should not \textit{primarily} be aimed at helping
publishing houses, record companies or similar middle men to survive. They do
not have a value in themselves for the copyright legislation to meet. Culture is
however inf\hbox{}luenced by how the conditions are formulated. As technology has
developed that has inf\hbox{}luenced storage of information, expanded duplication or
distribution possibilities so have dif\hbox{}ferent opinions been heard. Some claim
that the incentives to create disappear when the originators no longer have full
control over the copies. Internet and f\hbox{}ile sharing however af\hbox{}fects dif\hbox{}ferent
types of creativity dif\hbox{}ferently. The f\hbox{}ilm industry may stand before a larger
transition or challenge than the music industry, due to its larger and more
expensive projects. However, in the changes of the premises for storage and
distribution, and communication, one can establish that some types of creativity
will likely see harsher times, and other types of creativity will def\hbox{}initely
thrive. It is a part of the change. Let us not forget that totally new forms
also will emerge, many without retrieving any revenues from the existing
copyright system whatsoever.

Is copyright strong or weak in these days of digitalization? And what will
happen in the future? Lawrence Lessig, the Stanford Law professor and Creative
Commons Licence promoter, paints a bleak picture of when it comes to the balance
between content that should be accessible and that which should be protected. He
sees a development towards an increase in protecting copyrighted material: 

\begin{quote}
``We are not entering a time when copyright is more threatened than it is in
real space. We are instead entering a time when copyright is more ef\hbox{}fectively
protected than at any time since Gutenberg. The power to regulate access to and
use of copyrighted material is about to be perfected. \ldots in such an age, the
real question for law is not, how can law aid in that protection? But rather, is
the protection too great? \ldots. But the lesson in the future will center not
on copy-right but on copy-duty – the duty of owners of protected property to
make that property accessible.''\cite{darling-lessig06}. 
\end{quote}

An important question that lurks behind these disputes of ideals is what kind of
protection can exist without an absurd amount of control over human actions?
Communication technology is not just a bad habit of the young generation, it is
a fundamental part of how this generation leads the life. In a study conducted
in February 2009 by a Swedish research project called Cybernorms, with more than
1000 persons between 15 and 25 years old, the results clearly indicated that
there existed no social norms that hinder illegal f\hbox{}ile sharing. And the
surrounding persons of these youngsters imposed no moral or normative
obstruction for the respondents' f\hbox{}ile sharing of copyrighted
content\footnote{I am part of this research group, tied to Lund University in
Sweden. See \url{http://ur1.ca/f6pe} for a presentation in Swedish. See also the
debate article from the research group published in Dagens Nyheter 23 February
2009 \url{http://ur1.ca/f6pg}}. In line with this the study also found that
more than 60 per cent of the respondents rather paid for services that made them
anonymous online and kept on illegally f\hbox{}ile sharing than paying for the
content\footnote{\url{http://ur1.ca/f6ph} visited 14 June 2009.}. Many were
however willing to pay for content, but not \textit{via} the traditional model
of paying for each piece. It was the f\hbox{}low that was of importance, for
which the respondents were willing to pay, and in which the copyrighted content
was included among other things. 

When speaking of law and social norms one is often inclined to speak about the
legitimacy of the legal regulations. The biggest threat to a law is losing its
legitimacy. When a law is less right, it is no longer the trusted interpreter of
what actions are right and wrong in terms of the social norms. One could claim
that no law is stronger than the underlying social norms (which Håkan
Hydén\cite{darling-hyden02} does), and that the social norms are functions of
the conditions for them. The conditions that are embedded as conceptions in
copyright law have fundamentally, or even paradigmatically changed. The
preconditions for the social norms have drastically changed as society has
become digitalised. The social norms among many and the law do not match.

Law is strongly interconnected with society. Do not mistake behaviour in a
society simply for a function of its laws, and that it therefore is easy to
change society. This is where a problem lies, connected to legitimacy of legal
regulations. The understanding of this article is that conceptions can be tied
to a specif\hbox{}ic world order, to a way in which a society is organized. This leads
to what the title is asserting: societies change and the conceptions that have
been more or less deeply founded in them can face problems when translated into
the new context. Clashes are inevitable. The rules and norms will collide and
confuse. The example of f\hbox{}ile sharing, the Internet and the copyright debate has
here been used to show the clashes of such a societal transition and the
conceptions within. 


\subsubsection{Say it with a song}
\label{sss:darling_conceptions:darling_conceptions:conclusions:song}

The song \textit{The darling conceptions of your time} is a creative expression.
It is also an experiment, an attempt to understand and to test a non-traditional
model for content distribution and the functionality of the copyright regulation
\textit{via} the Creative Commons Licence. I am still the creator, but I make a
contract with anyone who wants to do something with the song. It is a way to
meet the new conditions for distribution and creativity. I am handing over the
song to the commons to use, to re-mix, to share, or not. Democracy decides. 

So, the changes and the embedded problems have to do with how we view society,
what interpretations we make of the conditions it brings. It has never been as
searchable and interconnected as it is today, bringing along a type of
vulnerability and questions about how this interconnectedness is used. 

And from the corner of the bar, when most guests have left, the three man combo
still plays. One pictures the last drunken man at the very end of the bar,
Galileo Galilei, who unsteadily rises to silence the imagined mumbling crowd
around him with a movement of his hand. He looks a bit sadly towards them, and
then starts to sing with a broken voice: 

\begin{quote}
\textit{It's not the eyes that fool you}\newline
\textit{It's not the ears that can't hear}\newline
\textit{It's the darling conceptions of your time}\newline
\textit{that makes you feel this way}\newline
\textit{that makes you feel this way}
\end{quote}

