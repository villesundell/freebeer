% Copyright 2009 FSCONS, Superflex and the individual authors.
% This entire book and all its source files is licenced under Creative Commons Attribution-ShareAlike 2.5
\begin{savequote}
    \qauthor{\LARGE{Rasmus F\hbox{}leischer}}
\end{savequote}
\chapter[Kopimi]{Kopimi\\ \Large{Proceedings from autumn 2008}}
\label{c:kopimi}

The decade between 1995 and 2005 roughly marks out the breakthrough of
f\hbox{}irst the \textit{www} (world wide web) and then \textit{p2p}
(peer-to-peer f\hbox{}ile-sharing).  Those were the times when it was still
possible to imagine a shift from an old and material to a new and virtual world,
most distinctive in the Californian ideology of John Perry Barlow's
\textit{Declaration of Independence for the Cyberspace} (1996). It still made
some sense to use bandwidth as a symbol for community and freedom, proclaiming
that ``Welfare starts at 100 mbit'', as we did with Piratbyrån on May Day 2005,
just before releasing the anthology \textit{Copy Me} – which in retrospect reads
as a time document over a brief but interesting era, published exactly at that
end point.

Since then, we have moved ahead. After reaching the point when one realizes that
\textit{the f\hbox{}iles have been downloaded}, the question is no longer one of
\textit{access} but of \textit{action}. What to do with all these f\hbox{}iles?
My hypothesis is that, on a kind of collective level, this point was somehow
reached in 2005, at the time when f\hbox{}ile-sharing also stabilized around the
Bittorrent protocol. Of course the exchange of f\hbox{}iles will continue to
increase quantitatively, but what really counts is not how fast a connection one
has to the network, but how this abundance of data is actually used in space and
time.

Some ideas which had a liberating potential in the last decade (1995-2005) –
especially the idea of the digital as a ``second life'', detached from the old
powers – may even have become reactionary or paralysing in the decade in which
we now live (2005-2015).

On the one hand, copyright law continues to expand in the direction of
neo-corporatism and of a permanent state of exception, which is something one
has to deal with regardless of one's involvement in actual copyright
infringements. On the other hand, we must deal with ethical and aesthetic
questions which demand that we \textit{ignore} copyright, or at least regard it
as a thing of the past.

Now we can also realize that the exclusive attention that was given to bandwidth
must be supplemented with other aspects of the digital, like storage. The simple
fact is that storage capacity is increasing exponentially and much faster than
internet bandwidth. Some simple quantitative extrapolation of this fact may help
us formulate new, qualitative questions for the time we live in. I will do this
from the perspective of music, as it is the most ambivalent of art forms,
in-between product and process, poiesis and praxis.

We are approaching a point, predicted to occur within 10-15 years, when any
cheap, pocket-size media player will have have space to store practically
\textit{all recorded music that has ever been released}. This gargantuan pocket
archive will be created, and it will be copied from friend to friend. There will
be absolutely no way for a rights holder to prevent that from happening.

Such a scenario is not good or bad in itself. But it opens the question: Will
all music ever recorded have \textit{any value at all} for us? How could the
simple addition of one more song on top of such an archive produce any feeling
whatsoever in us? When you sit there with all the music ever recorded – what do
you do? The idea of just pressing ``shuf\hbox{}f\hbox{}le'', to let musical
history be played randomly, seems to open up an almost existential horror. The
opposite idea of playing it all in alphabetic order is just plain stupid and
would exceed human lifetimes.

It is actually doubtful whether any of these two choices would produce something
that could seriously be called ``music''. Because music, as any improvising
musician knows, can only be something in between total predictability and total
randomness.

Imagining this archive of ``all music ever'' is not just speculation in some
hypothetic future, because we already have access to much more media than we can
incorporate in our lives. Through these common small white earphones, we are
already – more or less – able to listen to any piece of recorded music,
whenever, wherever, while doing whatever. That means that any piece of recorded
music – considered in isolation – is deprived of all its remaining emotional
value.

Both 19th century western classical music and 20th century pop music were
cultures resting on the belief that the sound of music could in itself reveal
meaning to the listening individual. Still today, that logic is used
conventionally to explain the dif\hbox{}ference between good and bad music. It
is preserved f\hbox{}irst of all, of course, by the record industry and by the
mass media, but it is also very present in various on-line music communities,
including f\hbox{}ile-sharing sites. We must now discard that convention, and
stop pretending that there can be any inherent value in a digital f\hbox{}ile.
F\hbox{}irst the complete denial of this value allows us to explore and
af\hbox{}f\hbox{}irm new values. This process is well under way, but we may not
yet have all the concepts needed to complete it.

When we can listen to any piece of music, whenever, wherever, while doing
whatever – then we begin desiring musical experiences which can \textit{not} be
accessed anywhere and at any time. We begin seeking out contexts which are
specif\hbox{}ic for a time or a place, an occasion or a friendship. Some of
these contexts are by convention known as ``live'' music. Others are personal,
like the association of a certain play-list to bus rides through foggy November
mornings. In between the big and the small is a space for multiplication of
informal habits.

One way to f\hbox{}ind directions for exploration is to simply negate everything
that the iPod stands for. Using a strictly materialist approach, that negation
drives us downwards, towards the sub-bass spectrum. Bass-centred music can
\textit{not} be experienced anywhere, because of the very physical need for very
large speakers to produce really deep frequencies. It can indeed be recorded,
digitalized and transported in the pocket, but it cannot be listened to in
headphones during the transport. All you can listen to is a simulation. Such
simulations are vital for creating a cultural continuity – but their musical
value is never inherent in the hearing of any track, but is derived from the
bodily memories of bass and the anticipations of being physically present at
future occasions.

In fact, sub-bass is almost never an individual experience. Low frequencies have
less respect for physical architecture (ask your neighbours), if played at the
volumes that bass-centred music demands. They have, however, more respect for
human ears than the higher-frequency sounds of a traditional rock concert.

I am talking about dub-step, which is a phenomenon rather than a musical genre.
What keeps it together? F\hbox{}irst, a few clubs with extremely large bass
woofers, primarily in South London, and in many cases using squatted space.
Second, a certain combination of internet protocols: internet radio (shout-cast
protocol) with DJs playing in their own bedrooms while being in real-time
interaction with the community in chat rooms (irc), with sessions being
afterwards freely available in MP3 format on the web (http). Third, there are
indeed record labels, usually integrated with the clubs, releasing most tunes
only on vinyl.  In short, the material constellation of dub-step is one possible
way to create meaning out of abundance, while simultaneously maintaining an
informal economy which does not really depend on copyright law, by
systematically integrating the very digital with the very analogue.

It is not a coincidence that dub-step, as an extremely bass-centred musical
phenomenon, emerged exactly in 2005. That was the year when the f\hbox{}iles had
been downloaded, when the digital abundance had again to become anchored in time
and space. Dub step is music for the current transitory decade of 2005-2015.

But of course, gigantic bass woofers are not the solution for everything. The
morning after, we are back in front of the screen, with access to \textit{all
music ever recorded}, thinking about where to start. We will not just press
``shuf\hbox{}f\hbox{}le'', and not just play the tracks alphabetically. And as
anyone knows who has been in a similar situation, it is not simply to reconsider
``what one likes''. For the contemporary music fan in the climate of abundance,
there is not even such a thing as a unitary individual taste, independent of a
particular context in time and space.

Rather than individuals, we are ``dividuals''. That is also why all these
automatic recommendation systems are still very primitive, def\hbox{}ining
``taste'' just in terms of personalized listening statistics. Amazing
developments on this f\hbox{}ield will come, for sure, as soon as we accept
being geographically tracked, allowing certain parts of the city to be
associated with certain musical tracks (which in its turn will performativize
individual listening, knowing that it contributes to the databases containing
these associations).

Automatic recommendation systems are a necessary help, and will continue to
change our relations to music in many ways, but they can not solve the basic
problem of having too much choice. You can always switch to an alternative
software algorithm, just as the forward button on your iPod is keeping you aware
that you can always shuf\hbox{}f\hbox{}le on to the next song (which is a far
more important dif\hbox{}ference between iPods and cassette tapes than any
``sound quality'').

Pure freedom could never be musical, just as the absence of any freedom
couldn't. Musical experience happens in between, when you have a choice within
certain limits, to work against something – and this goes for all musical
activities, ``passive listening'' as well as ``active playing''. A melody or a
rhythm is a limit, just like a musical instrument, the acoustics of a room, or
the human body when one sings or dances. Most importantly, the very presence of
other people with other expectations is in itself a limit.

In order to f\hbox{}ind out what we want to enjoy, to create meaning out of
abundance, we surely need some software, but most of all we need community. Only
reference to collective contexts can save us from the terror of the
shuf\hbox{}f\hbox{}le button, and from the forced performativity of automated
recommendation systems.

The digital poses questions whose answers can not remain within the digital, but
demands the formation of provisional communities, where people can engage in a
common selection, indexing, combination and actualization, connecting the
digital to time and space. Size does matter a lot. Some recent experiments have
been demonstrating how groups of 17\footnote{Bill Drummond's choral project
\textit{The 17} (\url{http://ur1.ca/f6o5}), recently documented in a book with
the same title, and the related performance No Music Day
(\url{http://ur1.ca/f6o6}), generally resonates a lot with some standpoints
expressed in this article.} or 23\footnote{In 2008, Piratbyrån acquired an old
city bus, named it S23M and drove it in the summer with 23 passengers and 100
mix-tapes, from Stockholm to the Manifesta Biennale in Südtirol, as an
experiment in enacting a ``digital'' community to a very ``analogue'' context.
This experiment has greatly inf\hbox{}luenced this whole article, and led to
innumerable follow-up actions, including the autumnal journey S23X taking the
bus eastwards to Ljubljana and Belgrade.} or 47\footnote{When I am writing this
sentence, I am listening to the dub-step net radio SubFM
(\url{http://ur1.ca/f6o7}), in look up how many listeners we are at the very
moment, getting the number 47. That's low, because right now they only reprise a
session from an earlier night. Listener numbers go up a lot in the evenings when
it is possible to interact directly with the radio DJ.} participants (for some
weird reason this tends towards prime numbers) can further certain dynamics
which are not possible either in the biggest stadium-size or the smallest
kitchen-size event. Many times, these communities seem to thrive best in the
grey zone in between what is usually regarded as the public sphere and the
private sphere, often also in between the purely commercial and the purely
non-commercial.

And here we get back to copyright! Because grey zones are generally not
recognized by copyright law, copyright licences or copyright collecting
societies. Copyright is dichotomizing. It always recognizes some kind of private
sphere. Within the family you may copy without restrictions. You may even invite
friends to your home to watch a movie, or to hear you sing a song, without
asking for special permission or paying extra to any rights holder.

Copyright law does not step in to the picture until the copying or the
performing becomes ``public'', at which point a completely dif\hbox{}ferent set
of rules starts to apply. Where to draw this line between private and public is,
however, a matter of uncertainty and modulation.

Think about a group of people getting together every week to watch and discuss a
selected movie and maybe also listen to some music. Week after week the group
slowly grows, and it has to move to larger spaces. Sooner or later this group –
or any informal activity emerging in the spectrum between private and public –
will be pressured by copyright law to choose one of two paths: Either it has to
keep small-scale and hidden from the public. Or it has to turn fully commercial,
to put up advertisements or start selling expensive cocktails, so that licences
to the industry can be paid.

Copyright is not just a repressive power, but is also productive. It shapes the
contexts in which people can get together to create meaning out of abundance, by
attempting to erase exactly the grey zones which we need most. Copyright
materializes in the city, as well as in the architecture of computer networks.

In the latter, however, the def\hbox{}inite walls seem to be lacking and must be
simulated by software. Because computers operate by copying information all the
time, and don't seem to care about physical distance, copyright law has quite
serious problems with drawing a credible line between private use and public
distribution through computer networks. Distinctions which where formerly within
physical infrastructure, like the one between record distribution and radio
broadcasting, actually collapses when on the internet the only dif\hbox{}ference
between ``downloading'' and ``streaming'' is how the receiver's own software is
conf\hbox{}igured. This is the main reason why today's conf\hbox{}licts over to
copyright law are essentially about access to \textit{tools} (indexing services
like The Pirate Bay, stream ripping software, or codes for circumventing dvd
encryption).  The conf\hbox{}licts are not any more, like in the 20th century,
about access to copyrighted \textit{works}.

We must stop asking how artworks are best distributed within networks. Copyright
conf\hbox{}licts concern the very meaning of terms like ``artworks'' and
``networks''.  In the rhetoric about so-called Creative Industries, especially
at a European policy level, ``creativity'' is def\hbox{}ined as the production
of ever more "content", irrespective of its context. Pure information,
inf\hbox{}initely reproducible even if tightly controlled.

This discourse subscribes to an idea of the digital as a substitute for
place-specif\hbox{}ic activities – an idea which somehow resembles the utopian
net discourse of the previous decade.

Now we start realizing that one of the most fascinating properties of digital
communications is that they can awaken a strong desire for exactly those things
which they cannot communicate. The digital is not a separate world, as the
dominant ideology of 1995-2005 used to preach. It is always a complement to
something else. But for what we never know in advance. We must invent it and
that is an adventure that must take some time. All we know is that there can not
be one single solution for everything.

The anxious search for ``the solution'' might be necessary to trigger the
process of moving on. But in every such process comes a certain point when the
anxiety must be unconditionally left behind.

Now our main task can't any more be to give more answers, to create more
``content'', or to invent fresh business models. Much more relevant than drawing
up blueprints for how stuf\hbox{}f should work in the future, is to here and now
try out new ways to put all existing content into context. The general problem
is abundance, not scarcity. What counts in the end is action, not access.

With Piratbyrån, we are co-developing a method known as kopimi. Kopimi is about
af\hbox{}f\hbox{}irming the will to copy and to be copied, without reservation,
and to acknowledge the active and selective moment in all copying. It is, at the
same time, about exploring that which can not be copied, that which slips away –
and to enjoy it as it slips away. It is about valuing the very process of
copying, while recognizing that no copy will be identical. Mutations always
happen when as a copy it is connected to another place and another time.

Kopimi is an imperative – copy me! – not a theory. Thus it has no real origin,
but is said to have emerged from a dance. When it is def\hbox{}ined, it is
always by means of selecting and copying def\hbox{}initions of other phenomena,
letting these def\hbox{}initions mutate. That kind of process is probably the
only ``alternative'' to copyright that kopimi can propose – an alternative not
for individual ``artists'', but for artistic practise at large.

Of course, answers will be formulated, ``content'' will be created, and business
models will be invented. Don't worry. From the perspective of kopimi, however,
this comes merely as a side-ef\hbox{}fect to something much more crucial: the
quest for ways to integrate the inf\hbox{}inite abundance of information into
our f\hbox{}inite lives.
