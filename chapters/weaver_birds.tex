% Copyright 2009 FSCONS, Superflex and the individual authors.
% This entire book and all its source files is licenced under Creative Commons Attribution-ShareAlike 2.5
\begin{savequote}
    \qauthor{\LARGE{Denis Jaromil Rojo}}
\end{savequote}
\chapter{The Weaver Birds}
\label{c:weaver_birds}

\section{Hackers spinning the Dharma wheel}
\label{s:weaver_birds:dharma_wheel}

You are welcome to join the new wheel spin of our history.

This document is an open (in f\hbox{}ieri) \textbf{Magna Carta Libertatum}: A
programmatic, visionary and inclusive document to reclaim the space for the GNU
generations, proposing a plan to be shared that is already being shared by many.

The dyne.org hackers network has become eight years old this year. Of course,
this text does not just talk about "us". Being an open network, we include
multiple contexts around the world with which we share mutual help; as with our
free software development activity and the sharing of on-line and on-site
spaces. This document talks about our dreams, which are slowly but steadily
becoming reality.

For all this we are inf\hbox{}initely grateful to the GNU Project\footnote{See
\url{http://ur1.ca/f6o9}}, that let us discover how to get hold of knowledge,
take control of the architecture we live in and start building a new planet :)


\section{Dharma youth}
\label{s:weaver_birds:dharma_youth}

\begin{quote}
\textit{The only people for me are the mad ones, the ones who are mad to live,
mad to talk, mad to be saved, desirous of everything at the same time, the ones
who never yawn or say a commonplace thing, but burn, burn, burn, like fabulous
yellow roman candles exploding like spiders across the stars.} (Jack Kerouac,
Dharma Bums)
\end{quote}

F\hbox{}irst let us declare who we are: After eight years, we are able to trace
a common denominator among the people active in our network, interconnected by a
nomadic approach to development and life.

We are young dreamers. We often like to stir limitations and invent
dif\hbox{}ferent models by which to learn, communicate, share and live
dif\hbox{}ferently to those proposed by the societies where we are caged. We
have in common that we survived out of the commonplaces, we cultivated our
thoughts and sharing methods, knowledge and tools, keeping them out of any box.

This is the time in our history in which we will speak with young voices, as we
are taking some crucial steps on which we will base our architectures, hopefully
mixing the inner with the outer, the Ying with the Yang.

Some of us are nomads, some settle in dif\hbox{}ferent places from time to time,
some live in the same marginal neighbourhoods of the world where they were born,
some are working for multinational IT companies, some are riding bicycles all
around the world, some are lecturing in schools, some are living in the
wilderness, some are exhibiting in art galleries and some are squatting houses.
And yes, you are probably one of these, or you have been in contact with us at
least once.

What we are proposing here is a new model, as we acquire a practical vision to
develop it in harmony with our dif\hbox{}ferent environments.

Please continue reading if you like to discover why and how.


\section{Freedom of Creativity}
\label{s:weaver_birds:freedom_creativity}

\begin{quote}
\textit{The growth of the network rendered the non-propertarian alternative even
more practical. What scholarly and popular writing alike denominate as a thing
("the Internet") is actually the name of a social condition: the fact that
everyone in the network society is connected directly, without intermediation,
to everyone else. The global interconnection of networks eliminated the
bottleneck that had required a centralized software manufacturer to rationalize
and distribute the outcome of individual innovation in the era of the
mainframe.} (Eben Moglen)
\end{quote}

Free (as in "libre") software is, when referring to the original principles
endorsed by the Free Software Foundation\footnote{see \url{http://ur1.ca/f6ob}}
(FSF), a new model for distribution, development and marketing of immaterial
goods. While recommending you to look at the philosophy pages published by the
FSF, we will highlight some implications which are most important for us, by
motivating our activities and enabling them.

Free software implies a distribution model based on collaboration instead of
competition, f\hbox{}itting in the f\hbox{}ields of academic research where
sharing of knowledge is fundamental and where the joint ef\hbox{}forts of
dif\hbox{}ferent developers can be better sustained when distributed across
various nodes. In this regard we quote John Nash (Nobel in 1994) saying that
``the best result will come from everybody in the group doing what is best for
himself, and the group''.

Imagine then that all creations reproduced in this way can also be sold freely
by anyone in each context. This opens up a horizon of new business models that
are local, thus avoiding globalised exploitation, but share a global pool of
knowledge useful to everyone.

Furthermore, in the f\hbox{}ields of education we believe that independence from
commercial inf\hbox{}luences is crucial in order to empower students with a
knowledge that they really own.

We want to liberate our minds and the minds of the ones who will come.

\begin{quote}
Here is where the dif\hbox{}ference between free software and open source starts
to matter. Open source focuses on new models for development. Free software is
not interested in how the program is developed. We are interested in the ethics
of how the program is distributed. (Richard M. Stallman)
\end{quote}


\section{No nationhood}
\label{s:weaver_birds:nationhood}

\begin{quote}
\textit{Per far che i secoli tacciano di quel Trattato\footnote{Trattato di
Campoformio} che traf\hbox{}f\hbox{}icò la mia patria, insospettì le nazioni e
scemò dignità al tuo nome.} (A Bonaparte liberatore, Ugo Foscolo, 1778-1827)
\end{quote}

\begin{quote}
\textit{One Planet, One Nation} (Public Enemy)
\end{quote}

Our homelands are displaced, are sometimes very dif\hbox{}ferent, sometimes
dif\hbox{}f\hbox{}icult to be put in contact with due to the boundaries given by
nations. In fact we think that nation states should come to an end, for the
borders they impose are not matching our aspirations and current abilities to
relate to each other.

During the few years of our lives we have been taught to interact and describe
ourselves within national schemes, but the only real boundaries are the
dif\hbox{}ferences between our languages, which boundaries we have learned to
cross.

From our national histories we mostly inherited fears and hunger. But with this
network we have learned how to bury them, as they do not belong to us any more.
What is left is a just a problem that can be solved: we will stop representing
us as part of dif\hbox{}ferent nations. Even if we could, we do not intend to
build our own nation, nor propose a new social contract, but rather to cross all
of these borders as a unique networked planet, to start a new cartography.

We have a planet! And it is young enough to heal the scars left by the last
centuries of war, imperialism, colonisation and prevarication that left most
people cultivating dif\hbox{}ferences and fake identities, represented by
f\hbox{}lags and nationalist propaganda.

We aren't claiming to open the borders for the speculation of multinationals,
since we are well aware this can be a rhetoric used by neo-liberist interests to
tramp over the autonomy of developing countries. The contextual
integrity\footnote{see Nissenbaum, H, (2007) Contextual Integrity -
\url{http://ur1.ca/f6od}} of dif\hbox{}ferent social ecosystems needs to be
respected, but as of today, the national borders do not succeed in preserving
it.

With some exceptions, most of the national programmes and cultural funds we
agreed to work with were pretending each of us would dress in a f\hbox{}lag, as
we were recruited in a decadent game of national pride and competition, with an
agenda of cultural, economical and physical domination. Tracing all our
movements, they assimilated them to leviathans that were playing the last
violent moves of a chess game in which we were just pawns.

This does not make sense to our generation any more. We refuse to identify with
the governments holding our passports, especially since these governments now
work for the mega-corporations that maintain their power over us. We look
forward to relating to each other on the bases of dialogue and exchange,
approaches and architectures that can be imagined globally and developed locally
in an open way like the channels that let us speak to you right now.

Therefore we declare \textbf{the end of nations}, as our generation is connected
by a far more complicated intersection of wills, destinies and, most
importantly, problems to be solved.


\section{Networked cities}
\label{s:weaver_birds:networked_cities}

\begin{quote}
\textit{Creo que con el tiempo mereceremos no tener gobiernos.} (Jorge Luis
Borges, 1899-1986)
\end{quote}

Naturally, our cartography draws connections among nodes, hubs of intelligence
that are closer in the cyber space than in the physical. In the last century we
have learned how we can share music, lyrics, stories and images, and, for a few
decades, we have been able to copy them without marginal costs across the whole
world.

This lets us relate to each other with an outreach that is amplif\hbox{}ied by
the density of our living environments: the urban spaces that somehow
of\hbox{}fer enough gaps for our agency. Those who pretend to govern our living
are now busy in controlling those voids, while every tree in a public square
represents an obstacle for their cameras, omnipresent eyes patronising our
evolution.

We found shelter in the ancestral practices of trance\footnote{Lapassade, G.
(1976) Essai sur la transe, Éditions universitaires}, opening the doors of our
perception to the unknown, resonating our own bones, enhancing the agility of
our tongues to follow the hip-hop f\hbox{}low of radical thoughts, skating over
the universe in which we are constrained, painting fantasy over the imposed
walls of our cities, jumping higher to join the loose ends of our parkas.

These practices are now common in all of our cities\footnote{De Jong, A,
Schuilenburg, M. (2006) Mediapolis. Popular culture and the city, Rotterdam:
010-Publishers}, seeded by our own need to evolve, to inf\hbox{}luence a
governance that doesn't listen to us. Some kids turn into a dark army of
vengeance, some lose the faith in future, some fall in the virtual loopholes
of\hbox{}fered by the magnetic startups of the dot.com boom. We need to
of\hbox{}fer ourselves an alternative to this hopeless conf\hbox{}lict and the
f\hbox{}irst step is to build a narrative that respects all choices, that does
not neglect suf\hbox{}ferance.

All this creativity and despair is shared among our cities, stuf\hbox{}fed by
unnecessary needs and mirages of success of the "creative industries", while we
already elaborate a concentric vision that is linked to the density of our lives
and the cultural f\hbox{}low of our errant knowledge.

Therefore we declare the birth of a \textbf{planet of networked
cities}\footnote{Batten, D.F. (1995), Network Cities: Creative Urban
Agglomerations for the 21st Century, SAGE}, spiral architectures of living
swirling above our heads and across our f\hbox{}ingers, as they evolve in a
common practice of displacement and re-conjunction, joining the loose ends of
our future.

Our plan is simple and our project is already in motion. In fact, if you look
around yourself, you will already f\hbox{}ind us close. While the current
economical and political systems face the dif\hbox{}f\hbox{}iculty of hiding
their own incoherence, we are able to implement their principles better and,
most importantly, we are elaborating new ones.

We are reclaiming the infrastructures, the liberty to adapt them to our needs,
our right to property without strings attached, the freedom to confront ideas
without any manipulative mediation, peer to peer, face to face, city to city,
human to human.

The possibility of growing local communities and economies, eliminating
globalised monopolies, and living up from our own creations, is there. We are
f\hbox{}illing the empty spaces left in our own cities, we are setting our own
desires and are collectively able to satisfy them.

Furthermore, some of us are seeking contacts with the lower strata of societies,
to share a growing autonomy: as much as they are excluded by the society they
serve, that much they are closer to freedom, while it is clear that autonomy is
the solution to present crisis. These marginal communities were the villagers
who, mostly because of rural poverty, could no longer survive on agriculture, as
well the migrants and refugees who had to escape their birth places, or who
never had a homeland. They came to the city and they found neither work nor
shelter. They created their own jobs out of the cynical logics of capitalism,
mostly in refuse recycling. They look ugly to the minorities in power, while
most architects and urban planners unjustly call their shelters "illegal
settlements". Some of them they organise to gain power with solidarity, and
those are the squatters.

During the past decades we have learned to enhance our own autonomy in the urban
contexts\footnote{Lapassade, G. (1971), L'Autogestion pédagogique,
Gauthiers-Villars}, diving across the dif\hbox{}ferent contexts composing the
cities, disclosing the inner structures of their closed networks, developing a
dif\hbox{}ferent texture made of relationships that no company can buy.

We are the \textbf{Weaver Birds}, burung-burung manyar\footnote{Burung-Burung
Manyar means "Weaver Birds" in bahasa indonesia, is a book by Romo Mengun
published in 1992 by Gramedia (Jakarta)}, we share our nests in a network, we
f\hbox{}low as the river of the spontaneous settlement of Code in
Yogyakarta\footnote{the Code riverbank was considered an ``illegal settlement''
of squatters, while Romo Mengun has been active between 1981 and 1986, gathering
the sympathy of intellectuals believing that these poor members of society
should be accepted and helped to improve their living conditions. The government
of Indonesia planned its forced removal in 1983, but as protests followed the
plans were cancelled. Nine years later in 1992 Kampung Code was selected as the
winner of the Aga Khan Award for Architecture in the Muslim World. The Code
riverside settlement continues to exist until this day, as a remarkable example
of urban architecture.}, the gypsy neighbourhood of Sulukule in Instanbul, the
Chaos Computer Club, all the hacklabs across the world, the self-organised
squatters in Amsterdam, Berlin, Barcelona and more, the hideouts of 2600 and all
the other temporary hacker spaces where our future, and your future, is being
homebrewed.

This document is just the start for a new course, revealing an analysis that is
shared among a growing number of young hackers and artists, nourished by their
autonomy and knowledge. Our hacker spaces are quickly proliferating as we do
notneed to build more space as opposed to penetrating existing empty space. We
are highly adaptive and we aim at connecting rather than separating, at being
inclusive rather than exclusive, at being ef\hbox{}fective rather than acquiring
status.


\section{Horizontal media}
\label{s:weaver_birds:horizontal_media}

\begin{quote}
\textit{Whoever controls the media -the images- controls the culture.} (Allen
Ginsberg, 1926-1997)
\end{quote}

Our concern about freedom in media is serious. The current urgency
justif\hbox{}ies all our acts of rebellion, as they have become necessary. One
of our main activities is patiently weaving the threads for open networks that
put us all in contact. But greedy national regimes and criminal organisations
threaten us as if they can avoid revealing their fascist nature, while
opportunist provokers use our open grounds, as if they had been granted the
right to of\hbox{}fend and generate more wars.

About media we certainly accumulated enough knowledge to trace a clear path for
our development, as we have been doing since the early days of our existence. We
are active in implementing the liberties that the digital age grants us. This
intellectual freedom is very important for the development of humanity, for its
capacity to analyse its own actions, to weave its faith in harmony.

Our plan is to keep on developing more on-site and on-line public space for
discussion, following a \textbf{decentralised pattern} that grants access to
most people on our planet. We created tools for independent media, in order to
multiply the voices in protection of common visions, to avoid a few media
tycoons taking over democracies, as is happening in many dif\hbox{}ferent places
of the world.

We are aware of the limits of the present implementation of democracy: while
they are busy celebrating their own success over archaic regimes, these systems
stopped updating their own architecture and have fallen in control of new
enemies which they now cannot even recognise.

The solution we propose is simple: maximise the possibilities to recycle
existing media infrastructures, open as many channels as possible, free the
airwaves, let communication f\hbox{}low in its multiplicity, avoid any
mono-directional use of it, give everyone the possibility to run a radio or TV
station for its own digital and physical neighbours, following an organic
pattern that will modularise the sharing of sense and let ideas propagate in a
horizontal, non- hierarchical way.

If these media architectures are linked with educational models that foster
tolerance we have a hope that they will accelerate the evolution of our planet
and grant protection to the minorities that are populating it.


\section{Freedom of identity}
\label{s:weaver_birds:freedom_identity}

We believe that current governmental ef\hbox{}forts of biometric control by
governments, private data mining operated by companies and public schools
watching over students' activity, prof\hbox{}iling programmes that are targeting
people worldwide are crimes against humanity.

Each of those ef\hbox{}forts is not taking into careful consideration what can
be done when dictatorial regimes take control of such systems. In fact, this
already happened half a century ago when the f\hbox{}irst action of the Nazis
was numbering people and labelling them with a symbol marking their biological
ethnicities (as biometry can nowadays).

Conscious of the lack of responsibility of current governments worldwide, we
will oppose with all means necessary their ef\hbox{}forts to number and control
all people in the name of a safe and unreachable security that, as we hackers
can demonstrate, cannot be enforced by such means.

As hackers we are very conscious of information f\hbox{}lows and how several
leaks in the digital domain are actually disclosing personal information of
large amounts of people worldwide. We believe that people should not be numbered
and included in databases, which probably is what still dif\hbox{}ferentiates
governments from operating systems, merely suppressing the processes that are
not optimised for their tasks.

Our generation includes a large critical mass concerned on these issues, as
proof, see the recent success of \textit{Freedom not Fear}\footnote{Worldwide
protests against surveillance, every 12 October - \url{http://ur1.ca/f6og}},
while an entertaining and poetical description of our feelings is also depicted
in the movie Gattaca\footnote{1997, Directed by Andrew Niccol. With Ethan Hawke,
Uma Thurman, Gore Vidal - \url{http://ur1.ca/f6oh}}.


\section{Education}
\label{s:weaver_birds:education}


\begin{quote}
\textit{Because this New Order of ours is a military order, an authoritarian
order, commando style, there is no education. There is only instruction, a mere
taming experience.} (\textit{Romo Mangun})
\end{quote}

As privatisation of educational structures progresses, the academy assumes a
corporate and business mindset, which assists a shift of the educational mission
in society from \textit{inclusive} to \textit{exclusive}.

The inf\hbox{}luential play of industries has permeated most academical
disciplines, in particular regarding the adoption of technologies. The choice of
educators has become biased by logics of short term prof\hbox{}it, rather than
\textbf{Solid Knowledge}.

On the other hand, notions are rapidly becoming universally available.
\textit{Heuristic}, \textit{maieutic} and \textit{infrastructure} functions
provided by academies are best satisf\hbox{}ied by the global action of the free
software communities' \textbf{horizontal} sharing methods, experiences and
working implementations, on distributed and versioned R\&D platforms.

As components can be combined and redistributed, copied and
modif\hbox{}ied\footnote{following the GNU project philosophy and further
applying to more f\hbox{}ields of human knowledge.} students learn a knowledge
that is durable, without restrictions on their rights to produce and
redistribute creations.  This situation will provide an advantage for new
generations, as it does for developing countries.

Media hubs and hacker spaces constitute a great potential to activate cultural
growth, fulf\hbox{}illing an educational role that is progressively lacking in
higher schools and universities.

In 1998, during the f\hbox{}irst edition of the hackmeeting\footnote{see
\url{http://ur1.ca/f6oi} and the book Networking Art \url{http://ur1.ca/f6oj}
(Costa \& Nolan)\\ ISBN:88-7437-047-4 ISBN:978-88-7437-047-4} in
F\hbox{}irenze, its assembly launched the idea of \textit{independent
universities of hacking}, spawning numerous hacklabs across the networked
cities, with annual meetings that have been taking place until today in various
places in the south of Europe. We believe the results of these initiatives have
been greatly inf\hbox{}luential for our own cultural and technical development,
as they hosted an errant knowledge otherwise dispersed and neglected by the
academies, with the participation of people like Wau Holland, Richard Stallman,
Tetsuo Kogawa, Andy Muller-Magoon, Emmanuel Goldstein and even more collectives
and individuals.

With such a short but intense history behind us we are well motivated to
continue developing our independent paths of knowledge, an auto-didactic
literature that liberates the students from corporate interests and opens up a
horizon of variety and creativity that cannot be envisioned by the most
advanced, yet faulty, implementations of the so called ``creative industries''.


\section{Consolidation}
\label{s:weaver_birds:consolidation}

\begin{quote}
\textit{Inverno. Come un seme il mio animo ha bisogno del lavoro nascosto di
questa stagione.} (Giuseppe Ungaretti, 1888-1970)
\end{quote}

If you have read this far, and you think our plans deserve support, then you
should know that we are really struggling for better quality, a part of our
vision we haven't fully reached yet. That is what we call consolidation.

As our activity mostly focuses on free and open source software development, we
have to admit that we are not yet there, in satisfying all the needs of the
various communities relying on them.

For example, the on-line radio streaming software MuSE\footnote{see
\url{http://ur1.ca/f6ok} - a tool that is well documented for usage by the
f\hbox{}lossmanuals project at \url{http://ur1.ca/f6ol}}, being developed for
eight years now, to provide a user friendly tool for community on-line radio
streaming, and used by various radios worldwide, is not yet fully developed to
the point it should, and we have a hard time in keeping the pace with updating
it.

Another example is the popular GNU/Linux multimedia liveCD
dyne:bolic\footnote{see \url{http://ur1.ca/f6om} - also listed among the few
100\% free distribution by the Free Software Foundation, as well nominated among
the top-10 open source projects in 2005 by the \textit{Independent} UK.} which
has been developed since 2001 and reached version 2.5.2 last Winter. It focuses
on several important issues, such as supporting old hardware, implementing
privacy for users, of\hbox{}fering media production tools and providing all
development tools on its single liveCD. We won't hide that we are experiencing
major problems in keeping the project alive, lacking funds to involve more
developers for such a huge ef\hbox{}fort. In fact, since more recent
"philanthropic" startups (that, considering the nature of their funding, are not
grassroot at all) obscured our long-standing grassroot development, we have been
deprived of the media attention that is also necessary to gather support. This
all follows the logic of the big f\hbox{}ish eating the smaller f\hbox{}ishes,
killing variety even in the open source context.

Yet another example is the FreeJ vision mixer software\footnote{see
\url{http://ur1.ca/f6on}} which has been developed since 2002, implementing an
open platform for producing and broadcasting audio/video online in a completely
open way, also relying on development done by the xiph.org
foundation\footnote{see \url{http://ur1.ca/f6op}}. With FreeJ we hope to
rehabilitate the vast knowledge about the javascript language with a tool that
lets it be used for video production, as a 100\% free alternative to
F\hbox{}lash and other recent commercial startups. The horizon for this project
is very promising, as Ogg/Vorbis/Theora support is f\hbox{}inally being natively
integrated in Mozilla F\hbox{}irefox\footnote{see \url{http://ur1.ca/f6or}}, and
we are actively seeking funding support for a short term development sprint,
which never really arrives.

In economic terms all these projects have been developed with very little
support so far, and actually don't need much to go on. Still, proper expertise
is needed and that, in most cases, requires a budget to keep people committed on
a medium or long term.

What we are seeking for our consolidation is to develop a publication platform
that lets us modestly merchandise these products, keeping them still free and
available online, plus eventually some benefactors trusting our work and
investing their philanthropic instincts in the visions hereby described.
Suggestions regarding possible consolidation paths are very welcome and, of
course, donations are needed\footnote{see \url{http://ur1.ca/f6os}}.


\section{Infrastructure}
\label{s:weaver_birds:infrastructure}

\begin{quote}
\textit{It is best to keep one's own organization intact; to crush the enemy's
organization is only second best.} (Sun Tzu, 6th century BC)
\end{quote}

We are planning (and realising already) a decentralised structure of on-line and
on-site facilities to be independently shared among us.

On-site we successfully link to squats and liminal practices among our networked
cities, developing patterns that can be implemented locally and shared globally.
Re-use of existing empty structures is a crucial point, as it is keeping these
initiatives independent from corporate and national inf\hbox{}luence, freeing
the potential of the various cultures composing them.

On-line we are yet more powerful, having established a redundant network of
servers and protocols that, even if opposed by corporate interests, are
f\hbox{}lourishing and well spread across the populace.

In this phase we are still very young and we need all your support to help us
stay independent, host our ef\hbox{}forts in dif\hbox{}ferent contexts and share
their visibility.

As we have composed a comprehensive cartography of such ef\hbox{}forts, you can
be conf\hbox{}ident that all the economic and practical support contributed will
be carefully shared by all nodes and documented by a growing literature of
examples, facts and periodic reports which will keep all our network informed.\\

\textbf{On site}

So far we are emerging in two locations: the poetry hacklab\footnote{see:
\url{http://ur1.ca/f6ot}} in Palazzolo Acreide, near Siracusa, where we are
struggling to establish a museum of historical working computers\footnote{see:
\url{http://ur1.ca/f6ou}} (also reachable online) as a permanent interactive
exhibition where visitors can experiment with the machines, an educational
ef\hbox{}fort that also implies the preservation of our digital past.

Second is our hacktive squatted community in Amsterdam, a city that is probably
among the last places in the world tolerating the occupation of empty spaces,
resulting in a balanced urban architecture that is open to independent cultural
initiatives and grassroot social movements, helping to control the growing
speculative trend on private properties by business magnates and criminals
white-washing their money.

And next are even more grassroot run places ready to be emerging, with which we
plan to share common plans about sustainability, open source practices and open
spaces for the global and local communities crossing them.\\

\textbf{On line}

The network of servers we are so far relying on is very much resembling our
on-site architecture, where hospitality plays a main role, as several
independent organisations or institutions of\hbox{}fered us hosting space for
our projects, while half of the f\hbox{}leet is hosted on a limited number of
commercial co-locations f\hbox{}inanced by self-taxation.

All software employed is free and open source: servers run stable versions of
Debian GNU/Linux, code development is hosted using Git\footnote{fast and
distributed code versioning system, see: \url{http://ur1.ca/f6ow}}, webpages are
served by a custom written setup (that we plan to evolve following this wheel
spin) using Apache, PHP and Mysql, while whenever possible we use static pages.
Open discussion forums are provided using Mailman, IRC and in future phpBB,
while open publishing and editorial f\hbox{}lows are hosted using the MoinMoin
wiki platform. Most of our facilities are made redundant and, of course, we keep
backups, having preserved so far every single bit composing our digital history.

Besides the dyne.org website itself, we host several artists and activists
engaged in projects as Streamtime\footnote{free blogging from Iraq, see
\url{http://ur1.ca/f6ox}}, Idiki\footnote{a wiki for ideas, see
\url{http://ur1.ca/f6oy}}, ib-arts\footnote{ib\_project for the arts, see
\url{http://ur1.ca/f6p0}}, Morisena\footnote{collaborative art, ecology,
sustainability, summer camps, yoga,\\see: \url{http://ur1.ca/f6p3}} and more,
plus some free independent radios\footnote{see: \url{http://ur1.ca/f6p4}} and,
in future, more TV, as software like FreeJ will soon be ready for it.


\section{Collaboration}
\label{s:weaver_birds:collaboration}

\begin{quote}
\textit{Nadie es patria. Todos lo somos.} (Jorge Luis Borges, 1899-1986)
\end{quote}

Thanks for reading this far. In case we sparked some interest in you with this
document, then f\hbox{}inally let us point out some practical ways to get
involved and collaborate with us.

Being still a young phase of our evolution, we need to carefully economise
participation in our development. So we are looking for talented hackers wishing
to contribute to software development, as well as independent communities
wanting to join our network and amplify our practices and dreams across the
world.

As we will hopefully get some funding (and this phase basically opens our
network to such opportunities) we will not neglect to support your participation
with money. In fact we plan to pay out fees for specif\hbox{}ic development
tasks, as the ones described in the Consolidation chapter, which will be
progressively detailed on our websites.

We also plan to open up residencies and remote stage programmes, in
collaboration with educational institutions recognising our ef\hbox{}forts and
the involvement of their students in them.

Please get in touch\footnote{\url{http://ur1.ca/f6p5}}, then! By specifying your
email address, we will reply to your mail and plan our future collaborations.

This document was drafted by Jaromil in eight years of extensive travels in very
dif\hbox{}ferent contexts around and between Europe and Asia, nourished by
several exchanges along the way and f\hbox{}inally made public on the 8 aAugust
2008. While it is impossible to enumerate all of us and our collective soul, we
still like to say thanks to the following individuals for witnessing the birth
of this document. After eight years it would take too long to thank everyone
involved, so let the people now remind the many others not mentioned: Richard M.
Stallman, Gustaf\hbox{}f Harriman Iskandar, Venzha Christawan, Irene Agrivina,
Timbil Budiarto, Viola van Alphen and Kees de Groot, Elisa Manara, Julian
Abraham, Nancy Mauro-F\hbox{}lude, Gabriele Zaverio: they
witnessed\footnote{except for RMS with whom I had email exchange during those
days, and others who were in connection that day climbing other vulcanoes} the
birth of this document under the Vulcano Merapi, our minds in vibrant exchange
during the Cellsbutton\footnote{Organised by the House of Natural F\hbox{}iber,
\url{http://ur1.ca/f6p7}} festival and Helarfest\footnote{Organised by Common
Room, \url{http://ur1.ca/f6p9}} in Bandung and Yogyakarta.

Thanks, a thousand f\hbox{}lowers will blossom!

