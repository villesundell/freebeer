\begin{savequote}
    \qauthor{\LARGE{Jeremiah Foster}}
\end{savequote}
\chapter{Creating Debian packages from CPAN}
\label{c:debs_cpan}

CPAN is a well-known and useful archive of Perl modules, a pearl in the Perl
world. While it serves many Perl developers and users, it cannot by its very
nature cater for further distribution because it does not know what form that
distribution has to take. In other words, how is cpan supposed to know if it
needs to morph into a specif\hbox{}ic format to allow a module to be installed on a
specif\hbox{}ic platform? It cannot and should not, it should provide instead a stable
API and a distributed database allowing for easy packaging ``downstream'', which
is what it does. One can install from source if one prefers, or with the cpan
and cpanp tools, but sometimes you need or want a more complete and f\hbox{}lexible
system for installing software.

As we move downstream, we get closer to the user and the user's system. Hic sunt
dracones, you need to be pretty careful about how and what you install lest you
create instability and bugs. Cpan tries to handle installation elegantly by
installing dependencies with whatever module you are installing. This is a
``Good Thing\texttrademark'', it helps the end-user immeasurably and helps to
avoid ``dependency hell''; a painful state which describes the situation of
having some of your needed software installed, but not all of it.

Since a cpan module is agnostic to its f\hbox{}inal destination and tries to be as
cross-platform as possible, it will not know about the specif\hbox{}ic peculiarities of
the operating system upon which it is to reside. In fact, one might argue a good
deal of cpants is directed at this problem, determining the quirks of the OS.
Workarounds include the inclusion of multiple operating-system-specif\hbox{}ic tools
and functions, yuck. 

A better solution might be ``package management'' which allows for a cpan module
to be wrapped in a way that allows for simpler installation. This is of course
operating system specif\hbox{}ic and rightly so, the OS needs to  determine how to
install, where to install, and what. So cpan can just do its thing while the OS
communicates directly with cpan, gets the required module(s), any Perl
dependencies, and does the installation work. The OS then checks to see if there
are operating system required dependencies above and beyond the Perl
dependencies, satisf\hbox{}ies those dependencies, resulting in a single call to the
package manager to install software without having to search the internet for
some arbitrary .so f\hbox{}ile.

This article aims to explain this packaging process for Debian and Debian
derived operating systems such as Ubuntu, allowing for Perl modules to be
installed as debs and even submitted to Debian itself. The Debian system has
many users, receives security notif\hbox{}ications, is known for its stability, and
gets regular updates. These are things your Perl modules will automatically get
as well when you submit them to Debian.

There is a dedicated group of Debian hackers, both ``Debian Developers'' and
non-developers, who maintain Perl modules in Debian. I am one of those who works
on the Debian-Perl team\cite{debs_cpan-debian_perl_team} and would like to
describe the development of debs from cpan, including some of its gory details,
so that others can be familiar with ``best practices'' of packaging software for
Debian. 

Let us begin with a tool called dh-make-perl, shall we? Dh-make-perl (the dh
stands for Debian helper) is a wrapper around the cpan tool, plus a whole lot
more. We call it the same way as we would call cpan, with a module name. It then
goes to cpan for the source of our deb because the goal of a deb is to have the
source code separate and pristine. Debian makes no changes to the upstream
source for packaging. Occasionally someone in Debian might patch the source to
f\hbox{}ix a bug, but in Debian-Perl we try to use patch to patch things and always try
to pass our patch upstream at least into RT, Perl's bug tracker. 

Choosing something to package is actually quite important. I will choose
Test::F\hbox{}ile because I f\hbox{}ind it useful and have some familiarity with it - two
things one needs to generate the interest and motivation when there are bug
reports or new features. Packaging is actually considerable work over time, a
stale package is both a potential security risk and quickly forgotten.

Now we use our f\hbox{}irst tool, the powerful dh-make-perl. I will show the call to
dh-make-perl and then go through it a bit since I am going to pass a lot of
arguments just to show some features.

\lstset{
    basicstyle=\footnotesize\ttfamily,
    commentstyle=\textit,
    %numbers=left,
    %numberstyle=\tiny,
    %stepnumber=1,
    %numbersep=5pt,
    breaklines=true,
    tabsize=4,
    keywordstyle=\color{red},
    frame=single,
    showspaces=false,
    showtabs=false,
    xleftmargin=15pt,
    framexleftmargin=9pt,
    framexrightmargin=9pt,
    %backgroundcolor=\color{lightgray},
    showstringspaces=false
}
\begin{lstlisting}[frame=trbl,label=cl:debs_cpan:dh-make-perl,caption=dh-make-perl command]{}
dh-make-perl --cpan Test::File --desc "Test file attributes with perl." --arch all --version 1.25 -e jeremiah@jeremiahfoster.com --dh 7 --requiredeps --build
\end{lstlisting}

We call dh-make-perl with a bunch of parameters.  This of course is not
necessary, you can make your call much smaller, but I want to show some of these
parameters because they make life a little easier and you may want to use them.
Of course the canonical source of dh-make-perl parameters and functions is in
the man page for dh-make-perl, this is good to check on occasion since it has
been getting updated recently\cite{debs_cpan-git_dh_make_perl}.

The f\hbox{}irst parameter, or really argument to dh-make-perl, is the {-}{-}cpan f\hbox{}lag
which tells dh-make-perl to go and get the module from cpan as opposed to
f\hbox{}inding it locally. From the man page: ``If neither {-}{-}cpan nor a directory
is given as argument, dh-make-perl tries to create a Perl package from the data
in .'' i.e. the current directory.  So if you have a module you want to install
locally or for some reason do not want to push up to Debian, you can create
local debs for your own local machines or mirror, no need to push them
downstream as it were.

Next we give the name of our module in the same way we would if we were using
cpan, i.e. Foo::Bar. The {-}{-}desc switch tells dh-make-perl what to use for
Debian's short description and the {-}{-}arch f\hbox{}lag is for the architecture. Here
we are using all because perl works on all the architectures that Debian
of\hbox{}f\hbox{}icially (and unof\hbox{}f\hbox{}icially) supports. 

Shockingly enough the {-}{-}version f\hbox{}lag provides a way to inform dh-make-perl
about the version of the package we are packaging, so this is the  current
version of Test::F\hbox{}ile; -e is the email address f\hbox{}lag, it wants an email address
after it; {-}{-}dh is a call to debhelper itself and after {-}{-}dh you have to
specify the version of debhelper you want to use. This is a little tricky
because dif\hbox{}ferent versions of debhelper create dif\hbox{}ferent artefacts, specif\hbox{}ically
dif\hbox{}ferent debian/rules f\hbox{}iles. So you want most likely to use version 7 for
debhelper. To paraphrase the dh-make-perl man page, {-}{-}dh will set desired
the debhelper version. If ``ver'' is 7, the generated debian/rules f\hbox{}ile is
minimalist, using the auto-mode of debhelper. This minimalist version is what
you want, unless you are going to package an XS module or need to do some crazy
stuf\hbox{}f at build time.

Fortunately we do not have to mess about with our debian/rules f\hbox{}ile, so I am
going to continue discussing the rest of the arguments to dh-make-perl, but I
want to say that there is a great deal to discuss regarding debian/rules and you
would do well to consider reading about it in the Debian developers'
documentation in places like the New Maintainer's
Guide\cite{debs_cpan-new_maint_guide-rules}. If you are reading this in front of
a Debian command line, you can simply do an ``aptitude install maint-guide'' to
get the documentation.

The {-}{-}requiredeps f\hbox{}lag tells dh-make-perl to require Perl dependencies, that
is to say, if we do not f\hbox{}ind all the modules needed to build, we should fail to
build our deb. This is really good because it makes your deb package more
portable and all the Perl module dependencies will get installed when you
install your package on another machine, very convenient. For this call to work
you need to have apt-f\hbox{}ile installed on the machine on which you are building the
package. Apt-f\hbox{}ile is an excellent tool, written in Perl (of course!). It allows
you to search for f\hbox{}iles in Debian packages, even packages that are not installed
on your system. This means that apt-f\hbox{}ile is really the canonical tool to f\hbox{}ind
things in Debian or Ubuntu packages. A quick example: say we wanted to install
libtest-more-perl and we called aptitude to install it thusly, ``aptitude
install libtest-more-perl''. Aptitude says:

\begin{lstlisting}[frame=trbl,label=cl:debs_cpan:apt-install,caption=aptitude install libtest-more-perl output]{}
E: Unable to locate package libtest-more-perl
\end{lstlisting}

But we are certain that this fundamental perl module is in Debian! Haven't we
seen Test::More output in fact? Indeed we have, but this module does not exist
on its own. Debian has included it with the package perl-modules because it is
such a fundamental tool, and so much else in Debian requires it. So looking for
it with ``dpkg -L libtest-more-perl'' will produce these rather unhelpful
results:

\begin{lstlisting}[frame=trbl,label=cl:debs_cpan:dpkg-search,caption=dpkg -L libtest-more-perl output]{}
Package ``libtest-more-perl'' is not installed.
\end{lstlisting}

But in fact, when we search with ``apt-f\hbox{}ile search Test/More.pm'' (which is the
format we need to specify since we are looking at the f\hbox{}ile system) we will f\hbox{}ind
that apt-f\hbox{}ile f\hbox{}inds it for us:

\begin{lstlisting}[frame=trbl,label=cl:debs_cpan:apt-file,caption=apt-file search Test/More.pm output]{}
perl-modules: /usr/share/perl/5.10.0/Test/More.pm
\end{lstlisting}

This output tells us that the f\hbox{}ile Test/More.pm is under /usr/share/perl/5.10.0
and it is in the Debian package perl-modules. This is a handy and reliable way
to f\hbox{}ind if the Perl module you are looking for is already packaged in Debian.
All of these commands were issued on a Debian testing system.

F\hbox{}inally we pass {-}{-}build which ``builds only a binary package (by calling
`fakeroot debian/rules binary') and does not sign the package. It is meant for a
quick local install of a package, not for creating a package ready for
submission to the Debian archive.'' So says the man page for dh-make-perl. I
like to build the package with dh-make-perl because then certain build problems
come to the fore sooner. It is not a requirement to build the package with
dh-make-perl however.

Once we have run dh-make-perl, we watch all sorts of interesting output f\hbox{}ly by,
like output from cpan, the test suite of our module, etc. The debhelper build
process takes over after cpan has worked its magic and we get a f\hbox{}inished two
f\hbox{}iles and a directory when we are done. They are:

\begin{lstlisting}[frame=trbl,label=cl:debs_cpan:dh-make-perl-output,caption=dh-make-perl output]{}
File: libtest-file-perl_1.25_all.deb
File: libtest-file-perl_1.25.orig.tar.gz
Dir: Test-File-1.25
\end{lstlisting}



\section{The anatomy of a package}
\label{s:debs_cpan:anatomy}

You would be tempted to say ``Well I have built my deb, I'm done!'' Doing a dpkg
{-}{-}contents libtest-f\hbox{}ile-perl\_1.25\_all.deb ought to show this output on our new
deb:

\begin{lstlisting}[frame=trbl,label=cl:debs_cpan:dpkg-contents,caption=dpkg {-}{-}contents libtest-file-perl output]{}
drwxr-xr-x root/root         0 2009-02-09 15:39 ./
drwxr-xr-x root/root         0 2009-02-09 15:39 ./usr/
drwxr-xr-x root/root         0 2009-02-09 15:39 ./usr/share/
drwxr-xr-x root/root         0 2009-02-09 15:39 ./usr/share/man/
drwxr-xr-x root/root         0 2009-02-09 15:39 ./usr/share/man/man3/
-rw-r--r-- root/root      4142 2009-02-09 15:39 ./usr/share/man/man3/Test::File.3.gz
drwxr-xr-x root/root         0 2009-02-09 15:39 ./usr/share/perl5/
drwxr-xr-x root/root         0 2009-02-09 15:39 ./usr/share/perl5/Test/
-rw-r--r-- root/root     27027 2008-06-10 19:59 ./usr/share/perl5/Test/File.pm
drwxr-xr-x root/root         0 2009-02-09 15:39 ./usr/share/doc/
drwxr-xr-x root/root         0 2009-02-09 15:39 ./usr/share/doc/libtest-file-perl/
-rw-r--r-- root/root        69 2007-02-09 02:30 ./usr/share/doc/libtest-file-perl/README
-rw-r--r-- root/root      1476 2009-02-09 15:39 ./usr/share/doc/libtest-file-perl/copyright
drwxr-xr-x root/root         0 2009-02-09 15:39 ./usr/share/doc/libtest-file-perl/examples/
-rw-r--r-- root/root        69 2007-02-09 02:30 ./usr/share/doc/libtest-file-perl/examples/README
-rw-r--r-- root/root       164 2009-02-09 15:39 ./usr/share/doc/libtest-file-perl/changelog.gz
drwxr-xr-x root/root         0 2009-02-09 15:39 ./usr/lib/
drwxr-xr-x root/root         0 2009-02-09 15:39 ./usr/lib/perl5/
drwxr-xr-x root/root         0 2009-02-09 15:39 ./usr/lib/perl5/auto/
drwxr-xr-x root/root         0 2009-02-09 15:39 ./usr/lib/perl5/auto/Test/
drwxr-xr-x root/root         0 2009-02-09 15:39 ./usr/lib/perl5/auto/Test/File/
-rw-r--r-- root/root       195 2009-02-09 15:39 ./usr/lib/perl5/auto/Test/File/.packlist
drwxr-xr-x root/root         0 2009-02-09 15:39 ./usr/lib/perl/
drwxr-xr-x root/root         0 2009-02-09 15:39 ./usr/lib/perl/5.10/
-rw-r--r-- root/root       214 2009-02-09 15:39 ./usr/lib/perl/5.10/perllocal.pod
\end{lstlisting}

But in fact we are not done, we need to build the deb with dpkg-buildpackage and
we need to modify some of the f\hbox{}iles in the Debian directory. F\hbox{}irst we will start
by modifying the f\hbox{}iles in the Debian directory to make sure we have a proper
package. The f\hbox{}irst thing we need to do is to change the name of our directory.
Debian has  a requirement that says the package name has to be lowercase which
means that our directory has to be lower case. So we move Test-F\hbox{}ile to
libtest-f\hbox{}ile-perl-1.25. This format is the standard format for Debian Perl
packages. While one might say it is not the most beautiful format, it has its
strengths. Those strengths are that the format informs the user it is a library
package, part of a larger system which might require dependencies. It has the
suf\hbox{}f\hbox{}ix -perl which indicates that it is a Perl library. There are a few modules
in Debian which are not labelled this way, and there is no absolute law saying
you have to call your module this way, but if you do not you are in fact doing
the user a grave disservice, because anyone who is used to Debian or Debian
derivatives will search for a module as libfoo-bar-perl and they will not f\hbox{}ind
your module if it is not so labelled. 

So once we have moved Test-F\hbox{}ile-1.25 to libtest-f\hbox{}ile-perl-1.25 we will change
into that directory and take a look around. We f\hbox{}ind that it is just like the
untarred module from CPAN only with the addition of a Debian directory. We will
take a closer look at the Debian directory now which is at the heart of
packaging. According to the New Maintainer's
guide\cite{debs_cpan-new_maint_guide} ``The most important of them are
`control', `changelog', `copyright' and `rules', which are required for all
packages.'' Let us start by taking a look at the control f\hbox{}ile: 

\lstset{
    basicstyle=\footnotesize\ttfamily,
    commentstyle=\textit,
    numbers=left,
    numberstyle=\tiny,
    stepnumber=1,
    numbersep=5pt,
    breaklines=true,
    tabsize=4,
    keywordstyle=\color{red},
    frame=single,
    showspaces=false,
    showtabs=false,
    xleftmargin=15pt,
    framexleftmargin=9pt,
    framexrightmargin=9pt,
    %backgroundcolor=\color{lightgray},
    showstringspaces=false
}

\begin{lstlisting}[frame=trbl,label=cl:debs_cpan:control-file,caption=control]{}
Source: libtest-file-perl
Section: perl
Priority: optional
Build-Depends: debhelper (>= 7)
Build-Depends-Indep: perl (>= 5.6.0-12), libtest-manifest-perl (>= 1.14)
Maintainer: Debian Perl Group <pkg-perl-maintainers@lists.alioth.debian.org>
Uploaders: Jeremiah C. Foster <jeremiah@jeremiahfoster.com>
Standards-Version: 3.8.0
Homepage: http://search.cpan.org/dist/Test-File/
Vcs-Svn: svn://svn.debian.org/pkg-perl/trunk/libtest-file-perl/
Vcs-Browser: http://svn.debian.org/viewsvn/pkg-perl/trunk/libtest-file-perl/

Package: libtest-file-perl
Architecture: all
Depends: ${perl:Depends}, ${misc:Depends}, libtest-manifest-perl (>= 1.14)
Description: Test file attributes with Perl.
 Test::Files provides a collection of test utilities for file attributes.
 .
 Some file attributes depend on the owner of the process testing the file in
 the same way the file test operators do. For instance, root (or super-user or
 Administrator) may always be able to read files no matter the permissions.
 .
 Some attributes don't make sense outside of Unix, either, so some tests
 automatically skip if they think they won't work on the platform. If you have
 a way to make these functions work on Windows, for instance, please send me a
 patch. :)
 .
 This description was "automagically" extracted from the module by dh-make-perl.
\end{lstlisting}

I will move quickly through the f\hbox{}irst lines of the control f\hbox{}ile but I would like
to point out lines 4 and 5 where Build-Depends and Build-Depends-Indep are
def\hbox{}ined. This is where the magic at the core of aptitude lies, and why the apt
system is so powerful. Here we def\hbox{}ine the relationships between packages in the
operating system and within Perl which will be satisf\hbox{}ied at build time. These
dependencies were calculated by dh-make-perl but there are other mechanisms to
do this as well and sometimes we will even need to do this by hand. Looking in
the source directory for the package and even the META.yml and Makef\hbox{}ile.PL can
reveal dependencies that might otherwise be missed. Usually dh-make-perl gets it
right however and this is not necessary.

In our Build-Depends line we are saying we depend on debhelper and we will not
be able to build our package unless this dependency is satisf\hbox{}ied, it is an
absolute dependency. The apt system will check automatically for dependencies on
your dependencies, so you only specify the dependencies you need for your
package, you do not have to rummage around to f\hbox{}ind out what they depend on.
Build-Depends is only for dependencies required to build a binary package on
your architecture, it is not a complete selection of build-time relationships.
In our package, we also need Build-Depends-Indep which def\hbox{}ines other packages
that our package will need to run, not just to build. 

This is fairly esoteric stuf\hbox{}f, and Perl largely abstracts the ``building'' of
binaries away from the Perl programmer in the interest of simplicity and ease of
use. You can dig into this stuf\hbox{}f if you want, there is much more to learn about
building Perl both on the Perl side and on the Debian side, but since it is a
rather large subject area I am going to gloss over the really hairy details and
refer you to the Debian policy\cite{debs_cpan-debian_policy} and your own Google
prowess to get more info than that I have presented here.

Most of the other stuf\hbox{}f in the debian/control f\hbox{}ile is pretty self-explanatory;
resources for the source code, who was responsible for the package uploading,
etc. I would like to direct you to the last line where we see some packaging
boilerplate which ought to be removed, i.e. line 28.

If we now turn our attention to debian/copyright we can see the power of Free
Software and copyright. The Debian Free Software Guidelines require that a
copyright be assigned so that a licence can be enforced. Perl is under the
Artistic licence, a licence that has won important legal victories in the United
States, and also under the GPL. This dual licensing is ef\hbox{}fective but only when
there is a copyright specif\hbox{}ied and many Perl hackers forget to do this. I would
like to encourage you to document your copyright, even if you received the
copyright by default when you authored new code, this makes it easier to package
your software. Here is what our copyright f\hbox{}ile looks like:

\begin{lstlisting}[frame=trbl,label=cl:debs_cpan:copyright-file,caption=copyright]{}
Format-Specification:
    http://wiki.debian.org/Proposals/CopyrightFormat?action=recall&rev=196
Upstream-Maintainer: brian d foy <bdfoy@cpan.org>
Upstream-Source: http://search.cpan.org/dist/Test-File/
Upstream-Name: Test-File
Disclaimer: This copyright info was automatically extracted 
    from the Perl module. It may not be accurate, so you better 
    check the module sources in order to ensure the module for its 
    inclusion in Debian or for general legal information. Please, 
    if licensing information is incorrectly generated, file a bug 
    on dh-make-perl.

Files: *
Copyright: brian d foy <bdfoy@cpan.org>
License-Alias: Perl
License: Artistic | GPL-1+

Filend: debian/*
Copyright: 2009, Jeremiah C. Foster <jeremiah@jeremiahfoster.com>
Licence: Artistic | GPL-1+

Licence: Artistic
    This program is free software; you can redistribute it and/or modify
    it under the terms of the Artistic Licence, which comes with Perl.
    On Debian GNU/Linux systems, the complete text of the Artistic Licence
    can be found in `/usr/share/common-licences/Artistic'

Licence: GPL-1+
    This program is free software; you can redistribute it and/or modify
    it under the terms of the GNU General Public Licence as published by
    the Free Software Foundation; either version 1, or (at your option)
    any later version.
    On Debian GNU/Linux systems, the complete text of the GNU General
    Public Licence can be found in `/usr/share/common-licences/GPL'
\end{lstlisting}

This f\hbox{}ile is pretty straight-forward. We will remove the boilerplate from lines
6 through 11 and then f\hbox{}ill in the exact date of the copyright for the software,
in this case we'll have to go to cpan and f\hbox{}ind out that it is 2008, but after
that we are done with the copyright f\hbox{}ile.

The compat and watch f\hbox{}iles play minor roles in our package building drama. The
watch f\hbox{}ile is a tool to check to see if there have been any new releases, it
gets used by a tool called uscan which allows one to update a new cpan module
into an existing Debian package quickly. The compat f\hbox{}ile is merely a
``compatibility'' number for some of the other Debian tools, I will leave that
to you to explore. 




\section{Building the package with dpkg-buildpackage}
\label{s:debs_cpan:building}

Now it is time to look at the main build tool for building Perl debs,
dpkg-buildpackage. There are plenty of build tools in Debian and there seems to
be a new one every month. For example there is now one called git-buildpackage
and for all I know it may be great. I like dpkg-buildpackage so that is what I
am going to tell you about.

As with every build tool there are ten thousand options, but I am just going to
describe the juicy parts. I call dpkg-buildpackage like this:

\lstset{
    basicstyle=\footnotesize\ttfamily,
    commentstyle=\textit,
    numbers=none,
    %numberstyle=\tiny,
    %stepnumber=1,
    %numbersep=5pt,
    breaklines=true,
    tabsize=4,
    keywordstyle=\color{red},
    frame=single,
    showspaces=false,
    showtabs=false,
    xleftmargin=15pt,
    framexleftmargin=9pt,
    framexrightmargin=9pt,
    %backgroundcolor=\color{lightgray},
    showstringspaces=false
}
\begin{lstlisting}[frame=trbl,label=cl:debs_cpan:dpkg-buildpackage,caption=dpkg-buildpackage command]{}
dpkg-buildpackage -rfakeroot -D -kjeremiah@jeremiahfoster.com
\end{lstlisting}

What we have right after the call is the f\hbox{}lag -r with the word fakeroot right
after it, that is the command used to gain root. The -D is for checking
conf\hbox{}licts and dependencies which I highly recommend although you can do it
without checking dependencies but that would most likely not be portable.
F\hbox{}inally, -k and my email address is the key I use to sign the package.

This tool is a Perl tool, of course, and if you look at the source you will see
the name Ian Jackson in the copyright section. Ian Jackson is the guy who
started Debian, he is in fact the Ian of Debian with his wife Debra being the
deb part. You can also see that this f\hbox{}ile is not very well documented, no pod
for example, which is a shame. There are other modules also being pulled into
this one, modules like dpkg and dpkg::Version which is useful for checking
version numbers of packages. Why won't you f\hbox{}ind these packages on cpan? Good
question. It is one of my long term goals to expose all these tools to cpan and
get the public to examine them and help with development and documentation. The
developers in Debian seem to think these tools are only relatively interesting
to a Debian developer, which may be true, but I suspect it is valuable to have
tools that work on such a fundamental level with Debian packages since Debian is
so widespread. Then people can either use them themselves or even devise tools
on top of them that might be useful, like the cpan2dist tool in cpanplus. I can
also see these tools as potentially being useful for a distribution agnostic
linux packaging program. In any case, I think Debian should follow the best
practices of the Perl community either way and make the tools available and I
intend to do that work if someone does not beat me to it.

In the meantime, what happened when we built our package? Since we passed -D to
check dependencies, dpkg-buildpackage called dpkg-checkbuilddeps and found that
we cannot build our package because we are missing a dependency; Test::Manifest.
You can run dpkg-checkbuilddeps separately and this is the output:

\begin{lstlisting}[frame=trbl,label=cl:debs_cpan:dpkg-checkbuilddeps,caption=dpkg-checkbuilddeps output]{}
dpkg-checkbuilddeps: Unmet build dependencies: libtest-manifest-perl (>= 1.14)
\end{lstlisting}

The above line tells us that the Perl module Test::Manifest needs to be included
for and that it already exists in Debian as the package libtest-manifest-perl.
Marvel at the power of the apt system! It saved us a journey to dependency hell.
We simply install libtest-manifest-perl and try to build again\ldots

This time, success! Dpkg-buildpackage will ask me for my key passphrase, which I
give it, and it signs the package for me. Now if we look in our dir we have:

\begin{lstlisting}[frame=trbl,label=cl:debs_cpan:post-dpkg-checkbuilddeps,caption=Directory after successful dpkg-checkbuilddeps run]{}
libtest-file-perl-1.25  
libtest-file-perl_1.25-1_all.deb 
libtest-file-perl_1.25-1.dsc  
libtest-file-perl_1.25-1_i386.changes  
libtest-file-perl_1.25-1.tar.gz
\end{lstlisting}

Hooray! We have our deb, signed and sealed. You can install it now with dpkg -i
libtest-f\hbox{}ile-perl\_1.25-1\_all.deb but before we pass it out far and wide, let
us take one f\hbox{}inal step and build it in a ``clean room'' or a minimal Debian
install. This we can use as a baseline and assume that if it builds and installs
here it can build and install anywhere. To do this we are going to use pbuilder
which is a ``personal package builder''. It creates a chroot, downloads a
minimal Debian install, adds your package and any dependencies and builds a deb
for you. If that works, you can be reasonably sure it will work out in the
greater wide world of the Debian installed base.

Here is the call: 


\begin{lstlisting}[frame=trbl,label=cl:debs_cpan:pbuilder-1,caption=pbuilder command]{}
sudo pbuilder build libtest-file-perl_1.25-1.dsc
\end{lstlisting}

I will go through an arbitrary selection of pbuilder's output:

\begin{lstlisting}[frame=trbl,label=cl:debs_cpan:pbuilder-1,caption=pbuilder output]{}
I: using fakeroot in build.
Current time: Wed Feb 11 16:22:37 CET 2009
pbuilder-time-stamp: 1234365757
Building the build Environment
 -> extracting base tarball [/var/cache/pbuilder/base.tgz]
\end{lstlisting}

The base tarball gets unpackaged to create the build environment (f\hbox{}igure
\ref{cl:debs_cpan:pbuilder-1}).

\begin{lstlisting}[frame=trbl,label=cl:debs_cpan:pbuilder-2,caption=pbuilder process continued]{}
Get:1 http://ftp.debian.org sid Release.gpg [189B]
Get:2 http://ftp.debian.org sid Release [80.6kB]
Get:3 http://ftp.debian.org sid/main Packages/DiffIndex [2038B]
Get:4 http://ftp.debian.org sid/main 2009-02-10-2012.30.pdiff [5047B]
\end{lstlisting}

Here (f\hbox{}igure \ref{cl:debs_cpan:pbuilder-2}) pbuilder updates the base Debian
install with the latest dif\hbox{}fs of packages so your clean room is up-to-date. You
can update it manually as well and change the distribution you want to use, I
prefer to use testing but you might want to use stable.

\begin{lstlisting}[frame=trbl,label=cl:debs_cpan:pbuilder-3,caption=pbuilder process continued]{}
Copying source file
    -> copying [libtest-file-perl_1.25-1.dsc]
    -> copying [./libtest-file-perl_1.25-1.tar.gz]
Extracting source
\end{lstlisting}

pbuilder pulls in our source for the package (f\hbox{}igure
\ref{cl:debs_cpan:pbuilder-3}).

\begin{lstlisting}[frame=trbl,label=cl:debs_cpan:pbuilder-4,caption=dpkg-buildpackage takes over]{}
dpkg-buildpackage: source package libtest-file-perl
dpkg-buildpackage: source version 1.25-1
dpkg-buildpackage: source changed by Jeremiah C. Foster <jeremiah@jeremiahfoster.com>
dpkg-buildpackage: host architecture i386
\end{lstlisting}

dpkg-buildpackage takes over and does its stuf\hbox{}f.

\begin{lstlisting}[frame=trbl,label=cl:debs_cpan:pbuilder-5,caption=Test failure!]{}
Test::Manifest::test_harness found [t/load.t t/pod.t t/pod_coverage.t t/normalize.t t/test_files.t t/owner.t t/rt/30346.t]
t/load............ok                                                         
t/pod.............skipped
        all skipped: Test::Pod 1.00 required for testing POD
\end{lstlisting}

Aha! I missed a useful tool. Since Test::Pod gets called while running tests, I
should add it to Build-Depends-Indep in the debian/control f\hbox{}ile to get these
tests to run. Of course it builds without it, but it is better to run all our
tests as the original developer envisioned. Once I add that module and the
module Test::Pod::Coverage which is also used in tests to the debian/control
f\hbox{}ile, all the tests pass and the package gets built. This is a pretty good
indication that this package will build on someone else's machine.

To conf\hbox{}irm that we are in accordance with policy we ought to run the package
through lintian, the Debian policy checker. I run it with the -i and -I f\hbox{}lags
which provides much more verbose output, it has a {-}{-}pedantic switch as well.
We might run it against our deb like this:

\begin{lstlisting}[frame=trbl,label=cl:debs_cpan:lintian-command,caption=lintian command]{}
lintian -i -I libtest-file-perl_1.25-1_all.deb 
\end{lstlisting}

And get output like this:

\begin{lstlisting}[frame=trbl,label=cl:debs_cpan:lintian-output,caption=lintian output]{}
E: libtest-file-perl: perl-module-in-core-directory usr/lib/perl/5.10/
N: 
N:    Packaged modules must not be installed into the core Perl directories as
N:    those directories change with each upstream Perl revision. The vendor
N:    directories are provided for this purpose.
N:    
N:    Refer to Debian Perl Policy section 3.1 (Site Directories) for details.
N:    
N:    Severity: important, Certainty: certain
\end{lstlisting}\cite{debs_cpan-debian_perl_policy}

These warnings are good to have, were you to submit your package for inclusion
in Debian the expectation is that your package is ``lintian clean'' which means
without warnings from lintian. Now we can submit this to Debian or put it in our
own personal deb repo with conf\hbox{}idence.

The package goes through some automatic building on a variety of architectures,
sits in a queue for about ten days, then gets put into the Debian ``testing''
distro. Anyone who has Debian testing sources in the /etc/apt/sources.list will
now be able to install it just by calling aptitude. Now your package or software
is available to millions of users. Congratulations.
